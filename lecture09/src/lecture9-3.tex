\documentclass[aspectration=1610,t]{beamer}

\usepackage{mooc}

\usetikzlibrary{arrows, decorations.markings, shapes}
\usetikzlibrary{positioning}

\title{{\bf Программирование на языке \langcpp\protect\\Лекция
9\protect\vspace{1em}\\}Ассоциативные контейнеры}

\begin{document}
\begin{frame} 
  \titlepage
\end{frame}

\begin{frame}[fragile]{Общие сведения}
Ассоциативные контейнеры делятся на две группы:
\begin{itemize}
\item \textit{упорядоченные} (требуют отношение порядка), 
\item \textit{неупорядоченные} (требуют хеш-функцию).
\end{itemize}
\begin{block}{Общие методы}
\begin{enumerate}
    \item {\tt find} по ключу,
    \item {\tt count} по ключу,
    \item {\tt erase} по ключу.
\end{enumerate}
\end{block}
\end{frame}

\begin{frame}[fragile]{Шаблоны {\tt set} и {\tt multiset}}
\texttt{set} хранит упорядоченное множество (как двоичное дерево поиска).\\ 
Операции добавления, удаления и поиска работают за $O(\log n)$.
Значения, которые хранятся в \texttt{set}, неизменяемые.
\begin{itemize}
    \item {\tt lower\_bound, upper\_bound, equal\_range}.
\end{itemize}
\begin{lstlisting}
#include <set>
\end{lstlisting}\vspace{-2mm}

\begin{lstlisting}
std::set<int> primes = {2, 3, 5, 7, 11};
// дальнейшее заполнение
if (primes.find(173) != primes.end())
    std::cout << 173 << " is prime\n";
    
// std::pair<iterator, bool>
auto res = primes.insert(3);
\end{lstlisting}
В \texttt{multiset} хранится упорядоченное мультимножество.
\vspace{-1mm}
\begin{lstlisting}
std::multiset<int> fib = {0, 1, 1, 2, 3, 5, 8};
// iterator
auto res = fib.insert(13);
// pair<iterator, iterator>
auto eq = fib.equal_range(1);
\end{lstlisting}
\end{frame}

\begin{frame}[fragile]{Шаблоны {\tt map} и {\tt multimap}}
Хранит упорядоченное отображение (как дерево поиска по ключу).\\ 
Операции добавления, удаления и поиска работают за $O(\log n)$.
\begin{lstlisting}
    typedef std::pair<const Key, T> value_type;
\end{lstlisting}\vspace{-1mm}

\begin{itemize}
    \item {\tt lower\_bound,} {\tt upper\_bound,} {\tt equal\_range},
    \item {\tt operator[]}, {\tt at}.
\end{itemize}\vspace{-1mm}
\begin{lstlisting}
#include <map>
\end{lstlisting}\vspace{-1mm}
\begin{lstlisting}
std::map<std::string, int> phonebook;
phonebook.emplace("Marge", 2128506);
phonebook.emplace("Lisa",  2128507);
phonebook.emplace("Bart",  2128507);
// std::map<string,int>::iterator
auto it = phonebook.find("Maggie");
if ( it != phonebook.end())
    std::cout << "Maggie: " << it->second << "\n";
\end{lstlisting}\vspace{-1mm}
\begin{lstlisting}
std::multimap<std::string, int> phonebook;
phonebook.emplace("Homer", 2128506);
phonebook.emplace("Homer", 5552368);
\end{lstlisting}
\end{frame}

\begin{frame}[fragile]{Особые методы {\tt map}: {\tt operator[]} и {\tt at}}
\begin{lstlisting}
auto it = phonebook.find("Marge");
if (it != phonebook.end())
     it->second = 5550123;
else 
    phonebook.emplace("Marge", 5550123);
// или
phonebook["Marge"] = 5550123;
\end{lstlisting}
Метод {\tt \code{operator}[]}:
\begin{enumerate}
    \item работает только с неконстантным map,
    \item требует наличие у {\tt T} конструктора по умолчанию,
    \item работает за $O(\log n)$
    (не стоит использовать {\tt map} как массив).
\end{enumerate}
Метод {\tt at}:
\begin{enumerate}
    \item генерирует ошибку времени выполнения,\\
         если такой ключ отсутствует,
    \item работает за $O(\log n)$.
\end{enumerate}
\end{frame}

\begin{frame}[fragile]{Использование собственного компаратора}
Отношение строгого порядка:
$\neg (x \prec y) \land \neg(y \prec x) \Rightarrow x = y$

\begin{lstlisting}
struct Person { string name; string surname; };

bool operator<(Person const& a, Person const& b) {
    return a.name < b.name || 
          (a.name == b.name && a.surname < b.surname);
}
// уникальны по сочетанию имя + фамилия
std::set<Person> s1;

struct PersonComp {
    bool operator()(Person const& a, 
                    Person const& b) const {
        return a.surname < b.surname;
    }
};
// уникальны по фамилии
std::set<Person, PersonComp> s2;
\end{lstlisting}
\end{frame}


\begin{frame}[fragile]{Шаблоны {\tt unordered\_set} и {\tt unordered\_multiset}}
\texttt{unordered\_set} хранит множество как хеш-таблицу.\\ 
Операции добавления, удаления и поиска работают за $O(1)$ в среднем.
Значения, которые хранятся в \texttt{unordered\_set}, неизменяемые.
\begin{itemize}
	\item {\tt equal\_range}, {\tt reserve},
	\item методы для работы с хеш-таблицей.
\end{itemize}\vspace{-1mm}

\begin{lstlisting}
#include <unordered_set>
\end{lstlisting}\vspace{-2mm}

\begin{lstlisting}
unordered_set<int> primes = {2, 3, 5, 7, 11};
// дальнейшее заполнение
if (primes.find(173) != primes.end())
    std::cout << 173 << " is prime\n";
    
// std::pair<iterator, bool>
auto res = primes.insert(3);
\end{lstlisting}
В \texttt{unordered\_multiset} хранится мультимножество.
\vspace{-1mm}
\begin{lstlisting}
unordered_multiset<int> fib = {0, 1, 1, 2, 3, 5, 8};
// iterator
auto res = fib.insert(13);
\end{lstlisting}
\end{frame}

\begin{frame}[fragile]{Шаблоны {\tt unordered\_map} и {\tt unordered\_multimap}}
Хранит отображение как хеш-таблицу.\\ 
Операции добавления, удаления и поиска работают за $O(1)$ в среднем.

\begin{itemize}
	\item {\tt equal\_range}, {\tt reserve}, {\tt operator[]}, {\tt at},
	\item методы для работы с хеш-таблицей.
\end{itemize}\vspace{-1mm}
\begin{lstlisting}
#include <unordered_map>
\end{lstlisting}\vspace{-1mm}
\begin{lstlisting}
unordered_map<std::string, int> phonebook;
phonebook.emplace("Marge", 2128506);
phonebook.emplace("Lisa",  2128507);
phonebook.emplace("Bart",  2128507);
// unordered\_map<string,int>::iterator
auto it = phonebook.find("Maggie");
if ( it != phonebook.end())
    std::cout << "Maggie: " << it->second << "\n";
\end{lstlisting}\vspace{-1mm}
\begin{lstlisting}
unordered_multimap<std::string, int> phonebook;
phonebook.emplace("Homer", 2128506);
phonebook.emplace("Homer", 5552368);
\end{lstlisting}
\end{frame}

\begin{frame}[fragile]{Использование собственной хеш-функции}
\begin{lstlisting}
struct Person { string name; string surname; };

bool operator==(Person const& a, Person const& b) { 		
    return a.name    == b.name 
        && a.surname == b.surname; 
}

namespace std {
    template <> struct hash<Person> {
        size_t operator()(Person const& p) const {
              hash<string> h;
              return h(p.name) ^ h(p.surname);
        }
    };
}

// уникальны по сочетанию имя + фамилия
unordered_set<Person> s;
\end{lstlisting}
\end{frame}


%\begin{frame}[fragile]{{\tt insert} с подсказкой}
%\begin{lstlisting}
%std::map<K, V> m;
%
%K k = ...;
%V v = ...; 
%
%std::map<K, V>::iterator i = m.find(k); // returns m.end()
%
%std::map<K, V>::iterator hint = m.lower_bound(k);
%if (hint != m.end() && !(k < hint->first))
%    // gotcha!
%else
%    // use hint
%    m.insert(hint, std::make_pair(k, v));
%\end{lstlisting}
%\end{frame}

% TODO: добавить степ про type_index

% TODO: добавить степ про insert с подсказкой

\end{document}
