\documentclass[aspectration=1610,t]{beamer}

\usepackage{mooc}

\title{{\bf Программирование на языке \langcpp\protect\\Лекция
11\protect\vspace{1em}\\}Коллекция библиотек Boost}

\begin{document}
\begin{frame} 
  \titlepage
\end{frame}

\begin{frame}[fragile]{Boost}
    \begin{itemize}
        \item Это коллекция библиотек, расширяющих функциональность C++.

        \pitem Свободно распространяются по лицензии Boost Software License вместе с исходным кодом и документацией на 
        \url{www.boost.org}. 
        
        \pitem Лицензия позволяет использовать boost в коммерческих проектах.

        \pitem Библиотеки из boost являются кандидатами
            на включение\\ в стандарт C++.

        \pitem Некоторые библиотеки boost были включены в стандарты\\ C++ 2011/14 года (\texttt{std::function}, \texttt{std::thread}, \ldots).

        \pitem При включении библиотеки в boost она проходит\\ несколько этапов
            рецензирования.
            
        \pitem Библиотеки boost позволяют обеспечить переносимость.

        \pitem В текущей версии в boost более сотни библиотек.

    \end{itemize}
\end{frame}

\begin{frame}[fragile]{Категории библиотек Boost}
\medskip

    \begin{minipage}{.45\textwidth}
\begin{itemize}
    \item String and text processing
    \item Containers, 
    \item Iterators
    \item Algorithms
    \item Function objects and higher-order programming
    \item Generic Programming
    \item Template Metaprogramming
    \item Concurrent Programming
    \item Math and numerics
    \item Correctness and testing
    \item Data structures
    \item Domain Specific
    \item System
    \end{itemize}
\end{minipage}%
\begin{minipage}{.4\textwidth}
\begin{itemize}
    \item Input/Output
    \item Memory
    \item Image processing
    \item Inter-language support
    \item Language Features Emulation
    \item Parsing
    \item Patterns and Idioms
    \item Programming Interfaces
    \item State Machines
    \item Broken compiler workarounds
    \item Preprocessor Metaprogramming
\end{itemize}
\end{minipage}%
\end{frame}

\end{document}


\begin{frame}[fragile]{{\tt any}}
\small
\begin{lstlisting}
#include <boost/any.hpp>

using boost::any_cast;
typedef std::list<boost::any> many;

void append_int(many & values, int value) {
    boost::any to_append = value;
    values.push_back(to_append);
}
void append_string(many & values, const std::string & value) {
    values.push_back(value);
}
void append_char_ptr(many & values, const char * value) {
    values.push_back(value);
}
void append_any(many & values, const boost::any & value) {
    values.push_back(value);
}
void append_nothing(many & values) {
    values.push_back(boost::any());
}
bool is_string(const boost::any & operand) {
    return any_cast<std::string>(&operand);
}
\end{lstlisting}
\end{frame}

\begin{frame}[fragile]{{\tt assign}}
\small
\begin{lstlisting}
#include <boost/assign/list_inserter.hpp> // for 'insert()'
#include <boost/assert.hpp> 
using namespace std;
using namespace boost::assign;

int main() {
    vector<int> v; 
    v += 1,2,3,4,5,6,7,8,9;

    map<string,int> months;  
    insert( months )
        ( "january",   31 )( "february", 28 )
        ( "march",     31 )( "april",    30 )
        ( "may",       31 )( "june",     30 )
        ( "july",      31 )( "august",   31 )
        ( "september", 30 )( "october",  31 )
        ( "november",  30 )( "december", 31 );

    typedef pair< string,string > str_pair;
    deque<str_pair> deq;
    push_front( deq )( "foo", "bar")( "boo", "far" ); 
}
\end{lstlisting}
\end{frame}

\begin{frame}[fragile]{{\tt function}}
\small
\begin{lstlisting}
boost::function<float (int x, int y)> f;

struct int_div { 
  float operator()(int x, int y) const { return ((float)x)/y; }; 
};

f = int_div();

std::cout << f(5, 3) << std::endl;

boost::function<void (int values[], int n, int& sum, float& avg)>
    sum_avg;

void do_sum_avg(int values[], int n, int& sum, float& avg) {}

sum_avg = &do_sum_avg;

struct X { int foo(int); };

boost::function<int (X*, int)> f;
f = &X::foo;

X x;
f(&x, 5);
\end{lstlisting}
\end{frame}

\begin{frame}[fragile]{{\tt bind}}
\small
\begin{lstlisting}
struct image;

struct animation {
    void advance(int ms);
    bool inactive() const;
    void render(image & target) const;
};

std::vector<animation> anims;

template<class C, class P> void erase_if(C & c, P pred) {
    c.erase(std::remove_if(c.begin(), c.end(), pred), c.end());
}

void update(int ms) {
    std::for_each(anims.begin(), anims.end(), 
        boost::bind(&animation::advance, _1, ms));
    erase_if(anims, boost::mem_fn(&animation::inactive));
}

void render(image & target) {
    std::for_each(anims.begin(), anims.end(), 
        boost::bind(&animation::render, _1, boost::ref(target)));
}
\end{lstlisting}
\end{frame}

\begin{frame}[fragile]{{\tt bind} and {\tt function}}
\small
\begin{lstlisting}
struct button 
{
    boost::function<void()> onClick;
};

struct player
{
    void play();
    void stop();
};

button playButton, stopButton;
player thePlayer;

void connect()
{
    playButton.onClick = boost::bind(&player::play, &thePlayer);
    stopButton.onClick = boost::bind(&player::stop, &thePlayer);
}

\end{lstlisting}
\end{frame}

\begin{frame}[fragile]{{\tt lexical\_cast}}
\small
\begin{lstlisting}
#include <boost/lexical_cast.hpp>
#include <vector>

using boost::lexical_cast;
using boost::bad_lexical_cast;

int main(int /*argc*/, char * argv[]) {
    std::vector<short> args;
    while (*++argv) {
        try {
            args.push_back(lexical_cast<short>(*argv));
        }
        catch(const bad_lexical_cast &) {
            args.push_back(0);
        }
    }
} 

void log_message(const std::string &);

void log_errno(int yoko) {
    log_message("Error " + lexical_cast<std::string>(yoko) 
                         + ": " +  strerror(yoko));
}
\end{lstlisting}
\end{frame}


\begin{frame}[fragile]{{\tt optional}}
\small
\begin{lstlisting}
optional<char> get_async_input() {
    if ( !queue.empty() )
        return optional<char>(queue.top());
    else return optional<char>(); // uninitialized
}

void receive_async_message() {
    optional<char> rcv ;
    // The safe boolean conversion from 'rcv' is used here.
    while ( (rcv = get_async_input()) && !timeout() )
        output(*rcv);
}
////////////////////////////////
optional<string> name ;
if ( database.open() )
    name.reset ( database.lookup(employer_name) ) ;
else
    if ( can_ask_user )
        name.reset ( user.ask(employer_name) ) ;

if ( name )
    print(*name);
else 
    print("employer's name not found!");
\end{lstlisting}
\end{frame}

\begin{frame}[fragile]{{\tt static\_assert}}
\small
\begin{lstlisting}
#include <climits>
#include <limits>
#include <cwchar>
#include <iterator>
#include <boost/static_assert.hpp>
#include <boost/type_traits.hpp>

BOOST_STATIC_ASSERT(std::numeric_limits<int>::digits >= 32);
BOOST_STATIC_ASSERT(WCHAR_MIN >= 0);

template <class RandomAccessIterator >
RandomAccessIterator foo(RandomAccessIterator from,
                         RandomAccessIterator to) {
   // this template can only be used with random access iterators...
   typedef typename std::iterator_traits<
             RandomAccessIterator >::iterator_category cat;
   BOOST_STATIC_ASSERT(
      (boost::is_convertible<
         cat,
         const std::random_access_iterator_tag&>::value));
   //
   // detail goes here...
   return from;
}

\end{lstlisting}
\end{frame}

\begin{frame}[fragile]{String Algo}
\small
\begin{lstlisting}
#include <boost/algorithm/string.hpp>
using namespace std;
using namespace boost;

string str1(" hello world! ");
to_upper(str1);  // str1 == " HELLO WORLD! "
trim(str1);      // str1 == "HELLO WORLD!"

string str2=
   to_lower_copy(
      ireplace_first_copy(
         str1,"hello","goodbye")); // \verb!str2 == "goodbye world!"!

string str3("hello abc-*-ABC-*-aBc goodbye");
typedef vector< iterator_range<string::iterator> > find_vector_type;

find_vector_type FindVec; // \verb!#1: Search for separators!
ifind_all( FindVec, str3, "abc" ); // \verb!{ [abc],[ABC],[aBc] }!

typedef vector< string > split_vector_type;

split_vector_type SplitVec; // \verb!#2: Search for tokens!
split( SplitVec, str3, is_any_of("-*"), token_compress_on ); 
// \verb!{ "hello abc","ABC","aBc goodbye" }!
\end{lstlisting}
\end{frame}

\begin{frame}[fragile]{{\tt variant}}
\small
\begin{lstlisting}
#include "boost/variant.hpp"
#include <iostream>

struct my_visitor : public boost::static_visitor<int>
{
    int operator()(int i) const
    {
        return i;
    }
    
    int operator()(const std::string & str) const
    {
        return str.length();
    }
};

int main() {
    boost::variant< int, std::string > u("hello world");
    std::cout << u; // output: hello world

    int result = boost::apply_visitor( my_visitor(), u );
    std::cout << result; 
    // output: 11 (i.e., length of "hello world")
}
\end{lstlisting}
\end{frame}

\begin{frame}[fragile]{filesystem}
\begin{lstlisting}
int main(int argc, char* argv[])
{
  path p (argv[1]);   // p reads clearer than argv[1] in the following code

  if (exists(p))    // does p actually exist?
  {
    if (is_regular_file(p))        // is p a regular file?   
      cout << p << " size is " << file_size(p) << '\n';

    else if (is_directory(p))      // is p a directory?
      cout << p << "is a directory\n";

    else
      cout << p << 
        "exists, but is neither a regular file nor a directory\n";
  }
  else
    cout << p << "does not exist\n";

  return 0;
}
\end{lstlisting}
\end{frame}

\begin{frame}[fragile]{ASIO}
    \begin{lstlisting}
using boost::asio::ip::tcp; using boost::asio;
std::string make_daytime_string() {
  using namespace std; // For time\_t, time and ctime;
  time_t now = time(0);
  return ctime(&now);
}
int main() {
  try {
    io_service io_service;
    tcp::acceptor acceptor(io_service, tcp::endpoint(tcp::v4(), 13));

    for (;;) {
      tcp::socket socket(io_service);
      acceptor.accept(socket);

      std::string message = make_daytime_string();

      boost::system::error_code ignored_error;
      asio::write(socket, asio::buffer(message), ignored_error);
    }
  }
  catch (std::exception& e) { std::cerr << e.what() << std::endl; }
} 
    \end{lstlisting}
\end{frame}

\begin{frame}[fragile]{ASIO}
\lstset{basicstyle=\fontsize{6pt}{.6em}\ttfamily,  
      keywordstyle=\fontsize{6pt}{.6em}\color{MOOCBlue}\bfseries, 
      commentstyle=\fontsize{6pt}{.6em}\ttfamily\color{MOOCGreen}}
\begin{lstlisting}
using boost::asio::ip::tcp;
int main(int argc, char* argv[]) {
  try {
    boost::asio::io_service io_service;

    tcp::resolver resolver(io_service);
    tcp::resolver::query query(argv[1], "daytime");
    tcp::resolver::iterator endpoint_iterator = resolver.resolve(query);

    tcp::socket socket(io_service);
    boost::asio::connect(socket, endpoint_iterator);

    for (;;) {
      boost::array<char, 128> buf;
      boost::system::error_code error;

      size_t len = socket.read_some(boost::asio::buffer(buf), error);

      if (error == boost::asio::error::eof)
        break; // Connection closed cleanly by peer.
      else if (error)
        throw boost::system::system_error(error); // Some other error.

      std::cout.write(buf.data(), len);
    }
  } catch (std::exception& e) { std::cerr << e.what() << std::endl; }
}
\end{lstlisting}
\end{frame}
\end{document}

