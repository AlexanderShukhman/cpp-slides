\documentclass[aspectration=1610,t]{beamer}

\usepackage{mooc}

\title{{\bf Программирование на языке \langcpp\protect\\Лекция
10\protect\vspace{1em}\\}Гарантии безопасности исключений}

\begin{document}
\begin{frame} 
  \titlepage
\end{frame}

\begin{frame}[fragile]{Гарантии безопасности исключений}
    \begin{block}{Гарантия отсутствия исключений}
            “Ни при каких обстоятельствах функция\\ не будет генерировать исключения”.
    \end{block}\pause
    
    \begin{block}{Базовая гарантия}
            “При возникновении любого исключения\\
            состояние программы останется
            согласованным”.
    \end{block}\pause

    \begin{block}{Строгая гарантия}
            “Если при выполнении операции возникнет исключение,\\ 
            то программа останется том же в состоянии, \\которое было до начала выполнения
            операции”.
    \end{block}
\end{frame}
\begin{frame}[fragile]{Строгая гарантия безопасности исключений}
    \begin{itemize}
        \pitem В каком случае мы не можем обеспечить строгую 
        гарантию безопасности исключений?
            
            \pause \medskip
            \emph{При наличии взаимодействия со внешним окружением.}
            \medskip\pause
        
        \item Как обеспечить строгую гарантию безопасности исключений\\ в остальных случаях?
            
            \pause \medskip
            \emph{Выполнять операцию над копией состояния программы.\\ Если операция прошла успешно, заменить состояние на копию.}
            \medskip\pause
            
        \item Когда можно обеспечить строгую гарантию {\em эффективно}?
        
            \pause \medskip
            \emph{Это вопрос архитектуры приложения.}
            
    \end{itemize}
\end{frame}
\begin{frame}[fragile]{Как добиться строгой гарантии?}
    \begin{lstlisting}
template<class T>
struct Array 
{
    void resize(size_t n) 
    {
        T * ndata = new T[n];
        for (size_t i = 0; i != n && i != size_; ++i)
            ndata[i] = data_[i];

        delete [] data_;
        data_ = ndata;
        size_ = n;    
    }

    T *     data_;
    size_t  size_;
};
    \end{lstlisting}
\end{frame}

\begin{frame}[fragile]{Как добиться строгой гарантии: вручную}
    \begin{lstlisting}
template<class T> 
struct Array 
{
    void resize(size_t n) {
        T * ndata = new T[n];
        try { 
            for (size_t i = 0; i != n && i != size_; ++i)
                ndata[i] = data_[i];
        } catch (...) { 
            delete [] ndata;
            throw;
        }
        delete [] data_;
        data_ = ndata;
        size_ = n;    
    }
    
    T *     data_;
    size_t  size_;
};
    \end{lstlisting}
\end{frame}

\begin{frame}[fragile]{Как добиться строгой гарантии: RAII}
\begin{lstlisting}
template<class T>
struct Array 
{
    void resize(size_t n) {
        unique_ptr<T[]> ndata(new T[n]);

        for (size_t i = 0; i != n && i != size_; ++i)
            ndata[i] = data_[i];

        data_ = std::move(ndata);
        size_ = n;    
    }

    unique_ptr<T[]> data_;
    size_t          size_;
};
\end{lstlisting}
\end{frame}

\begin{frame}[fragile]{Как добиться строгой гарантии: \texttt{swap}}
\begin{lstlisting}
template<class T>
struct Array 
{
    void resize(size_t n) {
        Array t(n);
        for (size_t i = 0; i != n && i != size_; ++i)
            t[i] = data_[i];

        t.swap(*this);
    }

    T       * data_;
    size_t    size_;
};
\end{lstlisting}
\end{frame}


\begin{frame}[fragile]{Проектирование с учётом исключений}
    Рассмотрим традиционный интерфейс стека:
\begin{lstlisting}
template<class T> 
struct Stack 
{
    void push(T const& t) 
    { 
        data_.push_back(t); 
    }

    T pop() 
    {
        T tmp = data_.back();
        data_.pop_back();
        return tmp;
    }

    std::vector<T> data_;
};
\end{lstlisting}
\end{frame}

\begin{frame}[fragile]{Проектирование с учётом исключений}
    Рассмотрим традиционный интерфейс стека:
\begin{lstlisting}
template<class T> 
struct Stack 
{
    void push(T const& t) 
    { 
        data_.push_back(t); 
    }

    void pop(T & res) 
    {
        res = data_.back();
        data_.pop_back();
    }

    std::vector<T> data_;
};
\end{lstlisting}
\end{frame}

\begin{frame}[fragile]{Использование {\tt unique\_ptr}}
\begin{lstlisting}
template<class T> 
struct Stack 
{
    void push(T const& t) 
    { 
        data_.push_back(t); 
    }

    unique_ptr<T> pop() 
    {
        unique_ptr<T> tmp(new T(data_.back()));
        data_.pop_back();
        return tmp;
    }

    std::vector<T> data_;
};
\end{lstlisting}
\end{frame}


%\begin{frame}[fragile]{Проблемы с порядком операций}
%    Следите за порядком операций:
%\begin{lstlisting}
%void foo(unique_ptr<T> p, unique_ptr<V> q);
%\end{lstlisting}
%
%\begin{lstlisting}
%// потенциальный memory leak
%foo(unique_ptr<T>(new T()), unique_ptr<V>(new V()));
%
%// всё корректно
%unique_ptr<T> p(new T());
%foo(std::move(p), unique_ptr<V>(new V()));
%
%// корректно в C++14
%foo(make_unique<T>(), make_unique<V>());
%\end{lstlisting}
%\end{frame}

\begin{frame}[fragile]{Заключение}
    \begin{itemize}
        \item Проектируйте архитектуру приложения с учётом ислючений.
        
        \pitem Функции, не бросающие исключения, нужно \\объявлять
        как \code{noexcept}.
        
        \pitem Все использующие исключения функции \\должны
        обеспечивать как минимум базовую \\гарантию безопасности исключений.
        
        \pitem Там, где это возможно, старайтесь обеспечить\\
        строгую гарантию безопасности исключений.
        
        \pitem Используйте \texttt{swap}, умные указатели и другие
        RAII объекты\\ для обеспечения строгой безопасности исключений.
    \end{itemize}
\end{frame}


\end{document}
