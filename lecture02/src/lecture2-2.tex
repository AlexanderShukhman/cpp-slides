\documentclass{beamer}

\usepackage{mooc}

\usetikzlibrary{arrows, decorations.markings, shapes}
\usetikzlibrary{positioning}

% The face style, can be changed

\title{{\bf Программирование на языке \langcpp\protect\\Лекция
2\protect\vspace{1em}\\}Стек вызовов}

\begin{document}
\begin{frame} 
  \titlepage
\end{frame}

%Использование памяти в работающей программе.
%Стек и куча.

\begin{frame}[fragile]{Стек вызовов}
    \begin{itemize}
        \item Стек вызовов~— это сегмент данных, используемый для хранения локальных
            переменных и временных значений.

        \item Не стоит путать стек с одноимённой структурой данных, у~стека в \langcpp
            можно обратиться к произвольной ячейке.

        \item Стек выделяется при запуске программы.

        \item Стек обычно небольшой по размеру (4Мб).

        \item Функции хранят свои локальные переменные на стеке. 
        
        \item При выходе из функции соответствующая область стека
            объявляется свободной.

        \item Промежуточные значения, возникающие при вычислении
            сложных выражений, также хранятся на стеке.
    \end{itemize}
\end{frame}

\begin{frame}[fragile]{Устройство стека}
\begin{minipage}{.5\textwidth}
\begin{tikzpicture}[
      start chain=1 going below,node distance=-0.15mm
    ]
    \tikzstyle{cell} = [rectangle,minimum width=2cm,draw,thick,on chain=1];
    \tikzstyle{function} = [cell, minimum height=1cm]; 
    \tikzstyle{arrow} = [>=stealth',shorten >=2pt, shorten <=2pt]

    \node [function, fill=green!30]  (main) {main()};
    \node<1,5> [cell, minimum height=45mm, dashed] (empty) {};

    \node<2,4> [function, fill=blue!30]   (foo)  {foo()};
    \node<2,4> [cell, minimum height=35mm, dashed] (empty) {};

    \node<3> [function, fill=blue!30] (foo)  {foo()};
    \node<3> [function, fill=orange!30] (bar)  {bar()};
    \node<3> [cell, minimum height=25mm, dashed] (empty) {};

    \node<6> [function, fill=orange!30] (bar)  {bar()};
    \node<6> [cell, minimum height=35mm, dashed] (empty) {};

    \node [right of=main, xshift=3cm, yshift=8mm] (bottom) {дно стека};
    \path [arrow,->] (bottom) edge node {} (main.north east);

    \node [right of=empty, xshift=3cm, yshift=-8mm] (top) {вершина стека};
    \path [arrow,->] (top) edge node {} (empty.north east);
\end{tikzpicture}
\end{minipage}\hfill
\begin{minipage}{.4\textwidth}
\begin{lstlisting}
void bar( ) {
    int c;
}

void foo( ) {
    int b = 3;
    bar();
}

int main( ) {
    int a = 3;    
    foo();
    bar();

    return 0;
}
\end{lstlisting}
\end{minipage}
\end{frame}

\begin{frame}[fragile]{Вызов функции}
\begin{minipage}{.3\textwidth}
\small
\begin{tikzpicture}[
      start chain=1 going below,node distance=-0.15mm
    ]
    \tikzstyle{cell} = [rectangle,minimum width=2cm, minimum height=5mm,,draw,thick,on chain=1];
    \tikzstyle{function} = [cell, minimum height=1cm]; 
    \tikzstyle{arrow} = [>=stealth',shorten >=2pt, shorten <=2pt]
    \tikzstyle{mbrace} = [decorate,decoration={brace,amplitude=5pt}];

    \node<1-8> [cell, fill=green!30] (x)   {\tt x = 1};
    \node<9>   [cell, fill=red!30]   (x)   {\tt x = 2};
    \node [cell, fill=green!30] (y)   {\tt y = 2};


    \node<2-8> [cell, fill=green!30] (p3)   {\tt false};
    \node<2-8> [cell, fill=green!30] (p2)   {\tt 2};
    \node<2-8> [cell, fill=green!30] (p1)   {\tt 1};

%    \draw[mbrace] 
%        (p3.north east)--(p1.south east) node[midway,anchor=west,xshift=6pt]
%        {аргументы};

    \node<3-6> [cell, fill=green!30] (retv)  {ret val};
    \node<7-8> [cell, fill=red!30]   (retv)  {2};
    \node<3-8> [cell, fill=green!30] (reta)  {ret addr};
    \node<3-8> [cell, fill=green!30] (retr)  {registers};


    %\draw<1-3>[mbrace] 
    %    (x.north east)--(empty.north east) node[midway,anchor=west,xshift=6pt] {main};

%    \node[right of=retv,xshift=21mm] {результат};
%    \node[right of=reta,xshift=15mm] {IP};
%    \node[right of=retr,xshift=21mm] {состояние};


    \node<4-5> [cell, fill=blue!30] (d)   {\tt d};
    \node<4-5> [cell, fill=blue!30] (h)   {\tt h};
    \node<5>   [cell, fill=red!30] (tmp)  {\tt a * b = 2};
    \node<6-7> [cell, fill=blue!30] (d)   {\tt d = 5.42};
    \node<6-7> [cell, fill=blue!30] (h)   {\tt h = 2};


    \node<1,9> [cell, minimum height=55mm, dashed] (empty) {};
    \node<2> [cell, minimum height=40mm, dashed] (empty) {};
    \node<3,8> [cell, minimum height=25mm, dashed] (empty) {};
    \node<4> [cell, minimum height=15mm, dashed] (empty) {};
    \node<5> [cell, minimum height=10mm, dashed] (empty) {};
    \node<6-7> [cell, minimum height=15mm, dashed] (empty) {};

    \node [right of=y,xshift=22mm] (fp) {\parbox{15mm}{frame\\ pointer}};
    
    \path<1-3,8-9> [arrow,->] (fp) edge node {} (x.north east);

    \path<4-7> [arrow,->] (fp) edge node {} (d.north east);

    %\draw<4> [mbrace] 
    %    (d.north east)--(h.south east) node[midway,anchor=west,xshift=8pt] {foo};
                                    
    \node [right of=empty,xshift=22mm] (sp) {\parbox{15mm}{stack\\ pointer}};
    \path [arrow,->] (sp) edge node {} (empty.north east);

\end{tikzpicture}
\end{minipage}\hfill
\begin{minipage}{.6\textwidth}
\begin{lstlisting}
int foo(int a, int b, bool c) 
{
    double d = a * b * 2.71;
    int    h = c ? d : d / 2;
    return h;
}

int main( ) 
{
    int x = 1;
    int y = 2;
    x = foo (x, y, false);
    cout << x;
    return 0;
}
\end{lstlisting}
\end{minipage}
\end{frame}

\begin{frame}[fragile]{Вызов функции}
    \begin{itemize}
        \item При вызове функции на стек складываются:
            \begin{enumerate}
                \item aргументы функции,
                \item адрес возврата,
                \item значение frame pointer и регистров процессора.
            \end{enumerate}

        \item Кроме этого на стеке резервируется место под
            возвращаемое значение.

        \item Параметры передаются в обратном порядке, что позволяет
            реализовать функции с переменным числом аргументов.

        \item Адресация локальных переменных функции и аргументов
            функции происходит относительно frame pointer.

        \item Конкретный процесс вызова зависит от используемых 
            соглашений (cdecl,  stdcall, fastcall, thiscall).
    \end{itemize}
\end{frame}

\end{document}
