\documentclass{beamer}

\usepackage{mooc}

\usetikzlibrary{arrows, decorations.markings, shapes}
\usetikzlibrary{positioning}

% The face style, can be changed

\title{{\bf Программирование на языке \langcpp\protect\\Лекция
2\protect\vspace{1em}\\}Многомерные массивы}

\begin{document}
\begin{frame} 
  \titlepage
\end{frame}

%Использование памяти в работающей программе.
%Стек и куча.

\begin{frame}[fragile]{Многомерные встроенные массивы}
    \begin{itemize}
        \item \langcpp позволяет определять многомерные массивы:
        \begin{lstlisting}
int m2d[2][3] = { {1, 2, 3}, {4, 5, 6} };
for( size_t i = 0; i != 2; ++i ) {
    for( size_t j = 0; j != 3; ++j ) {
        cout << m2d[i][j] << ' ';
    }
    cout << endl;
}
        \end{lstlisting}
    \item Элементы {\tt m2d} располагаются в памяти ``по строчкам''.

    \item Размерность массивов может быть любой, но на практике
        редко используют массивы размерности $> 4$.

\begin{lstlisting}
int m4d[2][3][4][5] = {};
\end{lstlisting}
    \end{itemize}
\end{frame}

\begin{frame}[fragile]{Динамические массивы}
    \begin{itemize}
        \item Для выделения одномерных динамических массивов
            обычно используется оператор \code{new []}.
\begin{lstlisting}
int * m1d = new int[100];
\end{lstlisting}
        \item Какой тип должен быть у указателя на двумерный динамический
            массив?
            \begin{itemize}
                \item Пусть {\tt m}~--- указатель на двумерный массив типа              \code{int}.

                \item Значит {\tt m[i][j]} имеет тип \code{int} (точнее
                    \code{int \&}).

                \item {\tt m[i][j] $\Leftrightarrow$ *(m[i] + j)},
                    т.е. тип {\tt m[i]}~--- \code{int *}.

                \item аналогично, {\tt m[i] $\Leftrightarrow$ *(m + i)},
                    т.е. тип {\tt m}~--- \code{int **}.
            \end{itemize}
            \item Чему соответствует значение {\tt m[i]}?\\ 
                Это адрес строки с номером {\tt i}.

            \item Чему соответствует значение {\tt m}?\\
                Это адрес массива с указателями на строки.
    \end{itemize}
\end{frame}

\begin{frame}[fragile]{Двумерные массивы}
    Давайте рассмотрим создание массива $5\times 4$.
    
\begin{center}
\begin{tikzpicture}[                                                          
      start chain=1 going right, node distance=-0.15mm,
      start chain=2 going right
    ]
    \tikzstyle{cell} = [rectangle,minimum width=4mm, minimum height=4mm,draw];
    \tikzstyle{arrow} = [>=stealth']

    \begin{scope}[shift={(4cm,35mm)}]
    \foreach \i in {1,2,3,4} {
        \node[cell,on chain=2] (m\i-1) {};   
            \foreach \j/\k in {{2/1},{3/2},{4/3},{5/4}}
                \node[cell,below of=m\i-\k,yshift=-7mm] (m\i-\j) {};
    };
   
    \node[on chain=2] (t0) {{\tt m[0]}};   
    \foreach \i/\k in {{1/0},{2/1},{3/2},{4/3}}
        \node[below of=t\k,yshift=-7mm] (t\i) {{\tt m[\i]}};
    \end{scope}

    \node[on chain=1] (m) {{\tt m}};
    \foreach \i in {1,2,3,4,5} {
        \node[cell,on chain=1] (m\i) {};
        \path [arrow,->] (m\i.center) edge node {} (m1-\i.west);     
    };
\end{tikzpicture}
        \begin{lstlisting}
int ** m = new int * [5];
for (size_t i = 0; i != 5; ++i)
    m[i] = new int[4];
        \end{lstlisting}
\end{center}
\end{frame}

\begin{frame}[fragile]{Двумерные массивы}
    Выделение и освобождение двумерного массива размера $a\times b$.
        \begin{lstlisting}
int ** create_array2d(size_t a, size_t b) {
    int ** m = new int *[a];
    for (size_t i = 0; i != a; ++i)
        m[i] = new int[b];
    return m;
}

void free_array2d(int ** m, size_t a, size_t b) {
    for (size_t i = 0; i != a; ++i)
        delete [] m[i];
    delete [] m;
}
        \end{lstlisting}
При создании массива оператор \code{new} вызывается $(a + 1)$ раз.
\end{frame}

\begin{frame}[fragile]{Двумерные массивы: эффективная схема}
    Рассмотрим эффективное создание массива $5\times 4$.
    
\begin{center}
\begin{tikzpicture}[                                                          
      start chain=1 going right, node distance=-0.15mm,
      start chain=2 going right
    ]
    \tikzstyle{cell} = [rectangle,minimum width=4mm, minimum height=4mm,draw];
    \tikzstyle{arrow} = [>=stealth']
    \tikzstyle{mbrace} = [decorate,decoration={brace,amplitude=5pt}];

    \begin{scope}[shift={(0cm,25mm)}]
    \foreach \i in {0,1,2,3,4} {
        \node[cell,on chain=2] (m\i-1) {};
        \node[above of=m\i-1, shift={(6mm,7mm)}] (t\i) {{\tt\small m[\i]}};
         \foreach \j in {2,3,4}
             \node[cell,on chain=2] (m\i-\j) {};
        \draw[mbrace] 
            (m\i-1.north west)--(m\i-4.north east) node[midway,anchor=west,xshift=6pt] {};
    };
    \end{scope}

    \node[on chain=1] (m) {{\tt m}};
    \foreach \i in {0,1,2,3,4} {
        \node[cell,on chain=1] (m\i) {};
        \path [arrow,->] (m\i.center) edge node {} (m\i-1.south west);     
    };
\end{tikzpicture}
        \begin{lstlisting}
int ** m = new int * [5];
m[0] = new int[5 * 4];
for (size_t i = 1; i != 5; ++i)
    m[i] = m[i - 1] + 4;
        \end{lstlisting}
\end{center}
\end{frame}

\begin{frame}[fragile]{Двумерные массивы: эффективная схема}
    Эффективное выделение и освобождение двумерного массива размера $a\times b$.

    \begin{lstlisting}
int ** create_array2d(size_t a, size_t b) {
    int ** m = new int *[a];
    m[0] = new int[a * b];
    for (size_t i = 1; i != a; ++i)
        m[i] = m[i - 1] + b;
    return m;
}

void free_array2d(int ** m, size_t a, size_t b) {
    delete [] m[0];
    delete [] m;
}
        \end{lstlisting}
При создании массива оператор \code{new} вызывается 2 раза.
\end{frame}

\end{document}
