\documentclass[aspectration=1610,t]{beamer}

\usepackage{mooc}

\title{{\bf Программирование на языке \langcpp\protect\\Лекция
7\protect\vspace{1em}\\}Указатели на функции}

\begin{document}
\begin{frame} 
  \titlepage
\end{frame}

\begin{frame}[fragile]{Указатели на функции}
    Кроме указателей на значения в \langcpp присутствуют три особенных 
    типа указателей:
    \begin{enumerate}
        \item указатели на функции (унаследовано из \texttt{C}),
        \item указатели на методы,
        \item указатели на поля классов.
    \end{enumerate}
    
    \medskip\pause
    Указатели на функции (и методы) используются для
    \begin{enumerate}
        \item параметризации алгоритмов,
        \item обратных вызовов (callback),
        \item подписки на события (шаблон Listener),
        \item создания очередей событий/заданий.
    \end{enumerate}
\end{frame}

\begin{frame}[fragile]{Указатели на функции: параметризация алгоритмов}
    \begin{lstlisting}
void qsort (void* base, size_t num, size_t size,
         int (*compar)(void const*,void const*));
    \end{lstlisting}
    \begin{lstlisting}
int doublecmp(void const * a, void const * b) 
{ 
    double da = *static_cast<double const*>(a);
    double db = *static_cast<double const*>(b);
    if (da < db) return -1;
    if (da > db) return 1;
    return 0;
}

void sort(double * p, double * q) 
{
    qsort(p, q - p, sizeof(double), &doublecmp);
}
    \end{lstlisting}
\end{frame}

\begin{frame}[fragile]{Указатели на функции: параметризация алгоритмов}
Упростим предыдущий пример и сделаем его типобезопасным:
    \begin{lstlisting}
void sort(int * p, int * q, bool (*cmp)(int, int)) 
{
    for (int * m = q; m != p; --m)
        for (int * r = p; r + 1 < m; ++r)
         //  if ( *(r + 1) < *r )
            if ( cmp(*(r + 1), *r) )
                swap(*r, *(r + 1));
}

bool less   (int a, int b) { return a < b; }

bool greater(int a, int b) { return a > b; }

void sort(int * p, int * q, bool asc = true)
{
    sort(p, q, asc ? &less : &greater);
}
    \end{lstlisting}
\end{frame}

\begin{frame}[fragile]{О полезности \texttt{typedef}}
Что здесь объявлено?
    \begin{lstlisting}
char * (*func(int, int))(int, int, int *, float);
    \end{lstlisting}
\pause

Функция двух целочисленных параметров, возвращающая указатель на функцию,
которая возвращает указатель на \code{char} и имеет собственный список
формальных параметров вида: \lstinline{(int, int, int *, float)}

\medskip\pause

Как стоило это написать:
    \begin{lstlisting}
typedef char* (*MyFunction)(int,int,int*,float);

MyFunction func(int, int);
    \end{lstlisting}
\end{frame}

\end{document}



