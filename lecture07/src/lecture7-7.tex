\documentclass[aspectration=1610,t]{beamer}

\usepackage{mooc}

\title{{\bf Программирование на языке \langcpp\protect\\Лекция
7\protect\vspace{1em}\\}Пространства имён}

\begin{document}
\begin{frame} 
  \titlepage
\end{frame}

\begin{frame}[fragile]{Пространства имён}
    \emph{Пространства имён (namespaces)}~--- это способ разграничения областей видимости
    имён в \langcpp.
\medskip

    \pause Имена в \langcpp:
    \begin{enumerate}
        \item имена переменных и констант,
        \item имена функций,
        \item имена структур и классов,
        \item имена шаблонов,
        \item синонимы типов (typedef-ы),
        \item enum-ы и union-ы,
        \item имена пространств имён.
    \end{enumerate}           
\end{frame}

\begin{frame}[fragile]{Примеры}
    В \langc для избежания конфликта имён используются префиксы.\newline
    К примеру,  имена в библиотеке Expat начинаются с
    \texttt{XML\_}.
    \begin{lstlisting}
struct XML_Parser;
int XML_GetCurrentLineNumber(XML_Parser * parser);
    \end{lstlisting}
\medskip

\pause В \langcpp это можно было бы записать так:
    \begin{lstlisting}
namespace XML {
    struct Parser;
    int GetCurrentLineNumber(Parser * parser);
}    \end{lstlisting}

Тогда полные имена структуры и функции будут соответственно:  \texttt{XML::Parser} и \texttt{XML::GetCurrentLineNumber}.
\end{frame}


\begin{frame}[fragile]{Описание пространства имён}
\small
    \begin{enumerate}
        \item Пространства имён могут быть вложенными:
    \begin{lstlisting}
namespace items { namespace food { 
    struct Fruit {...};
}}
items::food::Fruit apple("Apple");
    \end{lstlisting}
         \pause\item Определение пространств имён можно разделять:
    \begin{lstlisting}
namespace weapons { struct Bow { ... }; }
namespace items   { 
    struct Scroll { ... };
    struct Artefact  { ... };  
}
namespace weapons { struct Sword { ... }; }
    \end{lstlisting}
\pause\item Классы и структуры определяют одноимённый \texttt{namespace}.
    \end{enumerate}
\end{frame}

\begin{frame}[fragile]{Доступ к именам}
    \begin{enumerate}
        \item Внутри того же \texttt{namespace} все имена доступны напрямую.

        \pause\item \texttt{NS::} позволяет обратиться внутрь пространства
            имён \texttt{NS}.
    \begin{lstlisting}
namespace NS { int foo() { return 0; } }
int i = NS::foo();
    \end{lstlisting}

        \pause\item Оператор \texttt{::} позволяет обратиться к \emph{глобальному пространству имён}.
    \begin{lstlisting}
struct Dictionary {...};

namespace items 
{
    struct Dictionary {...};

    ::Dictionary globalDictionary;
}
    \end{lstlisting}
    \end{enumerate}
\end{frame}

\begin{frame}[fragile]{Поиск имён}
    \emph{Поиск имён}~--- это процесс разрешения имени.

    \begin{enumerate}
        \item Если такое имя есть в текущем \texttt{namespace}
        \begin{itemize}
        \item выдать
                    \emph{все} одноимённые сущности в текущем \texttt{namespace}.
        \item завершить поиск.
        \end{itemize} 
        
        \item Если текущий \texttt{namespace}~--- глобальный
        \begin{itemize}
                \item завершить поиск и выдать ошибку.
                \end{itemize}

        \item Текущий \texttt{namespace} $\gets$ родительский \texttt{namespace}.

        \item Перейти на шаг 1.
    \end{enumerate}
\end{frame}

\begin{frame}[fragile]{Поиск имён}
    \begin{lstlisting}
int foo(int i) { return 1; } 

namespace ru 
{
    int foo(float f) { return 2; }

    int foo(double a, double b) { return 3; }

    namespace spb {
        int global = foo(5);
    }
}                          
    \end{lstlisting}
    \medskip\textbf{Важно:} поиск продолжается до первого совпадения.\\
    В перегрузке участвуют только найденные к этому \\моменту функции.
\end{frame}

\begin{frame}[fragile]{Ключевое слово \code{using}}
Существуют два различных использования слова \code{using}.
    \begin{lstlisting}
namespace ru 
{
    namespace msk {
        int foo(int i) { return 1; }
        int bar(int i) { return -1; }
    }

    using namespace msk; // все имена из msk
    using msk::foo;      // только msk::foo
    
    int foo(float f) { return 2; }
    int foo(double a, double b) { return 3; }
    
    namespace spb {
        int global = foo(5);
    }
}                          
    \end{lstlisting}
\end{frame}

\begin{frame}[fragile]{Поиск Кёнига}{}
    \begin{lstlisting}
namespace cg {
    struct Vector2 {...};
    Vector2 operator+(Vector2 a, Vector2 const& b);
}
    \end{lstlisting}\pause
    \begin{lstlisting}
    cg::Vector2 a(1,2);
    cg::Vector2 b(3,4);
    b = a + b; // эквивалентно: b = operator+(a, b)
    b = cg::operator+(a, b); // OK
    \end{lstlisting}
\pause\medskip
\begin{block}{Argument-dependent name lookup (ADL, Поиск Кёнига)}
При поиске имени функции на первой фазе рассматриваются\\ имена из 
текущего пространства имён {\em и пространств имён,\\ к которым 
принадлежат аргументы функции}.
\end{block}
\end{frame}


\begin{frame}[fragile]{Безымянный namespace}
    Пространство имён с гарантированно уникальным именем.
    \begin{lstlisting}
namespace { // безымянный namespace
    struct Test { std::string name; };
}
    \end{lstlisting}
Это эквивалентно:
    \begin{lstlisting}
namespace $GeneratedName$ {
    struct Test { std::string name; };
}
using namespace $GeneratedName$;
    \end{lstlisting}

\medskip
Безымянные пространства имён~--- замена для \code{static}.
\end{frame}

\begin{frame}[fragile]{Заключение}
    \begin{enumerate}
        \pause\item Используйте пространства имён для исключения конфликта имён.
        
        \pause\item Помните, что поиск имён прекращается после первого совпадения.
            Используйте \code{using} и полные имена.        

        \pause\item Не используйте \code{using namespace} в заголовочных файлах.

        \pause\item Всегда определяйте операторы в том же пространстве имён, что и
            типы, для которых они определены.

        \pause\item Используйте безымянные пространства имён для маленьких локальных классов и как замену слова \code{static}.

        \pause\item Для длинных имён \code{namespace}-ов используйте синонимы:
    \begin{lstlisting}
namespace csccpp17 = ru::spb::csc::cpp17;
    \end{lstlisting}
    \end{enumerate}
\end{frame}

\end{document}



