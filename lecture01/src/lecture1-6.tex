\documentclass{beamer}

\usepackage{mooc}

\usetikzlibrary{arrows, decorations.markings, shapes}
\usetikzlibrary{positioning}

\title{{\bf Программирование на языке \langcpp\protect\\Лекция
1\protect\vspace{1em}\\}Введение в синтаксис \langcpp}

\begin{document}
\begin{frame} 
  \titlepage
\end{frame}

\begin{frame}[fragile]{Типы данных}
\begin{itemize}
    \item Целочисленные:
        \begin{enumerate}
            \item \code{char} (символьный тип данных)
            \item \code{short int}
            \item \code{int}
            \item \code{long int}
        \end{enumerate}
    Могут быть беззнаковыми (\code{unsigned}).
    \begin{itemize}
        \item $-2^{n-1}\ldots (2^{n-1} - 1)$ ($n$~--- число бит)
        \item $0\ldots (2^{n} - 1)$ для $\code{unsigned}$
    \end{itemize}

    \item Числа с плавающей точкой:
        \begin{enumerate}
            \item \code{float}, 4 байта, 7 значащих цифр.
            \item \code{double}, 8 байт, 15 значащих цифр.
        \end{enumerate}
    \item Логический тип данных \code{bool}.
        
    \item Пустой тип \code{void}.
\end{itemize}
\end{frame}

\begin{frame}[fragile]{Литералы}
\begin{itemize}
    \item Целочисленные:
        \begin{enumerate}
            \item \code{'a'}~--- код буквы 'a', тип \code{char},
            \item \code{42} ~--- все целые числа по умолчанию типа \code{int},
            \item \code{1234567890L}~--- суффикс '\code{L}' соответствует типу \code{long},
            \item \code{1703U}~--- суффикс '\code{U}' соответствует типу \code{unsigned int},
            \item \code{2128506UL}~--- соответствует типу \code{unsigned long}.
        \end{enumerate}

    \item Числа с плавающей точкой:
        \begin{enumerate}
            \item \code{3.14}~--- 
                все числа с точкой по умолчанию типа \code{double},
            \item \code{2.71F}~--- суффикс '\code{F}' соответствует типу 
                \code{float},
            \item \code{3.0E8}~--- соответствует $3.0\cdot 10^{8}$.
        \end{enumerate}
    \item \code{true} и \code{false}~--- значения типа \code{bool}.

    \item Строки задаются в двойных кавычках: \verb!"Text string"!.

\end{itemize}
\end{frame}

\begin{frame}[fragile]{Переменные}
\begin{itemize}
    \item При определении переменной указывается её тип. При определении можно
        сразу задать начальное значение (инициализация).
    \begin{lstlisting}
int     i = 10;
short   j = 20;
bool    b = false;

unsigned long l = 123123;

double x = 13.5, y = 3.1415;
float z;
    \end{lstlisting}

    \item Нужно всегда инициализировать переменные.

    \item Нельзя определить переменную пустого типа \code{void}.
\end{itemize}
\end{frame}

\begin{frame}[fragile]{Операции}
\begin{minipage}{.59\textwidth}
    \begin{itemize}
        \item Оператор присваивания: \code{=}.
        \item Арифметические:
        \begin{enumerate}
            \item бинарные: \code{+ - * / \%},
            \item унарные:  \code{++ {-}{-}}.
        \end{enumerate}
        \item Логические:
        \begin{enumerate}
            \item бинарные:  \code{\&\& ||},
            \item унарные:   \code{!}. 
        \end{enumerate}
        \item Сравнения: \code{== != > < >= <=}.
        \item Приведения типов: \code{(type)}.
        \item Сокращённые версии бинарных операторов:
            \code{+= -= *= /= \%=}.
    \end{itemize}
\end{minipage}\hfill
\begin{minipage}{.4\textwidth}
\begin{lstlisting}
int i = 10;
i = (20 * 3) % 7;

int k = i++;
int l = --i;

bool b = !(k == l);

b = (a == 0) ||
    (1 / a < 1); 

double d = 3.1415;
float f = (int)d;

// d = d * (i + k)
d *= i + k;
\end{lstlisting}
\end{minipage}
\end{frame}

\begin{frame}[fragile]{Инструкции}
    \begin{itemize}
        \item Выполнение состоит из последовательности {\em инструкций}.

        \item Инструкции выполняются одна за другой.

        \item Порядок вычислений внутри инструкций не определён.
            \begin{lstlisting}
/* unspecified behavior */
int i = 10;
i = (i += 5) + (i * 4); 
            \end{lstlisting}
        \item Блоки имеют вложенную область видимости:
            \begin{lstlisting}
int k = 10;                
{   
    int k = 5 * i;  // не видна за пределами блока
    i = (k += 5) + 5; 
}
k = k + 1;
            \end{lstlisting}

    \end{itemize}
\end{frame}

\begin{frame}[fragile]{Условные операторы}
\begin{itemize}    
    \item Оператор \code{if}:
    \begin{lstlisting}
int d = b * b - 4 * a * c;
if ( d > 0 ) {
    roots = 2;
} else if ( d == 0 ){  
    roots = 1;
} else {
    roots = 0;
}
    \end{lstlisting}

    \item Тернарный условный оператор:
    \begin{lstlisting}
int roots = 0;
if (d >= 0)
    roots = (d > 0 ) ? 2 : 1;
    \end{lstlisting}
\end{itemize}
\end{frame}

\begin{frame}[fragile]{Циклы}
\begin{itemize}    
    \item Цикл \code{while}:
    \begin{lstlisting}
int squares = 0;
int k = 0; 
while ( k < 10 ) {
    squares += k * k;
    k = k + 1;
} 
    \end{lstlisting}

    \item Цикл \code{for}:
    \begin{lstlisting}
for ( int k = 0; k < 10; k = k + 1 ) {
    squares += k * k;
} 
    \end{lstlisting}
\item Для выхода из цикла используется оператор \code{break}.
\end{itemize}
\end{frame}

\begin{frame}[fragile]{Функции}
\begin{itemize}
    \item В сигнатуре функции указывается тип возвращаемого значений и типы параметров.

    \item Ключевое слово \code{return} возвращает значение.
    \begin{lstlisting}
double square(double x) {
    return x * x;
}
    \end{lstlisting}

    \item Переменные, определённые внутри функций,~— {\em локальные}.
    
    \item Функция может возвращать \code{void}.

    \item Параметры передаются по значению (копируются).

    \begin{lstlisting}
void strange(double x, double y) {
    x = y;
}
    \end{lstlisting}

\end{itemize}
\end{frame}

\begin{frame}[fragile]{Макросы}
\begin{itemize}
    \item Макросами в \langcpp называют инструкции препроцессора.

    \item Препроцессор \langcpp является самостоятельным языком,
        работающим с произвольными строками.

    \item Макросы можно использовать для определения функций:
    \begin{lstlisting}
int max1(int x, int y) { 
    return x > y ? x : y;
}

#define max2(x, y)   x > y ? x : y

a = b + max2(c, d);      //  b + c > d ? c : d;
    \end{lstlisting}

    \item Препроцессор ``не знает'' про синтаксис \langcpp.
    \end{itemize}
\end{frame}

\begin{frame}[fragile]{Макросы}
\begin{itemize}
    \item Параметры макросов нужно оборачивать в скобки:
    \begin{lstlisting}
#define max3(x, y)   ((x) > (y) ? (x) : (y))
    \end{lstlisting}

    \item Это не избавляет от всех проблем:
    \begin{lstlisting}
int a = 1;
int b = 1;
int c = max3(++a, b); 
// c = ((++a) > (b) ? (++a) : (b))
    \end{lstlisting}

    \item Определять функции через макросы~--- плохая идея.

    \item Макросы можно использовать для условной компиляции:
        \begin{lstlisting}
#ifdef DEBUG
    // дополнительные проверки
#endif
        \end{lstlisting}

\end{itemize}
\end{frame}

\begin{frame}[fragile]{Ввод-вывод}
\begin{itemize}    
    \item Будем использовать библиотеку \code{<iostream>}.
    \begin{lstlisting}
#include <iostream>
using namespace std;
    \end{lstlisting}

    \item Ввод:
    \begin{lstlisting}
    int a = 0;
    int b = 0;
    cin >> a >> b;
    \end{lstlisting}

    \item Вывод:
    \begin{lstlisting}
    cout << "a + b = " << (a + b) << endl;
    \end{lstlisting}
\end{itemize}
\end{frame}

\begin{frame}[fragile]{Простая программа}
    \begin{lstlisting}
#include <iostream>
using namespace std;

int main () 
{
    int a = 0;
    int b = 0;

    cout << "Enter a and b: ";
    cin >> a >> b;

    cout << "a + b = " << (a + b) << endl;

    return 0;
}
    \end{lstlisting}
\end{frame}
\end{document}

