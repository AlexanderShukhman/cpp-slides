\documentclass[aspectration=1610]{beamer}

\usepackage{mooc}

\title{{\bf Программирование на языке \langcpp\protect\\Лекция 1\protect\vspace{1em}\\}История языка \langcpp}

\begin{document}
\begin{frame} 
  \titlepage
\end{frame}


\begin{frame}[fragile]{Язык \langc}
    \begin{itemize}
        \item Язык программирования \langcpp создан на основе языка \langc.

        \item 
            Язык программирования \langc разработан в начале 1973 года в 
            компании Bell Labs {\em Кеном Томпсоном} и {\em Деннисом Ритчи}.

        \item            
            Язык \langc был создан для использования в операционной системе UNIX.

        \item
            В связи с успехом UNIX язык \langc получил широкое
            распространение. 
            
        \item 
            На данный момент \langc является одним из самых распространённых языков
            программирования\\ (доступен на большинстве платформ).
                
        \item \langc~--- основной язык для низкоуровневой разработки.
    \end{itemize}
\end{frame}

\begin{frame}{Особенности \langc}
    \begin{itemize}
        \item {\bf Эффективность.}\\
            Язык С позволяет писать программы, которые напрямую работают с железом. 

        \item {\bf Стандартизированность.}\\
            Спецификация языка \langc является международным стандартом.
        
        \item {\bf Относительная простота.}\\
            Стандарт языка \langc занимает 230 страниц\\
            (против 670 для \texttt{Java} и 1340 для \langcpp).
    \end{itemize}
\end{frame}

\begin{frame}{Создание \langcpp}
    \begin{itemize}
        \item Разрабатывается с начала 1980-х годов.
            
        \item Создатель~--- сотрудник Bell Labs {\em Бьёрн Страуструп}.
            
        \item Изначально это было расширение языка \langc для поддержки работы с 
            классами и объектами.
            
        \item Это позволило проектировать программы на более высоком уровне
            абстракции.
            
        \item Ранние версии языка назывались ``\langc with classes''.

        \item Первый компилятор \texttt{cfront}, перерабатывающий исходный код
            ``\langc с классами'' в исходный код на \langc. 
    \end{itemize}
\end{frame}

\begin{frame}{Развитие \langcpp}
    \begin{itemize}
        \item  К 1983 году в язык было добавлено много новых возможностей
            (виртуальные функции, перегрузка функций и операторов, ссылки, 
            константы, \dots)

        \item Получившийся язык перестал быть просто дополненной
            версией классического \langc и был переименован из ``\langc с классами'' в \langcpp. 
        
        \item Имя языка, получившееся в итоге, происходит от оператора 
            унарного постфиксного инкремента \langc{} '{\tt ++}' (увеличение значения переменной на единицу). 
            
        \item Язык также не был назван \texttt{D}, поскольку ``является расширением \langc
            и не пытается устранять проблемы путём удаления элементов \langc''.
        
        \item Язык начинает активно развиваться. Появляются новые компиляторы
             и среды разработки.
    \end{itemize}
\end{frame}

\begin{frame}{Стандартизация \langcpp}
    \begin{itemize}
        \item Лишь в 1998 году был ратифицирован международный стандарт языка 
            \langcpp: ISO/IEC 14882:1998 ``Standard for the \langcpp Programming
            Language''.
        
        \item В 2003 году был опубликован стандарт языка ISO/IEC 14882:2003, где были
            исправлены выявленные ошибки и недочёты предыдущей версии стандарта.

        \item В 2005 году был выпущен Library Technical Report 1 (TR1). 
            
        \item С 2005 года началась работа над новой версией стандарта, которая
            получила кодовое название C++0x. 
            
        \item В конце концов в 2011 году стандарт был принят и получил название
            C++11 ISO/IEC 14882:2011. 

        \item В данный момент ведётся одновременная работа над двумя версиями
        стандарта: C++14 и C++17.
    \end{itemize}
\end{frame}

\begin{frame}{Совместимость \langc и \langcpp}

    \begin{itemize}
        \item Один из принципов разработки стандарта \langcpp~— это
            сохранение совместимости с \langc. 
            
        \item Синтаксис \langcpp унаследован от языка \langc. 
            
        \item \langcpp не является в строгом смысле надмножеством \langc.
            
        \item Можно писать программы на \langc так, чтобы они успешно
            компилировались на \langcpp.
            
        \item \langc и \langcpp сильно отличаются как по сложности, так и по принятым 
            архитектурным решениям, которые используются в обоих языках.
    \end{itemize}
\end{frame}
\end{document}



