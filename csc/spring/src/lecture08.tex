\documentclass{beamer}
\usepackage{csc}
\title{STL: алгоритмы}

\date{
   \textbf{CS центр}\\
   27 марта 2018 \\
   Санкт-Петербург
}

\lstset{basicstyle=\fontsize{8pt}{0.85em}\ttfamily,  
        keywordstyle=\fontsize{8pt}{0.85em}\color{MOOCBlue}\bfseries, 
        commentstyle=\fontsize{8pt}{0.85em}\ttfamily\color{MOOCGreen}}

\begin{document}
\begin{frame} 
  \titlepage
\end{frame}

\begin{frame}[fragile]{Функторы и min/max алгоритмы}
\begin{itemize}
    \item \emph{Функтор}~--- класс, объекты которого ведут себя как функции,
            т.е. имеет перегруженные \code{operator}\texttt{()}.
    \item\emph{Предикат}~--- функтор, возвращающий \code{bool}.
\end{itemize}
\pause
Функторы в стандартной библиотеке:
\begin{itemize}
    \item {\tt less, greater, less\_equal, greater\_equal, not\_equal\_to, equal\_to},
    \item {\tt minus, plus, divides, modulus, multiplies},
    \item {\tt logical\_not, logical\_and, logical\_or}
    \item {\tt bit\_and, bit\_or, bit\_xor},
    \item {\tt hash}.
\end{itemize}
\pause
\begin{block}{Алгоритмы min/max}
    \begin{itemize}\tt
    \item min, max, minmax,
    \item min\_element, max\_element, minmax\_element.
    \end{itemize}
\end{block}

\end{frame}

\begin{frame}[fragile]{Немодифицирующие алгоритмы}
\begin{itemize}
    \tt
    \item all\_of, any\_of, none\_of,
    \item for\_each,
    \item find, find\_if, find\_if\_not, find\_first\_of,
    \item adjacent\_find,
    \item count, count\_if,
    \item equal, mismatch,
    \item is\_permutation,
    \item lexicographical\_compare,
    \item search, search\_n, find\_end.
\end{itemize}    

\textbf{Для упорядоченных последовательностей}
\begin{itemize}
    \tt
    \item lower\_bound, upper\_bound, equal\_range,
    \item set\_intersection, set\_difference,\\set\_union, set\_symmetric\_difference,
    \item binary\_search, includes.
\end{itemize}
\end{frame}

\begin{frame}[fragile]{Примеры}
\begin{lstlisting}
vector<int> v = {2,3,5,7,13,17,19};
size_t c = count_if(v.begin(), v.end(),
                    [](int x) {return x % 2 == 0;});
                    
auto it = lower_bound(v.begin(), v.end(), 11);

bool has7 = binary_search(v.begin(), v.end(), 7);
\end{lstlisting}
\begin{lstlisting}
vector<string> & db = getNames();

for_each(db.begin(), db.begin() + db.size() / 2, 
        [](string & s){cout << s << "\n";});

auto w = find(db.begin(), db.end(), "Waldo");

string agents[3] = {"Alice", "Bob", "Eve"};
auto it = find_first_of(db.begin(), db.end(),
                        agents, agents + 3);
\end{lstlisting}
\end{frame}


\begin{frame}[fragile]{Модифицирующие алгоритмы}
\begin{itemize}
    \tt
    \item fill, fill\_n, generate, generate\_n,
    \item random\_shuffle, shuffle,
    \item copy, copy\_n, copy\_if, copy\_backward,
    \item move, move\_backward,
    \item remove, remove\_if, remove\_copy, remove\_copy\_if,
    \item replace, replace\_if, replace\_copy, replace\_copy\_if,
    \item reverse, reverse\_copy,
    \item rotate, rotate\_copy,
    \item swap\_ranges,  
    \item transform,
    \item unique, unique\_copy,
    \item[\textbf{\textasteriskcentered}] accumulate, adjacent\_difference,\\ inner\_product, partial\_sum, iota.
\end{itemize}
\end{frame}

\begin{frame}[fragile]{Примеры}
\begin{lstlisting}
// случайныe
vector<int> a(100); 
generate(a.begin(), a.end(), [](){return rand() % 100;});

// 0,1,2,3,...
vector<int> b(a.size()); 
iota(b.begin(), b.end(), 0);

// c[i] = a[i] * b[i]
vector<int> c(b.size()); 
transform(a.begin(), a.end(), b.begin(), 
          c.begin(), multiplies<int>());

// c[i] *= 2 
transform(c.begin(), c.end(), c.begin(), 
          [](int x) {return x * 2;});

// сумма c[i]
int sum = accumulate(c.begin(), c.end(), 0);
\end{lstlisting}
\end{frame}

\begin{frame}[fragile]{Удаление элементов из последовательности}
\begin{lstlisting}
vector<int> v = {2,5,1,5,8,5,2,5,8};
remove(v.begin(), v.end(), 5);
\end{lstlisting}
Как изменится \texttt{v.size()}?\quad \only<2->{Не изменится.}\\ 
Какое содержимое вектора \texttt{v}?\quad \only<2->{\texttt{\{2,1,8,2,8,5,2,5,8\}}}
\pause\pause\\\medskip
Удаление элемента по значению:
\begin{lstlisting}
vector<int> v = {2,5,1,5,8,5,2,5,8};
v.erase(remove(v.begin(), v.end(), 5), v.end());
\end{lstlisting}
\begin{lstlisting}
list<int> l = {2,5,1,5,8,5,2,5,8};
l.remove(5);
\end{lstlisting}
\pause
Удаление одинаковых элементов:
\begin{lstlisting}
vector<int> v = {1,2,2,2,3,4,5,5,5,6,7,8,9};
v.erase(unique(v.begin(), v.end()), v.end());
\end{lstlisting}
\begin{lstlisting}
list<int> l = {1,2,2,2,3,4,5,5,5,6,7,8,9};
l.unique();
\end{lstlisting}
\end{frame}


\begin{frame}[fragile]{Удаление из ассоциативных контейнеров}
Неправильный вариант
\begin{lstlisting}
map<string, int> m;
for (auto it = m.begin(); it != m.end(); ++it)
    if (it->second == 0)
        m.erase(it);
\end{lstlisting}
\pause
Правильный вариант
\begin{lstlisting}
for (auto it = m.begin() ; it != m.end(); )
    if (it->second == 0) 
        it = m.erase(it);
    else
        ++it;
\end{lstlisting}
\pause
Альтернативный вариант (для старого стандарта)
\begin{lstlisting}
for (map<string,int>::iterator it = m.begin(); it != m.end(); )
    if (it->second == 0) 
        m.erase(it++);
    else                 
        ++it;
\end{lstlisting}
\end{frame}

% TODO:: сделать из этого текстовый степ
\begin{frame}[fragile]{Модифицирующие алгоритмы: предикаты}
\begin{lstlisting}
struct ElementN 
{
    ElementN(size_t n) : n(n), i(0) {}

    template<class T>
    bool operator()(T const& t) 
    { 
        return (++i == n); 
    }
    
    size_t n;
    size_t i;
};

vector<int> v = {0,1,2,3,4,5,6,7,8,9,10};

v.erase(remove_if(v.begin(), v.end(), ElementN(3)), v.end());
\end{lstlisting}
\end{frame}

\begin{frame}[fragile]{Модифицирующие алгоритмы: предикаты}
\begin{lstlisting}
template<class Iterator, class Pred>
Iterator remove_if(Iterator p, Iterator q, Pred pred) {
    Iterator s = find_if(p, q, pred);
    if (s == q)
        return q;

    Iterator out = s++;
    return remove_copy_if(s, q, out, pred);
}    

template<class Iterator, class OutIterator, class Pred>
Iterator remove_copy_if(Iterator p, Iterator q, 
                        OutIterator out, Pred pred) {
    for (; p != q; ++p)
        if (!pred(*p))
            *out++ = *p;
    return out;
}
\end{lstlisting}
\end{frame}

\begin{frame}[fragile]{Сортировка}
\begin{itemize}
    \tt
    \item is\_sorted, is\_sorted\_until,
    \item sort, stable\_sort,
    \item nth\_element, partial\_sort,
    \item merge, inplace\_merge,
    \item partition, stable\_partition, is\_partitioned, partition\_copy, partition\_point.
\end{itemize}
\pause
\begin{lstlisting}
vector<int> v = randomVector<int>();

auto med = v.begin() + v.size() / 2;
nth_element(v.begin(), med, v.end());
cout << "Median: " << *med;

auto m = partition(v.begin(), v.end(), 
         [](int x){return x % 2 == 0;});
sort(v.begin(), m);
v.erase(m, v.end());        
\end{lstlisting}
\end{frame}

\begin{frame}[fragile]{Что есть ещё?}
\begin{itemize}
    \item Операции с кучей:
        \begin{itemize}
            \tt 
            \item push\_heap, 
            \item pop\_heap, 
            \item make\_heap, 
            \item sort\_heap
            \item is\_heap,
            \item is\_heap\_until.
            
        \end{itemize}
    \item Операции с неинициализированными интервалами:
        \begin{itemize}
            \tt
            \item raw\_storage\_iterator,
            \item uninitialized\_copy,
            \item uninitialized\_fill,
            \item uninitialized\_fill\_n.
        \end{itemize}
    \item Операции с перестановками
        \begin{itemize}
            \tt 
            \item next\_permutation,
            \item prev\_permutation.
        \end{itemize}
\end{itemize}
\end{frame}

\end{document}
