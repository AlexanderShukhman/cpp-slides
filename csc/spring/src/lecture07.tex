\documentclass{beamer}
\usepackage{csc}
\title{STL: итераторы и умные указатели}

\date{
   \textbf{CS центр}\\
   20 марта 2018 \\
   Санкт-Петербург
}

\lstset{basicstyle=\fontsize{8pt}{0.85em}\ttfamily,  
        keywordstyle=\fontsize{8pt}{0.85em}\color{MOOCBlue}\bfseries, 
        commentstyle=\fontsize{8pt}{0.85em}\ttfamily\color{MOOCGreen}}

\begin{document}
\begin{frame} 
  \titlepage
\end{frame}

\begin{frame}[fragile]{Категории итераторов}
\small
    \emph{Итератор}~— объект для доступа к элементам
    последовательности, синтаксически похожий на указатель.

    Итераторы делятся на пять категорий. 
    \begin{itemize}
        \item Random access iterator: {\tt ++, \verb!--!}, арифметика, {\tt <, >, <=, >=}.\\
        (\texttt{array, vector, deque})

        \item Bidirectional iterator: {\tt ++, \verb!--!}.\\
        (\texttt{list, set, map})

        \item Forward iterator: {\tt ++}.\\
        (\texttt{forward\_list, unordered\_set, unordered\_map})

        \item Input iterator: {\tt ++}, read-only.

        \item Output iterator: {\tt ++}, write-only.
    \end{itemize}

    Функции для работы с итераторами:
\begin{lstlisting}
    void   advance  (Iterator & it, long n);
    size_t distance (Iterator f, Iterator l);
    void   iter_swap(Iterator i, Iterator j);
\end{lstlisting}
\end{frame}

\begin{frame}[fragile]{{\tt iterator\_traits}}
\begin{lstlisting}
// заголовочный файл <iterator>
template <class Iterator>
struct iterator_traits {
   typedef difference_type   Iterator::difference_type;
   typedef value_type        Iterator::value_type;
   typedef pointer           Iterator::pointer;
   typedef reference         Iterator::reference;
   typedef iterator_category Iterator::iterator_category;
};

template <class T>
struct iterator_traits<T *> {
    typedef difference_type   ptrdiff_t;
    typedef value_type        T;
    typedef pointer           T*;
    typedef reference         T&;
    typedef iterator_category 
            random_access_iterator_tag;
};
\end{lstlisting}
\end{frame}
 
\begin{frame}[fragile]{{\tt iterator\_category}}
\begin{lstlisting}
// <iterator>
struct random_access_iterator_tag {};
struct bidirectional_iterator_tag {};
struct forward_iterator_tag {};
struct input_iterator_tag {};
struct output_iterator_tag {};
\end{lstlisting}
\begin{lstlisting}
template<class I>
void advance_impl(I & i, long n, random_access_iterator_tag) { 
    i += n; 
}
template<class I>
void advance_impl(I & i, size_t n, ... ) { 
   for (size_t k = 0; k != n; ++k, ++i );
}
template<class I>
void advance(I & i, size_t n) {
   advance_impl(i, n, typename 
     iterator_traits<I>::iterator_category());
}
\end{lstlisting}
\end{frame}

\begin{frame}[fragile]{{\tt reverse\_iterator}}
    У некоторых контейнеров есть обратные итераторы: 
\begin{lstlisting}
list<int> l = { 0, 1, 2, 3, 4, 5, 6, 7, 8, 9 }; 
// list<int>::reverse\_iterator
for(auto i = l.rbegin(); i != l.rend(); ++i)
    cout << *i << endl;
\end{lstlisting}

Конвертация итераторов:
\begin{lstlisting}
list<int>::iterator i = l.begin();
advance(i, 5); // i указывает на 5
// ri указывает на 4
list<int>::reverse_iterator ri(i); 
i = ri.base();
\end{lstlisting}

Есть возможность сделать обратный итератор из random access\\ или bidirectional при помощи шаблона 
\texttt{reverse\_iterator}.
\begin{lstlisting}
// <iterator>
template <class Iterator> 
class reverse_iterator {...};
\end{lstlisting}
\end{frame}


\begin{frame}[fragile]{Инвалидация итераторов}
    Некоторые операции над контейнерами делают существующие итераторы
    некорректными (\emph{инвалидация} итераторов).
    \bigskip
    
    \begin{enumerate}
        \pitem Удаление делает некорректным итератор на удалённый элемент в любом контейнере.
        \pitem В \texttt{vector} и \texttt{string}
            добавление потенциально инвалидирует все итераторы (может произойти выделение нового буфера), иначе инвалидируются только итераторы на все следующие элементы.
        \pitem В \texttt{vector} и \texttt{string} удаление элемента инвалидирует итераторы на все следующие элементы.
        \pitem В \texttt{deque} удаление/добавление инвалидирует все итераторы, кроме 
            случаев удаления/добавления первого или последнего элементов.
        
    \end{enumerate}
\end{frame}
\begin{frame}[fragile]{Advanced итераторы}
Для пополнения контейнеров:\\
{\tt back\_inserter, front\_inserter, inserter}.
\begin{lstlisting}
// в классе Database
    template<class OutIt>
    void findByName(string name, OutIt out);
\end{lstlisting}
\begin{lstlisting}
// размер заранее неизвестен
vector<Person> res; 
Database::findByName("Rick", back_inserter(res));
\end{lstlisting}
Для работы с потоками: \\{\tt istream\_iterator, ostream\_iterator}.
\begin{lstlisting}
ifstream file("input.txt");
vector<double> v((istream_iterator<double>(file)), 
                  istream_iterator<double>());
    
copy(v.begin(), v.end(), 
     ostream_iterator<double>(cout, "\n"));
\end{lstlisting}
\end{frame}

\begin{frame}[fragile]{Как написать свой итератор}
\begin{lstlisting}
// <iterator>
template 
<class Category, // iterator::iterator\_category
 class T,        // iterator::value\_type
 class Distance = ptrdiff_t,// iterator::difference\_type
 class Pointer = T*,        // iterator::pointer
 class Reference = T&       // iterator::reference
> class iterator;
\end{lstlisting}

\begin{lstlisting}
#include <iterator>

struct PersonIterator 
    : std::iterator<forward_iterator_tag, Person>
{
// operator++, operator*, ... 
};
\end{lstlisting}
\end{frame}

\begin{frame}[fragile]{Умные указатели}
\texttt{unique\_ptr}
\begin{itemize}
    \item Умный указатель с уникальным владением.
    \item Нельзя копировать, можно перемещать.
    \item Не подходит для разделяемых объектов.
\end{itemize}
\texttt{shared\_ptr}
\begin{itemize}
    \item Умный указатель с подсчётом ссылок.
    \item Универсальный указатель.
\end{itemize}
\texttt{weak\_ptr}
\begin{itemize}
    \item Умный указатель с для создания \emph{слабых ссылок}.
    \item Работает вместе с \texttt{shared\_ptr}.
\end{itemize}
\end{frame}

\end{document}


