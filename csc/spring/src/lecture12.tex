\documentclass{beamer}
\usepackage{csc}
\title{Метапрограммирование}

\date{
   \textbf{CS центр}\\
   24 апреля 2018 \\
   Санкт-Петербург
}

\lstset{basicstyle=\fontsize{8pt}{0.85em}\ttfamily,  
        keywordstyle=\fontsize{8pt}{0.85em}\color{MOOCBlue}\bfseries, 
        commentstyle=\fontsize{8pt}{0.85em}\ttfamily\color{MOOCGreen}}

\newcommand{\fakeitem}{\par{\color{MOOCBlue}\textbullet}\ \ }

\begin{document}
\begin{frame} 
  \titlepage
\end{frame}

\begin{frame}[fragile]{Метапрограммирование}
\begin{itemize}
    \item Метапрограммированием называют создание программ, \\которые порождают другие программы.
    
    \pitem Шаблоны \langcpp можно рассматривать как функциональный\\ язык
    для метапрограммирования.
    
    \pitem Метапрограммы \langcpp позволяют оперировать типами, \\шаблонами и значениями целочисленных типов.
    
    \pitem Метапрограммирование в \langcpp можно применять\\ для широкого круга задач:
    \begin{itemize}
        \item целочисленные compile-time вычисления,
        \item compile-time проверка ошибок,
        \item условная компиляция,
        \item генеративное программирование,
        \item \ldots
    \end{itemize}
    
    \pitem Для метапрограммирования существуют целые\\ библиотеки,
    например, MPL из boost.
\end{itemize}
\end{frame}

\begin{frame}[fragile]{Метафункции}
Метафункция~--- это шаблонный класс, 
который определяет \\имя типа \texttt{type} или целочисленную константу \texttt{value}.
\begin{itemize}
\item Аргументы метафункции~--- это аргументы шаблона.
\item Возвращаемое значение~--- это \texttt{type} или \texttt{value}.
\end{itemize}

Метафункции могут возвращать типы:
\begin{lstlisting}
template<typename T> 
struct AddPointer
{
    using type = T *; 
};
\end{lstlisting}
и значения целочисленных типов:
\begin{lstlisting}
template<int N> 
struct Square
{
    static int const value = N * N; 
};
\end{lstlisting}
\end{frame}


\begin{frame}[fragile]{Вычисления в compile-time}
\begin{lstlisting}
template<int N> 
struct Fact 
{
    static int const
        value = N * Fact<N - 1>::value;
};

template<> 
struct Fact<0> 
{
    static int const value = 1;
};

int main() 
{
    std::cout << Fact<10>::value << std::endl;
}
\end{lstlisting}

(Это вычисление можно реализовать через \code{constexpr}.) 
\end{frame}

%template<int N> struct Fib {
%    constexpr static int 
%        value = Fib<N - 1>::value + Fib<N - 2>::value;
%};
%template<> struct Fib<0> {
%    static const int value = 0;
%};
%template<> struct Fib<1> {
%    static const int value = 1;
%};
%int main() {
%    Fib <10>::value << std::endl;
%}

\begin{frame}[fragile]{Определение списка}
Шаблоны позволяют определять алгебраические типы данных.
\begin{lstlisting}
// определяем список
template <typename ... Types>
struct TypeList; 

// специализация по умолчанию
template <typename H, typename... T>
struct TypeList<H, T...> 
{
    using Head = H;
    using Tail = TypeList<T...>;
};

// специализация для пустого списка
template <>
struct TypeList<> { };
\end{lstlisting}
\end{frame}

\begin{frame}[fragile]{Длина списка}
\begin{lstlisting}
// вычисление длины списка
template<typename TL>
struct Length
{
    static int const value = 1 +
        Length<typename TL::Tail>::value;
};

template<>
struct Length<TypeList<>>
{
    static int const value = 0;
};

int main()
{
    using TL = TypeList<double, float, int, char>;
    std::cout << Length<TL>::value << std::endl;
    return 0;
}
\end{lstlisting}
\end{frame}

\begin{frame}[fragile]{Операции со списком}
\begin{lstlisting}
// добавление элемента в начало списка
template<typename H, typename TL>
struct Cons;

template<typename H, typename... Types>
struct Cons<H, TypeList<Types...>>
{
    using type = TypeList<H, Types...>;
};

// конкатенация списков
template<typename TL1, typename TL2>
struct Concat;

template<typename... Ts1, typename... Ts2>
struct Concat<TypeList<Ts1...>, TypeList<Ts2...>>
{
    using type = TypeList<Ts1..., Ts2...>;
};
\end{lstlisting}
\end{frame}

\begin{frame}[fragile]{Вывод списка}
\begin{lstlisting}
// вывод списка в поток os
template<typename TL>
void printTypeList(std::ostream & os)
{
    os << typeid(typename TL::Head).name() << '\n';
    printTypeList<typename TL::Tail>(os);
};

// вывод пустого списка
template<>
void printTypeList<TypeList<>>(std::ostream & os) {}

int main()
{
    using TL = TypeList<double, float, int, char>;
    printTypeList<TL>(std::cout);
    return 0;
}
\end{lstlisting}
\end{frame}

\begin{frame}[fragile]{Генерация классов}
\begin{lstlisting}
struct A { 
    void foo() {std::cout << "struct A\n";} 
};
struct B { 
    void foo() {std::cout << "struct B\n";} 
};
struct C { 
    void foo() {std::cout << "struct C\n";} 
};

using Bases = TypeList<A, B, C>;
\end{lstlisting}
\pause\begin{lstlisting}
template<typename TL>
struct inherit;

template<typename... Types>
struct inherit<TypeList<Types...>> : Types... {};

struct D : inherit<Bases> { };
\end{lstlisting}
\end{frame}


\begin{frame}[fragile]{Генерация классов}
\begin{lstlisting}
struct D : inherit<Bases> 
{
    void foo() { foo_impl<Bases>(); } 

    template<typename L> void foo_impl();
};
template<typename L> 
inline void D::foo_impl()
{
    // приводим this к указателю на базу из списка
    static_cast<typename L::Head *>(this)->foo();
    
    // рекурсивный вызов для хвоста списка
    foo_impl<typename L::Tail>();
}
template<> 
inline void D::foo_impl<TypeList<>>() 
{
}
\end{lstlisting}
\end{frame}

%\begin{frame}[fragile]{CRTP: Polymorphic copy construction}
%\begin{lstlisting}
%// Base class has a pure virtual function for cloning
%struct Shape {
%    virtual ~Shape() {}
%    virtual Shape *clone() const = 0;
%};

%// This CRTP class implements clone() for Derived
%template <typename Derived>
%struct Shape_CRTP : public Shape {
%    virtual Shape *clone() const {
%        return new Derived(static_cast<Derived const&>(*this));
%    }
%};
% 
%// Nice macro which ensures correct CRTP usage
%#define Derive_Shape_CRTP(Type) struct Type: Shape_CRTP<Type>
% 
%// Every derived class inherits from Shape\_CRTP instead of Shape
%Derive_Shape_CRTP(Square) {};
%Derive_Shape_CRTP(Circle) {};
%\end{lstlisting}
%\end{frame}

\begin{frame}[fragile]{Как определить наличие метода?}
\begin{lstlisting}
struct A {  void foo() { std::cout << "struct A\n"; } };
struct B {  }; // нет метода foo()
struct C {  void foo() { std::cout << "struct C\n"; } };
\end{lstlisting}

\begin{lstlisting}
template<typename L> 
inline void D::foo_impl() 
{
    // приводим this к указателю на базу из списка
    static_cast<typename L::Head *>(this)->foo();
    
    // рекурсивный вызов для хвоста списка
    foo_impl<typename L::Tail>();
}
\end{lstlisting}
\end{frame}

\begin{frame}[fragile]{Как проверить наличие родственных связей?}
\small
\begin{lstlisting}
typedef char YES;
struct NO { YES m[2]; };

template<class B, class D>
struct is_base_of
{
    static YES test(B * );
    static NO  test(...);
    
    static bool const value = 
        sizeof(YES) == sizeof(test((D *)0));
};

template<class D> 
struct is_base_of<D, D> 
{
    static bool const value = false;
};
\end{lstlisting}
\end{frame}

\begin{frame}[fragile]{SFINAE}
 SFINAE = Substitution Failure Is Not An Error.\\
 Ошибка при подстановке шаблонных параметров \\
 не является сама по себе ошибкой.
\begin{lstlisting}
// ожидает, что у типа T определён 
// вложенный тип value\_type
template<class T>
void foo(typename T::value_type * v);

// работает с любым типом
template<class T>
void foo(T t);
\end{lstlisting}
\begin{lstlisting}
// при инстанциировании первой перегрузки
// происходит ошибка (у int нет value\_type), 
// но это не приводит к ошибке компиляции
foo<int>(0);
\end{lstlisting}
\end{frame}

\begin{frame}[fragile]{Используем SFINAE}
\begin{lstlisting}
template<class T>
struct is_foo_defined
{
    // обёртка, которая позволит проверить
    // наличие метода foo с заданой сигнатурой
    template<class Z, void (Z::*)() = &Z::foo>
    struct wrapper {};

    template<class C>
    static YES check(wrapper<C> * p);

    template<class C>
    static NO  check(...);

    static bool const value = 
        sizeof(YES) == sizeof(check<T>(0));
};
\end{lstlisting}
\end{frame}

\begin{frame}[fragile]{Проверяем наличие метода}
\begin{lstlisting}
template<bool b> 
struct Bool2Type        { using type = YES; };

template<> 
struct Bool2Type<false> { using type = NO;  };
\end{lstlisting}

\begin{lstlisting}
template<class L> 
void foo_impl() 
{
    using Head = typename L::Head;
    
    constexpr bool has_foo = 
            is_foo_defined<Head>::value;
    
    using CALL = 
        typename Bool2Type<has_foo>::type;
    
    call_foo<Head>(CALL());
    foo_impl<typename L::Tail>();
}
\end{lstlisting}
\end{frame}

\begin{frame}[fragile]{Проверяем наличие метода (продолжение)}
\begin{lstlisting}
struct D : inherit<Bases> 
{
    // ... foo, foo\_impl

    template<class Base>
    void call_foo(YES) 
    { 
        static_cast<Base *>(this)->foo();
    }

    template<class Base>
    void call_foo(NO) { }
};

\end{lstlisting}
\end{frame}

\begin{frame}[fragile]{{\tt std::enable\_if}}
\begin{lstlisting}
// <type\_traits>
namespace std {
    template<bool B, class T = void> 
    struct enable_if {};
 
    template<class T>
    struct enable_if<true, T> { using type = T; };
}
\end{lstlisting}
\begin{lstlisting}
template<class T>
typename std::enable_if<
    std::is_integral<T>::value, T>::type
    div2(T t) { return t >> 1; }
 
template<class T>
typename std::enable_if<
    std::is_floating_point<T>::value, T>::type
    div2(T t) { return t / 2.0; }
\end{lstlisting}
\end{frame}

\begin{frame}[fragile]{{\tt std::enable\_if}}
\begin{lstlisting}
template<class T>
T div2(T t, typename std::enable_if<
        std::is_integral<T>::value, T>::type * = 0)
{ return t >> 1; }

template<class T, class E = typename std::enable_if<
    std::is_floating_point<T>::value, T>::type>
T div2(T t) 
{ return t / 2.0; }
\end{lstlisting}

\begin{lstlisting}
template<class T, class E = void> 
class A; 
 
template<class T>
class A<T, typename std::enable_if<
           std::is_integral<T>::value>::type> 
{};
\end{lstlisting}
\end{frame}


\end{document}
