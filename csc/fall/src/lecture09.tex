\documentclass{beamer}
\usepackage{csc}
\title{Лекция 9. Перегрузка операторов}

\date{
   \textbf{CS центр}\\
   17 октября 2017 \\
   Санкт-Петербург
}

\begin{document}
\begin{frame} 
  \titlepage
\end{frame}

\begin{frame}[fragile]{Основные операторы}
    \begin{block}{Арифметические}
        \begin{itemize}
            \item Унарные: префиксные {\color{blue}\verb!+ - ++ --!},
                постфиксные {\color{blue}\verb!++ --!}
            \item Бинарные: {\color{blue}\verb!+ - * / % += -= *= /= %=!}
        \end{itemize}
    \end{block}
    \begin{block}{Битовые}
        \begin{itemize}
            \item Унарные: {\color{blue}\verb!~!}.
            \item Бинарные: {\color{blue}\verb!& | ^ &= |= ^= >> << >>= <<=!}.
        \end{itemize}
    \end{block}
    \begin{block}{Логические}
        \begin{itemize}
            \item Унарные: {\color{blue}\verb#!#}. 
            \item Бинарные: {\color{blue}\verb!&& ||!}.
            \item Сравнения: {\color{blue}\verb#== != > < >= <=#}
        \end{itemize}
    \end{block}
\end{frame}

\begin{frame}[fragile]{Другие операторы}
        \begin{enumerate}
            \item Оператор присваивания: {\color{blue}\verb!=!}
            \item Специальные: 
                \begin{itemize}
                    \item префиксные {\color{blue}\verb!* &!}, 
                    \item постфиксные  {\color{blue}\verb!-> ->*!}, 
                    \item особые {\color{blue}\verb!,! \verb!. ::!}
                \end{itemize}
            \item Скобки: {\color{blue}\verb![] ()!}
            \item Оператор приведения {\color{blue} \verb!(type)!}
            \item Тернарный оператор: {\color{blue}\verb! x ? y : z !}
            \item Работа с памятью: {\color{blue}\verb!new new[] delete delete[]!}
        \end{enumerate}

    Нельзя перегружать операторы {\color{blue}\verb!.! \verb!::!} и тернарный
    оператор.
\end{frame}

\begin{frame}[fragile]{Перегрузка операторов}
    \begin{lstlisting}
Vector operator-(Vector const& v) {
    return Vector(-v.x, -v.y)
}

Vector operator+(Vector const& v, 
                 Vector const& w) {
    return Vector(v.x + w.x, v.y + w.y);
}

Vector operator*(Vector const& v, double d) {
    return Vector(v.x * d, v.y * d);
}

Vector operator*(double d, Vector const& v) {
    return v * d;
}
    \end{lstlisting}
\end{frame}

\begin{frame}[fragile]{Перегрузка операторов внутри классов}
    {}\small\textbf{NB:} Обязательно для {\color{blue}\verb!(type) [] () -> ->* =!}

    \begin{lstlisting}[basicstyle=\fontsize{9pt}{1em}\ttfamily]
struct Vector {
    Vector operator-() const { return Vector(-x, -y); }
    Vector operator-(Vector const& p) const {
        return Vector(x - p.x, y - p.y);
    }
    Vector & operator*=(double d) {
        x *= d;
        y *= d;
        return *this;
    }
    double operator[](size_t i) const { 
        return (i == 0) ? x : y;
    }
    bool    operator()(double d)   const   { ... }
    void    operator()(double a, double b) { ... }
    double x, y;
};
    \end{lstlisting}
\end{frame}

\begin{frame}[fragile]{Перегрузка инкремента и декремента}
    \begin{lstlisting}
struct BigNum {
    BigNum & operator++() { //prefix
        //increment
        ...
        return *this;
    }

    BigNum operator++(int) { //postfix
        BigNum tmp(*this);
        ++(*this);
        return tmp;
    }
    ...
};
    \end{lstlisting}
\end{frame}

\begin{frame}[fragile]{Переопределение операторов ввода-вывода}
    \begin{lstlisting}
#include <iostream>
        
struct Vector { ... };

std::istream& operator>>(std::istream & is, 
                         Vector & p) {
    is >> p.x >> p.y;
    return is;
}

std::ostream& operator<<(std::ostream &os, 
                         Vector const& p) {
    os << p.x << ' ' <<  p.y;
    return os;
}
    \end{lstlisting}
\end{frame}

\begin{frame}[fragile]{Умный указатель}
Реализует принцип: {\em ``Получение ресурса есть инициализация''
Resource Acquisition Is Initialization (RAII)}
\begin{lstlisting}
struct SmartPtr {
   Data & operator*()  const {return *data_;}
   Data * operator->() const {return  data_;}
   Data * get()        const {return  data_;}
   ... 
private:
   Data * data_;
};

bool operator==(SmartPtr const& p1,
                SmartPtr const& p2) {
    return p1.get() == p2.get();
}
    \end{lstlisting}
\end{frame}

\begin{frame}[fragile]{Оператор приведения}
    \begin{lstlisting}
struct String {
    operator bool() const {
        return size_ != 0;
    }

    operator char const *() const {
        if (*this)
            return data_;
        return "";
    }

private:
    char * data_;
    size_t size_;
};
    \end{lstlisting}
\end{frame}

\begin{frame}[fragile]{Операторы с особым порядком вычисления}
    \begin{lstlisting}
int main() {
    int a = 0;
    int b = 5;
    (a != 0) && (b = b / a);
    (a == 0) || (b = b / a);

    foo() && bar();
    foo() || bar();
    foo(), bar();
}

// no lazy semantics
Tribool operator&&(Tribool const& b1, 
                   Tribool const& b2) { 
    ... 
}
    \end{lstlisting}
\end{frame}

\begin{frame}[fragile]{Переопределение арифметических и битовых операторов}
    \begin{lstlisting}
struct String {
    String( char const * cstr ) { ... }
    String & operator+=(String const& s) {
        ...
        return *this;    
    }
    //\verb!String operator+(String const& s2) const {...}!
};
String operator+(String s1, String const& s2) { 
    return s1 += s2; 
}
    \end{lstlisting}
    \begin{lstlisting}
    String s1("world");
    String s2 = "Hello " + s1;
    \end{lstlisting}
\end{frame}

\begin{frame}[fragile]{``Правильное'' переопределение операторов сравнения}
    \begin{lstlisting}[basicstyle=\fontsize{9pt}{1em}\ttfamily]
bool operator==(String const& a, String const& b) 
{ return ...  }
bool operator!=(String const& a, String const& b) 
{ return !(a == b); }

bool operator<(String const& a, String const& b) 
{ return ...  }
bool operator>(String const& a, String const& b) 
{ return b < a; }
bool operator<=(String const& a, String const& b) 
{ return !(b < a); }
bool operator>=(String const& a, String const& b) 
{ return !(a < b); }
    \end{lstlisting}
\end{frame}

\begin{frame}[fragile]{О чём стоит помнить}
    \begin{itemize}
        \item Стандартная семантика операторов.
\begin{lstlisting}
void operator+(A const & a, A const& b) {}
\end{lstlisting}
        \item Приоритет операторов.
\begin{lstlisting}
    Vector a, b, c;
    c = a + a ^ b * a; //?????
\end{lstlisting}
    \item Хотя бы один из параметров должен быть пользовательским.
\begin{lstlisting}
    void operator*(double d, int i) {}
\end{lstlisting}
    \end{itemize}
\end{frame}

\end{document}

 
