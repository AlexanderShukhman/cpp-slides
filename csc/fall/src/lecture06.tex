\documentclass{beamer}
\usepackage{csc}
\title{Лекция 6. Инициализация и const}

\date{
   \textbf{CS центр}\\
   3 октября 2017 \\
   Санкт-Петербург
}

\begin{document}
\begin{frame} 
  \titlepage
\end{frame}

\begin{frame}[fragile]{Модификаторы доступа}{}
    Модификаторы доступа позволяют ограничивать 
    доступ к методам и полям класса.
\begin{lstlisting}
struct IntArray 
{
    explicit IntArray(size_t size) 
        : size_(size), data_(new int[size]) 
    {}
    ~IntArray() { delete [] data_;  }

    int &  get(size_t i) { return data_[i]; }
    size_t size()        { return size_; }

private:
    size_t size_;
    int *  data_;
};
\end{lstlisting}
\end{frame}

\begin{frame}[fragile]{Ключевое слово \code{class}}{}
    Ключевое слово \code{struct} можно заменить на 
    \code{class}, тогда поля и методы
    по умолчанию будут private.

\begin{lstlisting}
class IntArray 
{
public:
    explicit IntArray(size_t size) 
        : size_(size), data_(new int[size]) 
    {}
    ~IntArray() { delete [] data_;  }

    int &  get(size_t i) { return data_[i]; }
    size_t size()        { return size_; }
private:
    size_t size_;
    int *  data_;
};
\end{lstlisting}
\end{frame}

\begin{frame}[fragile]{Инварианты класса}{}
    \begin{itemize}
        \item Выделение {\em публичного интерфейса} позволяет
            поддерживать {\em инварианты класса}\\
            (сохранять
            данные объекта в согласованном состоянии).
\begin{lstlisting}
struct IntArray {
    ...
    size_t size_;
    int *  data_; // массив размера size\_
};
\end{lstlisting}
        
        \item Для сохранения инвариантов класса:
            \begin{enumerate}
                \item все поля должны быть закрытыми, 
                \item публичные методы должны сохранять инварианты класса.
            \end{enumerate}

        \item Закрытие полей класса позволяет абстрагироваться 
            от способа хранения данных объекта.
    \end{itemize}
\end{frame}

\begin{frame}[fragile]{Публичный интерфейс}{}
\begin{lstlisting}
struct IntArray 
{
    void resize(size_t nsize) 
    {
        int * ndata = new int[nsize];
        size_t n = nsize > size_ ? size_ : nsize;
        for (size_t i = 0; i != n; ++i)
            ndata[i] = data_[i];
        delete [] data_;
        data_ = ndata;
        size_ = nsize;
    }
private:
    size_t size_;
    int *  data_;
};
\end{lstlisting}
\end{frame}

\begin{frame}[fragile]{Абстракция}{}
\begin{lstlisting}
struct IntArray 
{
public:
    explicit IntArray(size_t size) 
        : size_(size), data_(new int[size]) 
    {}
    ~IntArray() { delete [] data_;  }

    int &  get(size_t i) { return data_[i]; }
    size_t size()        { return size_; }

private:
    size_t size_;
    int *  data_;
};
\end{lstlisting}
\end{frame}

\begin{frame}[fragile]{Абстракция}{}
\begin{lstlisting}
struct IntArray 
{
public:
    explicit IntArray(size_t size) 
        : data_(new int[size + 1]) 
    {
        data_[0] = size;
    }
    ~IntArray() { delete [] data_;  }

    int &  get(size_t i) { return data_[i + 1]; }
    size_t size()        { return data_[0]; }

private:
    int *  data_;
};
\end{lstlisting}
\end{frame}

\begin{frame}[fragile]{Определение констант}{}
    \begin{itemize}
        \item Ключевое слово \code{const} позволяет
            определять типизированные константы.
\begin{lstlisting}
double const pi = 3.1415926535;
int const day_seconds = 24 * 60 * 60;   
// массив констант
int const days[12] = {31, 28, 31,
                      30, 31, 30,
                      31, 31, 30,
                      31, 30, 31};
\end{lstlisting}
        \item Попытка изменить константные данные приводит к неопределённому
            поведению.
\begin{lstlisting}
    int * may = (int *) &days[4];
    *may = 30;
\end{lstlisting}
    \end{itemize}
\end{frame}

\begin{frame}[fragile]{Указатели и \code{const}}{}
В \langcpp можно определить как константный указатель,
так и указатель на константу:
\begin{lstlisting}
int a = 10;
const int * p1 = &a; // указатель на константу
int const * p2 = &a; // указатель на константу
*p1 = 20; // ошибка
p2  = 0;  // ОК

int * const p3 = &a; // константный указатель
*p3 = 30; // OK
p3  = 0;  // ошибка

// константный указатель на константу
int const * const p4 = &a;
*p4 = 30; // ошибка
p4  = 0;  // ошибка
\end{lstlisting}
\end{frame}

\begin{frame}[fragile]{Указатели и \code{const}}{}
    Можно использовать следующее правило:\\
    \begin{center}
    ``слово \code{const} делает неизменяемым тип слева от него''.
    \end{center}

\begin{lstlisting}
int a = 10;
int * p = &a;   

// указатель на указатель на const int
int const **  p1 = &p;

// указатель на константный указатель на int
int * const * p2 = &p;

// константный указатель на указатель на int
int ** const  p3 = &p;
\end{lstlisting}
\end{frame}

\begin{frame}[fragile]{Ссылки и \code{const}}{}
    \begin{itemize}
        \item Ссылка сама по себе является неизменяемой.
\begin{lstlisting}
int a = 10;
int & const b = a; // ошибка
int const & c = a; // ссылка на константу
\end{lstlisting}
        \item Использование константных ссылок позволяет избежать
            копирования объектов при передаче в функцию.
\begin{lstlisting}
Point midpoint(Segment const & s);
\end{lstlisting}
        \item По константной ссылке можно передавать rvalue.
\begin{lstlisting}
Point p = midpoint(Segment(Point(0,0),
                           Point(1,1)));
\end{lstlisting}
    \end{itemize}
\end{frame}

\begin{frame}[fragile]{Константные методы}{}
    \begin{itemize}
        \item Методы классов могут быть объявлены как \code{const}.
\begin{lstlisting}
struct IntArray {
    size_t size() const;
};
\end{lstlisting}
        \item Такие методы не могут менять поля объекта\\
            (тип \code{this}~--- указатель на \code{const}).

        \item У константных объектов (через указатель или ссылку на константу)
            можно вызывать только константные методы:

\begin{lstlisting}
IntArray const * p = foo();
p->resize(); // ошибка
\end{lstlisting}
        \item Внутри константных методов можно вызывать только константные методы.

    \end{itemize}
\end{frame}

\begin{frame}[fragile]{Две версии одного метода}{}
    \begin{itemize}
        \item Слово \code{const} является частью сигнатуры метода.
\begin{lstlisting}
size_t IntArray::size() const {return size_;}
\end{lstlisting}
        \item Можно определить две версии одного метода:
\begin{lstlisting}
struct IntArray 
{
    int get(size_t i) const { 
        return data_[i]; 
    }
    int & get(size_t i) { 
        return data_[i]; 
    }
private:
    size_t size_;
    int *  data_;
};
\end{lstlisting}

    \end{itemize}
\end{frame}

\begin{frame}[fragile]{Cинтаксическая и логическая константность}{}
\begin{itemize}
    \item Синтаксическая константность: константные методы
        не могут менять поля (обеспечивается компилятором).
    \item Логическая константность~--- нельзя менять те данные, 
        которые определяют состояние объекта.
\begin{lstlisting}
struct IntArray 
{
    void foo() const {
       // нарушение логической константности
        data_[10] = 1; 
    }
private:
    size_t size_;
    int *  data_;
};
\end{lstlisting}
\end{itemize}
    
\end{frame}


\begin{frame}[fragile]{Ключевое слово \code{mutable}}{}
Ключевое слово \code{mutable} позволяет определять поля,
которые можно изменять внутри константных методов:
\begin{lstlisting}
struct IntArray 
{
    size_t size() const { 
        ++counter_;
        return size_; 
    }

private:
    size_t size_;
    int *  data_;

    mutable size_t counter_;
};
\end{lstlisting}
\end{frame}


\begin{frame}[fragile]{Копирование объектов}{}
\begin{lstlisting}
struct IntArray 
{
    ...
private:
    size_t size_;
    int *  data_;
};

int main() {
    IntArray a1(10);
    IntArray a2(20);
    IntArray a3 = a1; // копирование
    a2 = a1; // присваивание

    return 0;
}
\end{lstlisting}
\end{frame}

\begin{frame}[fragile]{Конструктор копирования}{}
Если не определить конструктор копирования,
то он сгенерируется компилятором.
\begin{lstlisting}
struct IntArray 
{
    IntArray(IntArray const& a) 
        : size_(a.size_), data_(new int[size_]) 
    {
        for (size_t i = 0; i != size_; ++i)
            data_[i] = a.data_[i];    
    }
    ...
private:
    size_t size_;
    int *  data_;
};
\end{lstlisting}
\end{frame}

\begin{frame}[fragile]{Оператор присваивания}{}
Если не определить оператор присваивания,
то он тоже сгенерируется компилятором.
\begin{lstlisting}
struct IntArray 
{
    IntArray & operator=(IntArray const& a) {
        if (this != &a) {
            delete [] data_;
            size_ = a.size_;
            data_ = new int[size_];
            for (size_t i = 0; i != size_; ++i)
                data_[i] = a.data_[i];
        }
        return *this;
    }
    ...
};
\end{lstlisting}
\end{frame}

\begin{frame}[fragile]{Метод {\tt swap}}{}
\begin{lstlisting}
struct IntArray 
{
    void swap(IntArray & a) {
        size_t const t1 = size_;
        size_ = a.size_;
        a.size_ = t1;

        int * const t2 = data_;
        data_ = a.data_;
        a.data_ = t2;
    }
    ...
private:
    size_t size_;
    int *  data_;
};
\end{lstlisting}
\end{frame}

\begin{frame}[fragile]{Метод {\tt swap}}{}
Можно использовать функцию {\tt std::swap} и файла \code{algorithm}.
\begin{lstlisting}
#include <algorithm>

struct IntArray 
{
    void swap(IntArray & a) 
    {
        std::swap(size_, a.size_);
        std::swap(data_, a.data_);
    }
    ...
private:
    size_t size_;
    int *  data_;
};
\end{lstlisting}
\end{frame}

\begin{frame}[fragile]{Реализация оператора = при помощи {\tt swap}}{}
\begin{lstlisting}
struct IntArray 
{
    IntArray(IntArray const& a) 
        : size_(a.size_), data_(new int[size_]) {
        for (size_t i = 0; i != size_; ++i)
            data_[i] = a.data_[i];    
    }
    IntArray & operator=(IntArray const& a) {
        if (this != &a)
            IntArray(a).swap(*this);
        return *this;
    }
    ...
private:
    size_t size_;
    int *  data_;
};
\end{lstlisting}
\end{frame}

\begin{frame}[fragile]{Запрет копирования объектов}{}
    Для того, чтобы запретить копирование, нужно объявить конструктор копирования
    и оператор присваивания как \code{private} и не определять их.
\begin{lstlisting}
struct IntArray 
{
    ...
private:
    IntArray(IntArray const& a);
    IntArray & operator=(IntArray const& a);

    size_t size_;
    int *  data_;
};
\end{lstlisting}
\end{frame}

\begin{frame}[fragile]{Методы, генерируемые компилятором}{}
    Компилятор генерирует четыре метода:
    \begin{enumerate}
        \item конструктор по умолчанию,
        \item конструктор копирования,
        \item оператор присваивания,
        \item деструктор.
    \end{enumerate}

    Если потребовалось переопределить конструктор копирования,
    оператор присваивания или деструктор, то нужно 
    переопределить и остальные методы из этого списка. 
\end{frame}

\begin{frame}[fragile]{Поля и конструкторы}{}
\begin{lstlisting}
struct IntArray 
{
    explicit IntArray(size_t size) 
        : size_(size), data_(new int[size]) {
        for (size_t i = 0; i != size_; ++i)
            data_[i] = 0;
    }
    IntArray(IntArray const& a) 
        : size_(a.size_), data_(new int[size_]) {
        for (size_t i = 0; i != size_; ++i)
            data_[i] = a.data_[i];    
    }
    ...
private:
    size_t size_;
    int *  data_;
};
\end{lstlisting}
\end{frame}

\begin{frame}[fragile]{Деструктор, оператор присваивания и {\tt swap}}{}
\begin{lstlisting}
    ~IntArray() {
        delete [] data_;
    }

    IntArray & operator=(IntArray const& a) {
        if (this != &a)
            IntArray(a).swap(*this);
        return *this;
    }

    void swap(IntArray & a) {
        std::swap(size_, a.size_);
        std::swap(data_, a.data_);
    }
\end{lstlisting}
\end{frame}

\begin{frame}[fragile]{Методы}{}
\begin{lstlisting}
    size_t size() const { return size_; }

    int    get(size_t i)  const { 
        return data_[i]; 
    }
    int &  get(size_t i)        { 
        return data_[i]; 
    }

    void resize(size_t nsize) {
        IntArray t(nsize);
        size_t n = nsize > size_ ? size_ : nsize;
        for (size_t i = 0; i != n; ++i)
            t.data_[i] = data_[i];
        swap(t);
    }
\end{lstlisting}
\end{frame}

\end{document}
