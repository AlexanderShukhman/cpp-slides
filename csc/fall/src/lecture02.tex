\documentclass[aspectration=1610,t]{beamer}
\usepackage{csc}
\title{Лекция 2. Как выполняются программы на \langcpp}

\date{
   \textbf{CS центр}\\
   5 сентября 2017 \\
   Санкт-Петербург
}

\begin{document}
\begin{frame} 
  \titlepage
\end{frame}

\begin{frame}[fragile]{Типы данных}
\begin{itemize}
    \item Целочисленные:
        \begin{enumerate}
            \item \code{char} (символьный тип данных)
            \item \code{short int}
            \item \code{int}
            \item \code{long int}
        \end{enumerate}
    Могут быть беззнаковыми (\code{unsigned}).
    \begin{itemize}
        \item $-2^{n-1}\ldots (2^{n-1} - 1)$ ($n$~--- число бит)
        \item $0\ldots (2^{n} - 1)$ для $\code{unsigned}$
    \end{itemize}

    \item Числа с плавающей точкой:
        \begin{enumerate}
            \item \code{float}, 4 байта, 7 значащих цифр.
            \item \code{double}, 8 байт, 15 значащих цифр.
        \end{enumerate}
    \item Логический тип данных \code{bool}.
        
    \item Пустой тип \code{void}.
\end{itemize}
\end{frame}

\begin{frame}[fragile]{Литералы}
\begin{itemize}
    \item Целочисленные:
        \begin{enumerate}
            \item \code{'a'}~--- код буквы 'a', тип \code{char},
            \item \code{42} ~--- все целые числа по умолчанию типа \code{int},
            \item \code{1234567890L}~--- суффикс '\code{L}' соответствует типу \code{long},
            \item \code{1703U}~--- суффикс '\code{U}' соответствует типу \code{unsigned int},
            \item \code{2128506UL}~--- соответствует типу \code{unsigned long}.
        \end{enumerate}

    \item Числа с плавающей точкой:
        \begin{enumerate}
            \item \code{3.14}~--- 
                все числа с точкой по умолчанию типа \code{double},
            \item \code{2.71F}~--- суффикс '\code{F}' соответствует типу 
                \code{float},
            \item \code{3.0E8}~--- соответствует $3.0\cdot 10^{8}$.
        \end{enumerate}
    \item \code{true} и \code{false}~--- значения типа \code{bool}.

    \item Строки задаются в двойных кавычках: \verb!"Text string"!.

\end{itemize}
\end{frame}

\begin{frame}[fragile]{Переменные}
\begin{itemize}
    \item При определении переменной указывается её тип. При определении можно
        сразу задать начальное значение (инициализация).
    \begin{lstlisting}
int     i = 10;
short   j = 20;
bool    b = false;

unsigned long l = 123123;

double x = 13.5, y = 3.1415;
float z;
    \end{lstlisting}

    \item Нужно всегда инициализировать переменные.

    \item Нельзя определить переменную пустого типа \code{void}.
\end{itemize}
\end{frame}

\begin{frame}[fragile]{Операции}
\begin{minipage}{.59\textwidth}
    \begin{itemize}
        \item Оператор присваивания: \code{=}.
        \item Арифметические:
        \begin{enumerate}
            \item бинарные: \code{+ - * / \%},
            \item унарные:  \code{++ {-}{-}}.
        \end{enumerate}
        \item Логические:
        \begin{enumerate}
            \item бинарные:  \code{\&\& ||},
            \item унарные:   \code{!}. 
        \end{enumerate}
        \item Сравнения: \code{== != > < >= <=}.
        \item Приведения типов: \code{(type)}.
        \item Сокращённые версии бинарных операторов:
            \code{+= -= *= /= \%=}.
    \end{itemize}
\end{minipage}\hfill
\begin{minipage}{.4\textwidth}
\begin{lstlisting}
int i = 10;
i = (20 * 3) % 7;

int k = i++;
int l = --i;

bool b = !(k == l);

b = (a == 0) ||
    (1 / a < 1); 

double d = 3.1415;
float f = (int)d;

// d = d * (i + k)
d *= i + k;
\end{lstlisting}
\end{minipage}
\end{frame}

\begin{frame}[fragile]{Инструкции}
    \begin{itemize}
        \item Выполнение состоит из последовательности {\em инструкций}.

        \item Инструкции выполняются одна за другой.

        \item Порядок вычислений внутри инструкций не определён.
            \begin{lstlisting}
/* unspecified behavior */
int i = 10;
i = (i += 5) + (i * 4); 
            \end{lstlisting}
        \item Блоки имеют вложенную область видимости:
            \begin{lstlisting}
int k = 10;                
{   
    int k = 5 * i;  // не видна за пределами блока
    i = (k += 5) + 5; 
}
k = k + 1;
            \end{lstlisting}

    \end{itemize}
\end{frame}

\begin{frame}[fragile]{Условные операторы}
\begin{itemize}    
    \item Оператор \code{if}:
    \begin{lstlisting}
int d = b * b - 4 * a * c;
if ( d > 0 ) {
    roots = 2;
} else if ( d == 0 ){  
    roots = 1;
} else {
    roots = 0;
}
    \end{lstlisting}

    \item Тернарный условный оператор:
    \begin{lstlisting}
int roots = 0;
if (d >= 0)
    roots = (d > 0 ) ? 2 : 1;
    \end{lstlisting}
\end{itemize}
\end{frame}

\begin{frame}[fragile]{Циклы}
\begin{itemize}    
    \item Цикл \code{while}:
    \begin{lstlisting}
int squares = 0;
int k = 0; 
while ( k < 10 ) {
    squares += k * k;
    k = k + 1;
} 
    \end{lstlisting}

    \item Цикл \code{for}:
    \begin{lstlisting}
for ( int k = 0; k < 10; k = k + 1 ) {
    squares += k * k;
} 
    \end{lstlisting}
\item Для выхода из цикла используется оператор \code{break}.
\end{itemize}
\end{frame}

\begin{frame}[fragile]{Функции}
\begin{itemize}
    \item В сигнатуре функции указывается тип возвращаемого значений и типы параметров.

    \item Ключевое слово \code{return} возвращает значение.
    \begin{lstlisting}
double square(double x) {
    return x * x;
}
    \end{lstlisting}

    \item Переменные, определённые внутри функций,~— {\em локальные}.
    
    \item Функция может возвращать \code{void}.

    \item Параметры передаются по значению (копируются).

    \begin{lstlisting}
void strange(double x, double y) {
    x = y;
}
    \end{lstlisting}

\end{itemize}
\end{frame}

\begin{frame}[fragile]{Макросы}
\begin{itemize}
    \item Макросами в \langcpp называют инструкции препроцессора.

    \item Препроцессор \langcpp является самостоятельным языком,
        работающим с произвольными строками.

    \item Макросы можно использовать для определения функций:
    \begin{lstlisting}
int max1(int x, int y) { 
    return x > y ? x : y;
}

#define max2(x, y)   x > y ? x : y

a = b + max2(c, d);      //  b + c > d ? c : d;
    \end{lstlisting}

    \item Препроцессор ``не знает'' про синтаксис \langcpp.
    \end{itemize}
\end{frame}

\begin{frame}[fragile]{Макросы}
\begin{itemize}
    \item Параметры макросов нужно оборачивать в скобки:
    \begin{lstlisting}
#define max3(x, y)   ((x) > (y) ? (x) : (y))
    \end{lstlisting}

    \item Это не избавляет от всех проблем:
    \begin{lstlisting}
int a = 1;
int b = 1;
int c = max3(++a, b); 
// c = ((++a) > (b) ? (++a) : (b))
    \end{lstlisting}

    \item Определять функции через макросы~--- плохая идея.

    \item Макросы можно использовать для условной компиляции:
        \begin{lstlisting}
#ifdef DEBUG
    // дополнительные проверки
#endif
        \end{lstlisting}

\end{itemize}
\end{frame}

\begin{frame}[fragile]{Ввод-вывод}
\begin{itemize}    
    \item Будем использовать библиотеку \code{<iostream>}.
    \begin{lstlisting}
#include <iostream>
using namespace std;
    \end{lstlisting}

    \item Ввод:
    \begin{lstlisting}
    int a = 0;
    int b = 0;
    cin >> a >> b;
    \end{lstlisting}

    \item Вывод:
    \begin{lstlisting}
    cout << "a + b = " << (a + b) << endl;
    \end{lstlisting}
\end{itemize}
\end{frame}

\begin{frame}[fragile]{Простая программа}
    \begin{lstlisting}
#include <iostream>
using namespace std;

int main () 
{
    int a = 0;
    int b = 0;

    cout << "Enter a and b: ";
    cin >> a >> b;

    cout << "a + b = " << (a + b) << endl;

    return 0;
}
    \end{lstlisting}
\end{frame}

\begin{frame}[fragile]{Архитектура фон Неймана}
Современных компьютеры построены по принципам 
архитектуры фон Неймана:
\begin{enumerate}
    \item {\bf Принцип однородности памяти.}\\
        Команды и данные хранятся в одной и той же памяти и внешне в памяти неразличимы.
    \item {\bf Принцип адресности.}\\
        Память состоит из пронумерованных ячеек.
    \item {\bf Принцип программного управления.}\\
        Все вычисления представляются в виде последовательности команд.
    \item {\bf Принцип двоичного кодирования.}\\
        Вся информация (данные и команды) кодируются двоичными числами.
\end{enumerate}
\end{frame}

\begin{frame}[fragile]{Сегментация памяти}
    \begin{itemize}
        \item Оперативная память, используемая в программе на C++,
            разделена на области двух типов:
            \begin{enumerate}
                \item сегменты данных,
                \item сегменты кода (текстовые сегменты).
            \end{enumerate}

        \item В сегментах кода содержится код программы.

        \item В сегментах данных располагаются данные программы 
            (значения переменных, массивы и пр.).

        \item При запуске программы выделяются два сегмента данных:
            \begin{enumerate}
                \item сегмент глобальных данных,
                \item стек.
            \end{enumerate}

        \item В процессе работы программы могут выделяться и освобождаться
            дополнительные сегменты памяти.

        \item Обращения к адресу вне выделенных сегментов~--- ошибка времени
            выполнения (access violation, segmentation fault).
    \end{itemize}
\end{frame}

\begin{frame}[fragile]{Как выполняется программа?}
    \begin{itemize}
        \item Каждой функции в скомпилированном коде соответствует отдельная
            секция.
            
        \item Адрес начала такой секции — это адрес функции.

        \item Телу функции соответствует последовательность команд процессора.
            
        \item Работа с данными происходит на уровне байт,
            информация о типах отсутствует.

        \item В процессе выполнения адрес следующей инструкции хранится 
            в специальном регистре процессора IP (Instruction Pointer).

        \item Команды выполняются последовательно, пока не встретится
            специальная команда (например, условный переход или вызов 
            функции), которая изменит IP.
    \end{itemize}
\end{frame}

\begin{frame}[fragile]{Ещё раз о линковке}
    \begin{itemize}
        \item На этапе компиляции объектных файлов
            в места вызова функций подставляются имена функций.

        \item На этапе линковки в места вызова вместо имён функций
            подставляются их адреса.

        \item Ошибки линковки:
            \begin{enumerate}
                \item {\tt undefined reference}\\
                    Функция имеет объявление, но не имеет тела.

                \item {\tt multiple definition}\\
                    Функция имеет два или более определений.
            \end{enumerate}

        \item Наиболее распространённый способ получить 
            {\tt multiple definition}~--- определить функцию
            в заголовочном файле, который включён в несколько
            {\tt .cpp} файлов.
    \end{itemize}
\end{frame}

\begin{frame}[fragile]{Стек вызовов}
    \begin{itemize}
        \item Стек вызовов~— это сегмент данных, используемый для хранения локальных
            переменных и временных значений.

        \item Не стоит путать стек с одноимённой структурой данных, у~стека в \langcpp
            можно обратиться к произвольной ячейке.

        \item Стек выделяется при запуске программы.

        \item Стек обычно небольшой по размеру (4Мб).

        \item Функции хранят свои локальные переменные на стеке. 
        
        \item При выходе из функции соответствующая область стека
            объявляется свободной.

        \item Промежуточные значения, возникающие при вычислении
            сложных выражений, также хранятся на стеке.
    \end{itemize}
\end{frame}

\begin{frame}[fragile]{Устройство стека}
\begin{minipage}{.5\textwidth}
\begin{tikzpicture}[
      start chain=1 going below,node distance=-0.15mm
    ]
    \tikzstyle{cell} = [rectangle,minimum width=2cm,draw,thick,on chain=1];
    \tikzstyle{function} = [cell, minimum height=1cm]; 
    \tikzstyle{arrow} = [>=stealth',shorten >=2pt, shorten <=2pt]

    \node [function, fill=green!30]  (main) {main()};
    \node<1,5> [cell, minimum height=45mm, dashed] (empty) {};

    \node<2,4> [function, fill=blue!30]   (foo)  {foo()};
    \node<2,4> [cell, minimum height=35mm, dashed] (empty) {};

    \node<3> [function, fill=blue!30] (foo)  {foo()};
    \node<3> [function, fill=orange!30] (bar)  {bar()};
    \node<3> [cell, minimum height=25mm, dashed] (empty) {};

    \node<6> [function, fill=orange!30] (bar)  {bar()};
    \node<6> [cell, minimum height=35mm, dashed] (empty) {};

    \node [right of=main, xshift=3cm, yshift=8mm] (bottom) {дно стека};
    \path [arrow,->] (bottom) edge node {} (main.north east);

    \node [right of=empty, xshift=3cm, yshift=-8mm] (top) {вершина стека};
    \path [arrow,->] (top) edge node {} (empty.north east);
\end{tikzpicture}
\end{minipage}\hfill
\begin{minipage}{.4\textwidth}
\begin{lstlisting}
void bar( ) {
    int c;
}

void foo( ) {
    int b = 3;
    bar();
}

int main( ) {
    int a = 3;    
    foo();
    bar();

    return 0;
}
\end{lstlisting}
\end{minipage}
\end{frame}

\begin{frame}[fragile]{Вызов функции}
\begin{minipage}{.3\textwidth}
\small
\begin{tikzpicture}[
      start chain=1 going below,node distance=-0.15mm
    ]
    \tikzstyle{cell} = [rectangle,minimum width=2cm, minimum height=5mm,,draw,thick,on chain=1];
    \tikzstyle{function} = [cell, minimum height=1cm]; 
    \tikzstyle{arrow} = [>=stealth',shorten >=2pt, shorten <=2pt]
    \tikzstyle{mbrace} = [decorate,decoration={brace,amplitude=5pt}];

    \node<1-8> [cell, fill=green!30] (x)   {\tt x = 1};
    \node<9>   [cell, fill=red!30]   (x)   {\tt x = 2};
    \node [cell, fill=green!30] (y)   {\tt y = 2};


    \node<2-8> [cell, fill=green!30] (p3)   {\tt false};
    \node<2-8> [cell, fill=green!30] (p2)   {\tt 2};
    \node<2-8> [cell, fill=green!30] (p1)   {\tt 1};

%    \draw[mbrace] 
%        (p3.north east)--(p1.south east) node[midway,anchor=west,xshift=6pt]
%        {аргументы};

    \node<3-6> [cell, fill=green!30] (retv)  {ret val};
    \node<7-8> [cell, fill=red!30]   (retv)  {2};
    \node<3-8> [cell, fill=green!30] (reta)  {ret addr};
    \node<3-8> [cell, fill=green!30] (retr)  {registers};


    %\draw<1-3>[mbrace] 
    %    (x.north east)--(empty.north east) node[midway,anchor=west,xshift=6pt] {main};

%    \node[right of=retv,xshift=21mm] {результат};
%    \node[right of=reta,xshift=15mm] {IP};
%    \node[right of=retr,xshift=21mm] {состояние};


    \node<4-5> [cell, fill=blue!30] (d)   {\tt d};
    \node<4-5> [cell, fill=blue!30] (h)   {\tt h};
    \node<5>   [cell, fill=red!30] (tmp)  {\tt a * b = 2};
    \node<6-7> [cell, fill=blue!30] (d)   {\tt d = 5.42};
    \node<6-7> [cell, fill=blue!30] (h)   {\tt h = 2};


    \node<1,9> [cell, minimum height=55mm, dashed] (empty) {};
    \node<2> [cell, minimum height=40mm, dashed] (empty) {};
    \node<3,8> [cell, minimum height=25mm, dashed] (empty) {};
    \node<4> [cell, minimum height=15mm, dashed] (empty) {};
    \node<5> [cell, minimum height=10mm, dashed] (empty) {};
    \node<6-7> [cell, minimum height=15mm, dashed] (empty) {};

    \node [right of=y,xshift=22mm] (fp) {\parbox{15mm}{frame\\ pointer}};
    
    \path<1-3,8-9> [arrow,->] (fp) edge node {} (x.north east);

    \path<4-7> [arrow,->] (fp) edge node {} (d.north east);

    %\draw<4> [mbrace] 
    %    (d.north east)--(h.south east) node[midway,anchor=west,xshift=8pt] {foo};
                                    
    \node [right of=empty,xshift=22mm] (sp) {\parbox{15mm}{stack\\ pointer}};
    \path [arrow,->] (sp) edge node {} (empty.north east);

\end{tikzpicture}
\end{minipage}\hfill
\begin{minipage}{.6\textwidth}
\begin{lstlisting}
int foo(int a, int b, bool c) 
{
    double d = a * b * 2.71;
    int    h = c ? d : d / 2;
    return h;
}

int main( ) 
{
    int x = 1;
    int y = 2;
    x = foo (x, y, false);
    cout << x;
    return 0;
}
\end{lstlisting}
\end{minipage}
\end{frame}

\begin{frame}[fragile]{Вызов функции}
    \begin{itemize}
        \item При вызове функции на стек складываются:
            \begin{enumerate}
                \item aргументы функции,
                \item адрес возврата,
                \item значение frame pointer и регистров процессора.
            \end{enumerate}

        \item Кроме этого на стеке резервируется место под
            возвращаемое значение.

        \item Параметры передаются в обратном порядке, что позволяет
            реализовать функции с переменным числом аргументов.

        \item Адресация локальных переменных функции и аргументов
            функции происходит относительно frame pointer.

        \item Конкретный процесс вызова зависит от используемых 
            соглашений (cdecl,  stdcall, fastcall, thiscall).
    \end{itemize}
\end{frame}

\end{document}
