\documentclass{beamer}
\usepackage{csc}
\title{Лекция 4. Динамическая память}

\date{
   \textbf{CS центр}\\
   12 сентября 2017 \\
   Санкт-Петербург
}

\begin{document}
\begin{frame} 
  \titlepage
\end{frame}

\begin{frame}[fragile]{Зачем нужна динамическая память?}
    \begin{itemize}
        \item Стек программы ограничен. Он не предназначен
            для хранения больших объемов данных.
\begin{lstlisting}
// Не умещается на стек
double m[10000000] = {}; // 80 Mb
\end{lstlisting}

        \item Время жизни локальных переменных ограничено
            временем работы функции.

        \item Динамическая память выделяется в сегменте данных.

        \item Структура, отвечающая за выделение дополнительной
            памяти, называется {\bf кучей} (не нужно путать
            с одноимённой структурой данных).

        \item Выделение и освобождение памяти {\em управляется вручную}.
    \end{itemize}
\end{frame}

\begin{frame}[fragile]{Выделение памяти в стиле \langc}
    \begin{itemize}
        \item Стандартная библиотека \code{cstdlib}
            предоставляет четыре функции для управления памятью:
            {\small
\begin{lstlisting}
void * malloc (size_t size);
void * calloc (size_t nmemb, size_t size);
void * realloc(void * ptr, size_t size);
void   free   (void * ptr);
\end{lstlisting}
}
        \item \code{size\_t}~--- специальный целочисленный беззнаковый тип, 
            может вместить в себя размер любого типа в байтах.

        \item Тип \code{size\_t} используется для указания размеров типов данных,
            для индексации массивов и пр.

        \item \code{void *}~--- это указатель на нетипизированную память\\
            (раньше для этого использовалось \code{char *}).

    \end{itemize}
\end{frame}

\begin{frame}[fragile]{Выделение памяти в стиле \langc}
    \begin{itemize}
        \item Функции для управления памятью в стиле \langc:
            {\small
\begin{lstlisting}
void * malloc (size_t size);
void * calloc (size_t nmemb, size_t size);
void * realloc(void * ptr, size_t size);
void   free   (void * ptr);               
\end{lstlisting}
}
        \item \code{malloc}~--- выделяет область памяти размера $\ge$
            \code{size}. Данные не инициализируются.

        \item \code{calloc}~--- выделяет массив из \code{nmemb} элементов  размера 
            \code{size}. Данные инициализируются нулём.

        \item \code{realloc}~--- изменяет размер области памяти по указателю
            \code{ptr} на \code{size} (если возможно, то это делается на месте).

        \item \code{free}~--- освобождает область памяти, ранее выделенную
            одной из функций \code{malloc}/\code{calloc}/\code{realloc}.
            
    \end{itemize}
\end{frame}

\begin{frame}[fragile]{Выделение памяти в стиле \langc}
    \begin{itemize}
        \item Для указания размера типа используется оператор \code{sizeof}.
\begin{lstlisting}
// создание массива из 1000 int    
int * m = (int *)malloc(1000 * sizeof(int));
m[10] = 10;

// изменение размера массива до 2000
m = (int *)realloc(m, 2000 * sizeof(int));

// освобождение массива
free(m);

// создание массива нулей
m = (int *)calloc(3000, sizeof(int));
free(m);
m = 0;
\end{lstlisting}
    \end{itemize}
\end{frame}


\begin{frame}[fragile]{Выделение памяти в стиле \langcpp}
    \begin{itemize}
        \item Язык \langcpp предоставляет два набора операторов для
            выделения памяти:
            \begin{enumerate}
                \item \code{new} и \code{delete}~--- для одиночных значений,

                \item \code{new []} и \code{delete []}~--- для массивов.
            \end{enumerate}

        \item Версия оператора \code{delete} должна соответствовать
            версии оператора \code{new}.
\begin{lstlisting}
// выделение памяти под один int со значением 5
int * m = new int(5);
delete m; // освобождение памяти

// создание массива нулей
m = new int[1000](); // () означает обнуление 
delete [] m; // освобождение памяти
\end{lstlisting}
    \end{itemize}
\end{frame}

\begin{frame}[fragile]{Типичные проблемы при работе с памятью}
    \begin{itemize}
        \item Проблемы производительности: создание переменной на стеке
            намного ``дешевле'' выделения для неё динамической памяти.

        \item Проблема фрагментации: выделение большого количества
            небольших сегментов способствует фрагментации памяти.
        
        \item Утечки памяти:
\begin{lstlisting}
// создание массива из 1000 int    
int * m = new int[1000];

// создание массива из 2000 int    
m = new int[2000]; // утечка памяти

// Не вызван delete [] m, утечка памяти
\end{lstlisting}
    \end{itemize}
\end{frame}

\begin{frame}[fragile]{Типичные проблемы при работе с памятью}
    \begin{itemize}
    \item Неправильное освобождение памяти.
\begin{lstlisting}
int * m1 = new int[1000];
delete m1; // должно быть delete [] m1

int * p = new int(0);
free(p); // совмещение функций C++ и C

int * q1 = (int *)malloc(sizeof(int));
free(q1);
free(q1); // двойное удаление

int * q2 = (int *)malloc(sizeof(int));
free(q2);
q2 = 0;   // обнуляем указатель
free(q2); // правильно работает для q2 = 0
\end{lstlisting}
\end{itemize}
\end{frame}

\begin{frame}[fragile]{Многомерные встроенные массивы}
    \begin{itemize}
        \item \langcpp позволяет определять многомерные массивы:
        \begin{lstlisting}
int m2d[2][3] = { {1, 2, 3}, {4, 5, 6} };
for( size_t i = 0; i != 2; ++i ) {
    for( size_t j = 0; j != 3; ++j ) {
        cout << m2d[i][j] << ' ';
    }
    cout << endl;
}
        \end{lstlisting}
    \item Элементы {\tt m2d} располагаются в памяти ``по строчкам''.

    \item Размерность массивов может быть любой, но на практике
        редко используют массивы размерности $> 4$.

\begin{lstlisting}
int m4d[2][3][4][5] = {};
\end{lstlisting}
    \end{itemize}
\end{frame}

\begin{frame}[fragile]{Динамические массивы}
    \begin{itemize}
        \item Для выделения одномерных динамических массивов
            обычно используется оператор \code{new []}.
\begin{lstlisting}
int * m1d = new int[100];
\end{lstlisting}
        \item Какой тип должен быть у указателя на двумерный динамический
            массив?
            \begin{itemize}
                \item Пусть {\tt m}~--- указатель на двумерный массив типа              \code{int}.

                \item Значит {\tt m[i][j]} имеет тип \code{int} (точнее
                    \code{int \&}).

                \item {\tt m[i][j] $\Leftrightarrow$ *(m[i] + j)},
                    т.е. тип {\tt m[i]}~--- \code{int *}.

                \item аналогично, {\tt m[i] $\Leftrightarrow$ *(m + i)},
                    т.е. тип {\tt m}~--- \code{int **}.
            \end{itemize}
            \item Чему соответствует значение {\tt m[i]}?\\ 
                Это адрес строки с номером {\tt i}.

            \item Чему соответствует значение {\tt m}?\\
                Это адрес массива с указателями на строки.
    \end{itemize}
\end{frame}

\begin{frame}[fragile]{Двумерные массивы}
    Давайте рассмотрим создание массива $5\times 4$.
    
\begin{center}
\begin{tikzpicture}[                                                          
      start chain=1 going right, node distance=-0.15mm,
      start chain=2 going right
    ]
    \tikzstyle{cell} = [rectangle,minimum width=4mm, minimum height=4mm,draw];
    \tikzstyle{arrow} = [>=stealth']

    \begin{scope}[shift={(4cm,35mm)}]
    \foreach \i in {1,2,3,4} {
        \node[cell,on chain=2] (m\i-1) {};   
            \foreach \j/\k in {{2/1},{3/2},{4/3},{5/4}}
                \node[cell,below of=m\i-\k,yshift=-7mm] (m\i-\j) {};
    };
   
    \node[on chain=2] (t0) {{\tt m[0]}};   
    \foreach \i/\k in {{1/0},{2/1},{3/2},{4/3}}
        \node[below of=t\k,yshift=-7mm] (t\i) {{\tt m[\i]}};
    \end{scope}

    \node[on chain=1] (m) {{\tt m}};
    \foreach \i in {1,2,3,4,5} {
        \node[cell,on chain=1] (m\i) {};
        \path [arrow,->] (m\i.center) edge node {} (m1-\i.west);     
    };
\end{tikzpicture}
        \begin{lstlisting}
int ** m = new int * [5];
for (size_t i = 0; i != 5; ++i)
    m[i] = new int[4];
        \end{lstlisting}
\end{center}
\end{frame}

\begin{frame}[fragile]{Двумерные массивы}
    Выделение и освобождение двумерного массива размера $a\times b$.
        \begin{lstlisting}
int ** create_array2d(size_t a, size_t b) {
    int ** m = new int *[a];
    for (size_t i = 0; i != a; ++i)
        m[i] = new int[b];
    return m;
}

void free_array2d(int ** m, size_t a, size_t b) {
    for (size_t i = 0; i != a; ++i)
        delete [] m[i];
    delete [] m;
}
        \end{lstlisting}
При создании массива оператор \code{new} вызывается $(a + 1)$ раз.
\end{frame}

\begin{frame}[fragile]{Двумерные массивы: эффективная схема}
    Рассмотрим эффективное создание массива $5\times 4$.
    
\begin{center}
\begin{tikzpicture}[                                                          
      start chain=1 going right, node distance=-0.15mm,
      start chain=2 going right
    ]
    \tikzstyle{cell} = [rectangle,minimum width=4mm, minimum height=4mm,draw];
    \tikzstyle{arrow} = [>=stealth']
    \tikzstyle{mbrace} = [decorate,decoration={brace,amplitude=5pt}];

    \begin{scope}[shift={(0cm,25mm)}]
    \foreach \i in {0,1,2,3,4} {
        \node[cell,on chain=2] (m\i-1) {};
        \node[above of=m\i-1, shift={(6mm,7mm)}] (t\i) {{\tt\small m[\i]}};
         \foreach \j in {2,3,4}
             \node[cell,on chain=2] (m\i-\j) {};
        \draw[mbrace] 
            (m\i-1.north west)--(m\i-4.north east) node[midway,anchor=west,xshift=6pt] {};
    };
    \end{scope}

    \node[on chain=1] (m) {{\tt m}};
    \foreach \i in {0,1,2,3,4} {
        \node[cell,on chain=1] (m\i) {};
        \path [arrow,->] (m\i.center) edge node {} (m\i-1.south west);     
    };
\end{tikzpicture}
        \begin{lstlisting}
int ** m = new int * [5];
m[0] = new int[5 * 4];
for (size_t i = 1; i != 5; ++i)
    m[i] = m[i - 1] + 4;
        \end{lstlisting}
\end{center}
\end{frame}

\begin{frame}[fragile]{Двумерные массивы: эффективная схема}
    Эффективное выделение и освобождение двумерного массива размера $a\times b$.

    \begin{lstlisting}
int ** create_array2d(size_t a, size_t b) {
    int ** m = new int *[a];
    m[0] = new int[a * b];
    for (size_t i = 1; i != a; ++i)
        m[i] = m[i - 1] + b;
    return m;
}

void free_array2d(int ** m, size_t a, size_t b) {
    delete [] m[0];
    delete [] m;
}
        \end{lstlisting}
При создании массива оператор \code{new} вызывается 2 раза.
\end{frame}


\begin{frame}[fragile]{Строковые литералы}
    \begin{itemize}
        \item Строки~--- это массивы символов
            типа \code{char}, заканчивающиеся нулевым символом.
\begin{lstlisting}
// массив 'H', 'e', 'l', 'l', 'o', '\verb!\!0'
char s[] = "Hello";

\end{lstlisting}

        \item Строки могут содержать управляющие последовательности:
            \begin{enumerate}
                \item \verb!\n!~--- перевод строки,
                \item \verb!\t!~--- символ табуляции,
                \item \verb!\\!~--- символ '\verb!\!',
                \item \verb!\"!~--- символ '\verb!"!',
                \item \verb!\0!~--- нулевой символ.
            \end{enumerate}

\begin{lstlisting}
cout << "List:\n\t- C,\n\t- C++.\n";
\end{lstlisting}
    \end{itemize}
\end{frame}

\begin{frame}[fragile]{Работа со строками в стиле \langc}
    \begin{itemize}
        \item Библиотека \code{cstring} предлагает множество
            функций для работы со строками (\code{char *}).
\begin{lstlisting}
char s1[100] = "Hello";
cout << strlen(s1) << endl; // 5

char s2[] = ", world!";
strcat(s1, s2);
                               
char s3[6] = {72, 101, 108, 108, 111};
if (strcmp(s1, s3) == 0)
    cout << "s1 == s3" << endl;
\end{lstlisting}
        \item Работа со строками в стиле \langc предполагает
            кропотливую работу с ручным выделением памяти.
    \end{itemize}
\end{frame}

\begin{frame}[fragile]{Работа со строками в стиле \langcpp}
        Библиотека \code{string} предлагает обёртку
            над строками, которая позволяет упростить все
            операции со строками.

\small
\begin{lstlisting}
#include <string>
using namespace std;

int main() {
    string s1 = "Hello";
    cout << s1.size() << endl; // 5

    string s2 = ", world!";
    s1 = s1 + s2;
    if (s1 == s2)
        cout << "s1 == s2" << endl;
    return 0;
}
\end{lstlisting}
\end{frame}

\begin{frame}[fragile]{Ввод-вывод в стиле \langc}
    \begin{itemize}
        \item Библиотека \code{cstdio} предлагает функции
            для работы со стандартным вводом-выводом.

        \item Для вывода используется функция {\tt printf}:
\begin{lstlisting}
#include <cstdio>

int main() {
    int h = 20, m = 14;
    printf("Time: %d:%d\n", h, m);
    printf("It's %.2f hours to midnight\n", 
            ((24 - h) * 60.0 - m) / 60);
    return 0;
}
\end{lstlisting}
    \end{itemize}
\end{frame}

\begin{frame}[fragile]{Ввод-вывод в стиле \langc}
    \begin{itemize}
        \item Библиотека \code{cstdio} предлагает функции
            для работы со стандартным вводом-выводом.

        \item Для ввода используется функция {\tt scanf}:
\begin{lstlisting}
#include <cstdio>

int main() {
    int a = 0, b = 0;
    printf("Enter a and b: ");
    scanf("%d %d", &a, &b);
    printf("a + b = %d\n", (a + b));
    return 0;
}
\end{lstlisting}
    \item Ввод-вывод в стиле \langc{} достаточно сложен и
        небезопасен (типы аргументов не проверяются).
    \end{itemize}
\end{frame}

\begin{frame}[fragile]{Ввод-вывод в стиле \langcpp}
\begin{itemize}    
    \item В \langcpp ввод-вывод реализуется через библиотеку \code{iostream}.
    \begin{lstlisting}
#include <string>
#include <iostream>
using namespace std;

int main() {
    string name;
    cout << "Enter your name: ";
    cin >> name; // считывается слово
    cout << "Hi, " << name << endl;

    return 0;
}
    \end{lstlisting}
    \item Реализация ввода-вывода в стиле \langcpp типобезопасна.
\end{itemize}
\end{frame}

\begin{frame}[fragile]{Работа с файлами в стиле \langcpp}
\begin{itemize}    
    \small
    \item Библиотека \code{fstream} обеспечивает работу с файлами.
    \begin{lstlisting}
#include <string>
#include <fstream>
using namespace std;

int main() {
    string name;
    ifstream input("input.txt");
    input >> name;

    ofstream output("output.txt");
    output << "Hi, " << name << endl;
    return 0;
}
    \end{lstlisting}
    \item Файлы закроются при выходе из функции.
\end{itemize}
\end{frame}

\end{document}
