\documentclass{beamer}

\usepackage{mooc}

\usetikzlibrary{arrows, decorations.markings, shapes}
\usetikzlibrary{positioning}

% The face style, can be changed

\title{{\bf Программирование на языке \langcpp\protect\\Лекция
3\protect\vspace{1em}\\}Модификаторы доступа}

\begin{document}
\begin{frame} 
  \titlepage
\end{frame}

\begin{frame}[fragile]{Модификаторы доступа}{}
    Модификаторы доступа позволяют ограничивать 
    доступ к методам и полям класса.
\begin{lstlisting}
struct IntArray {
    explicit IntArray(size_t size) 
        : size_(size), data_(new int[size]) 
    {}
    ~IntArray() { delete [] data_;  }

    int &  get(size_t i) { return data_[i]; }
    size_t size()        { return size_; }

private:
    size_t size_;
    int *  data_;
};
\end{lstlisting}
\end{frame}

\begin{frame}[fragile]{Ключевое слово \code{class}}{}
    Ключевое слово \code{struct} можно заменить на 
    \code{class}, тогда поля и методы
    по умолчанию будут private.

\begin{lstlisting}
class IntArray {
public:
    explicit IntArray(size_t size) 
        : size_(size), data_(new int[size]) 
    {}
    ~IntArray() { delete [] data_;  }

    int &  get(size_t i) { return data_[i]; }
    size_t size()        { return size_; }

private:
    size_t size_;
    int *  data_;
};
\end{lstlisting}
\end{frame}

\begin{frame}[fragile]{Инварианты класса}{}
    \begin{itemize}
        \item Выделение {\em публичного интерфейса} позволяет
            поддерживать {\em инварианты класса}\\
            (сохранять
            данные объекта в согласованном состоянии).
\begin{lstlisting}
struct IntArray {
    ...
    size_t size_;
    int *  data_; // массив размера size\_
};
\end{lstlisting}
        
        \item Для сохранения инвариантов класса:
            \begin{enumerate}
                \item все поля должны быть закрытыми, 
                \item публичные методы должны сохранять инварианты класса.
            \end{enumerate}

        \item Закрытие полей класса позволяет абстрагироваться 
            от способа хранения данных объекта.
    \end{itemize}
\end{frame}

\begin{frame}[fragile]{Публичный интерфейс}{}
\begin{lstlisting}
struct IntArray {
    ...
    void resize(size_t nsize) {
        int * ndata = new int[nsize];
        size_t n = nsize > size_ ? size_ : nsize;
        for (size_t i = 0; i != n; ++i)
            ndata[i] = data_[i];
        delete [] data_;
        data_ = ndata;
        size_ = nsize;
    }
private:
    size_t size_;
    int *  data_;
};
\end{lstlisting}
\end{frame}

\begin{frame}[fragile]{Абстракция}{}
\begin{lstlisting}
struct IntArray {
public:
    explicit IntArray(size_t size) 
        : size_(size), data_(new int[size]) 
    {}
    ~IntArray() { delete [] data_;  }

    int &  get(size_t i) { return data_[i]; }
    size_t size()        { return size_; }

private:
    size_t size_;
    int *  data_;
};
\end{lstlisting}
\end{frame}

\begin{frame}[fragile]{Абстракция}{}
\begin{lstlisting}
struct IntArray {
public:
    explicit IntArray(size_t size) 
        : data_(new int[size + 1]) 
    {
        data_[0] = size;
    }
    ~IntArray() { delete [] data_;  }

    int &  get(size_t i) { return data_[i + 1]; }
    size_t size()        { return data_[0]; }

private:
    int *  data_;
};
\end{lstlisting}
\end{frame}

\end{document}
