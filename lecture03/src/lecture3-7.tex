\documentclass{beamer}

\usepackage{mooc}

\usetikzlibrary{arrows, decorations.markings, shapes}
\usetikzlibrary{positioning}

% The face style, can be changed

\title{{\bf Программирование на языке \langcpp\protect\\Лекция
3\protect\vspace{1em}\\}Конструктор копирования и оператор присваивания}

\begin{document}
\begin{frame} 
  \titlepage
\end{frame}

\begin{frame}[fragile]{Копирование объектов}{}
\begin{lstlisting}
struct IntArray {
    ...
private:
    size_t size_;
    int *  data_;
};

int main() {
    IntArray a1(10);
    IntArray a2(20);
    IntArray a3 = a1; // копирование
    a2 = a1; // присваивание

    return 0;
}
\end{lstlisting}
\end{frame}

\begin{frame}[fragile]{Конструктор копирования}{}
Если не определить конструктор копирования,
то он сгенерируется компилятором.
\begin{lstlisting}
struct IntArray {
    IntArray(IntArray const& a) 
        : size_(a.size_), data_(new int[size_]) 
    {
        for (size_t i = 0; i != size_; ++i)
            data_[i] = a.data_[i];    
    }
    ...
private:
    size_t size_;
    int *  data_;
};
\end{lstlisting}
\end{frame}

\begin{frame}[fragile]{Оператор присваивания}{}
Если не определить оператор присваивания,
то он тоже сгенерируется компилятором.
\begin{lstlisting}
struct IntArray {
    IntArray & operator=(IntArray const& a) 
    {
        if (this != &a) {
            delete [] data_;
            size_ = a.size_;
            data_ = new int[size_];
            for (size_t i = 0; i != size_; ++i)
                data_[i] = a.data_[i];
        }
        return *this;
    }
    ...
};
\end{lstlisting}
\end{frame}

\begin{frame}[fragile]{Метод {\tt swap}}{}
\begin{lstlisting}
struct IntArray {
    void swap(IntArray & a) {
        size_t const t1 = size_;
        size_ = a.size_;
        a.size_ = t1;

        int * const t2 = data_;
        data_ = a.data_;
        a.data_ = t2;
    }
    ...
private:
    size_t size_;
    int *  data_;
};
\end{lstlisting}
\end{frame}

\begin{frame}[fragile]{Метод {\tt swap}}{}
Можно использовать функцию {\tt std::swap} и файла \code{algorithm}.
\begin{lstlisting}
#include <algorithm>

struct IntArray {
    void swap(IntArray & a) {
        std::swap(size_, a.size_);
        std::swap(data_, a.data_);
    }
    ...
private:
    size_t size_;
    int *  data_;
};
\end{lstlisting}
\end{frame}

\begin{frame}[fragile]{Реализация оператора = при помощи {\tt swap}}{}
\begin{lstlisting}
struct IntArray {
    IntArray(IntArray const& a) 
        : size_(a.size_), data_(new int[size_]) {
        for (size_t i = 0; i != size_; ++i)
            data_[i] = a.data_[i];    
    }
    IntArray & operator=(IntArray const& a) {
        if (this != &a)
            IntArray(a).swap(*this);
        return *this;
    }
    ...
private:
    size_t size_;
    int *  data_;
};
\end{lstlisting}
\end{frame}

\begin{frame}[fragile]{Запрет копирования объектов}{}
    Для того, чтобы запретить копирование, нужно объявить конструктор копирования
    и оператор присваивания как \code{private} и не определять их.
\begin{lstlisting}
struct IntArray {
    ...
private:
    IntArray(IntArray const& a);
    IntArray & operator=(IntArray const& a);

    size_t size_;
    int *  data_;
};
\end{lstlisting}
\end{frame}

\begin{frame}[fragile]{Методы, генерируемые компилятором}{}
    Компилятор генерирует четыре метода:
    \begin{enumerate}
        \item конструктор по умолчанию,
        \item конструктор копирования,
        \item оператор присваивания,
        \item деструктор.
    \end{enumerate}

    Если потребовалось переопределить конструктор копирования,
    оператор присваивания или деструктор, то нужно 
    переопределить и остальные методы из этого списка. 
\end{frame}
\end{document}
