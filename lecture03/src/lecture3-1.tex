\documentclass{beamer}

\usepackage{mooc}

\usetikzlibrary{arrows, decorations.markings, shapes}
\usetikzlibrary{positioning}

% The face style, can be changed

\title{{\bf Программирование на языке \langcpp\protect\\Лекция
3\protect\vspace{1em}\\}Структуры}

\begin{document}
\begin{frame} 
  \titlepage
\end{frame}

\begin{frame}[fragile]{Зачем группировать данные?}
    Какая должна быть сигнатура у функции, которая вычисляет длину отрезка на
    плоскости?
\begin{lstlisting}
double length(double x1, double y1, 
              double x2, double y2);
\end{lstlisting}
    А сигнатура функции, проверяющей пересечение отрезков?
\begin{lstlisting}
bool intersects(double x11, double y11, 
                double x12, double y12,
                double x21, double y21, 
                double x22, double y22,
                double * xi, double * yi);
\end{lstlisting}
Координаты точек являются логически связанными данными, которые всегда передаются
вместе.\\
Аналогично связанны координаты точек отрезка.
\end{frame}

\begin{frame}[fragile]{Структуры}
    Структуры~--- это способ синтаксически (и физически)
    сгруппировать логически связанные данные.
\begin{lstlisting}
struct Point {
    double x;
    double y;
};

struct Segment {
    Point p1;
    Point p2;
};

double length(Segment s);  

bool intersects(Segment s1, 
                Segment s2, Point * p);
\end{lstlisting}
\end{frame}

\begin{frame}[fragile]{Работа со структурами}
Доступ к полям структуры осуществляется через
оператор '\code{.}':
\begin{lstlisting}
#include <cmath>

double length(Segment s) {
    double dx = s.p1.x - s.p2.x;
    double dy = s.p1.y - s.p2.y;
    return sqrt(dx * dx + dy * dy);
}
\end{lstlisting}
Для указателей на структуры используется оператор '\code{->}'.
\begin{lstlisting}
double length(Segment * s) {
    double dx = s->p1.x - s->p2.x;
    double dy = s->p1.y - s->p2.y;
    return sqrt(dx * dx + dy * dy);
}
\end{lstlisting}
\end{frame}

\begin{frame}[fragile]{Инициализация структур}
Поля структур можно инициализировать подобно массивам:
\begin{lstlisting}
    Point p1  = { 0.4, 1.4 };
    Point p2  = { 1.2, 6.3 };
    Segment s = { p1,  p2  };
\end{lstlisting}
Структуры могут хранить переменные разных типов.
\begin{lstlisting}
struct IntArray2D {
    size_t a;
    size_t b;
    int ** data;
};
\end{lstlisting}
\begin{lstlisting}
    IntArray2D a = {n, m, create_array2d(n, m)};
\end{lstlisting}
\end{frame}
\end{document}
