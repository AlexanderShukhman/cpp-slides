\documentclass[aspectration=1610,t]{beamer}

\usepackage{mooc}

\title{{\bf Программирование на языке \langcpp\protect\\Лекция
8\protect\vspace{1em}\\}Семантика перемещения}

\begin{document}
\begin{frame} 
  \titlepage
\end{frame}

% % % % % % % % % % % % % % % % % %
\begin{frame}[fragile]{Излишнее копирование}
\begin{lstlisting}
struct String {
    String() = default;
    String(String const & s);
    String & operator=(String const & s);
    //...
private:
    char * data_ = nullptr;
    size_t size_ = 0;
};
\end{lstlisting}

\begin{lstlisting}
String getCurrentDateString() {
    String date;
    // date заполняется "21 октября 2015 года"
    return date;
}
\end{lstlisting}

\begin{lstlisting}
String date = getCurrentDateString();
\end{lstlisting}
\end{frame}

% % % % % % % % % % % % % % %
\begin{frame}[fragile]{Перемещающий конструктор и перемещающий~оператор присваивания}
\begin{lstlisting}
struct String 
{
    String (String && s) // \&\& - rvalue reference
        : data_(s.data_)
        , size_(s.size_) {
        s.data_ = nullptr;
        s.size_ = 0;
    }
    String & operator = (String && s) {
        delete [] data_;
        data_ = s.data_;
        size_ = s.size_;
        s.data_ = nullptr;
        s.size_ = 0;
        return *this;
    }
};
\end{lstlisting}
\end{frame}                                

% % % % % % % % % % % % % % % %
\begin{frame}[fragile]{Перемещающие методы при помощи \code{swap}}
\begin{lstlisting}
#include<utility>

struct String 
{
    void swap(String & s) { 
        std::swap(data_, s.data_);
        std::swap(size_, s.size_);
    }

    String (String && s) { 
        swap(s);
    }
    
    String & operator = (String && s) {
        swap(s);
        return *this;
    }
};
\end{lstlisting}
\end{frame}

% % % % % % % % % % % % % % % %
\begin{frame}[fragile]{Использование перемещения}
\begin{lstlisting}
struct String {
    String() = default;
    String(String const & s);  // lvalue-reference
    String & operator=(String const & s);
    String(String && s);       // rvalue-reference
    String & operator=(String && s);
private:
    char * data_ = nullptr;
    size_t size_ = 0;
};
\end{lstlisting}

\begin{lstlisting}
String getCurrentDateString() {
    String date;
    // date заполняется "21 октября 2015 года"
    return std::move(date);
}
\end{lstlisting}

\begin{lstlisting}
String date = getCurrentDateString();
\end{lstlisting}
\end{frame}

% % % % % % % % % % % % % % % %
\begin{frame}[fragile]{Перегрузка с lvalue/rvalue ссылками}
При перегрузке перемещающий метод вызывается для временных объектов
и для явно перемещённых с помощью \texttt{std::move}.
\begin{lstlisting}
String a(String("Hello"));  // перемещение

String b(a);                // копирование

String c(std::move(b));     // перемещение

a = b;                      // копирование

b = std::move(c);           // перемещение

c = String("world");        // перемещение
\end{lstlisting}

Это касается и обычных методов и функций, \\ которые принимают
lvalue/rvalue-ссылки.  
\end{frame}


\begin{frame}[fragile]{Перемещающие особые методы}
    \begin{itemize}
        \item Особые методы класса:
            \begin{itemize}
                \item конструктор по умолчанию,
                \item конструктор копирования,
                \item оператор присваивания,
                \item деструктор,
                \item перемещающий конструктор,
                \item перемещающий оператор присваивания.
             \end{itemize}
                
        \pitem Перемещающие методы генерируются только, если в классе\\ отсутствуют пользовательские копирующие операции,\\ перемещающие операции и деструктор.

        \pitem Генерация копирующих методов для классов с \\ пользовательским конструктором признана устаревшей.
    \end{itemize}
\end{frame}

\begin{frame}[fragile]{Пример: {\tt unique\_ptr}}
    \begin{lstlisting}
#include <memory>
#include "units.hpp"

void foo(std::unique_ptr<Unit> p);

std::unique_ptr<Unit> bar();

int main() {
    // p1 владеет указателем
    std::unique_ptr<Unit> p1(new Elf());  

    // теперь p2 владеет указателем
    std::unique_ptr<Unit> p2(std::move(p1));
    
    p1 = std::move(p2);  // владение передаётся p1
    
    foo(std::move(p1));  // p1 передаётся в foo
    
    p2 = bar();  // std::move не нужен  
} 
    \end{lstlisting}
\end{frame}

\end{document}
