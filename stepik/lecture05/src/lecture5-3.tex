\documentclass{beamer}

\usepackage{mooc}

\usetikzlibrary{arrows, decorations.markings, shapes}
\usetikzlibrary{positioning}

% The face style, can be changed

\title{{\bf Программирование на языке \langcpp\protect\\Лекция
5\protect\vspace{1em}\\}Ключевые слова \code{static} и \code{inline}}

\begin{document}
\begin{frame} 
  \titlepage
\end{frame}

\begin{frame}[fragile]{Глобальные переменные}
Объявление глобальной переменной:
    \begin{lstlisting}
extern int global;

void f () {
    ++global; 
}
    \end{lstlisting}

Определение глобальной переменной:
    \begin{lstlisting}
int global = 10;
    \end{lstlisting}

Проблемы глобальных переменных:
\begin{itemize}
    \item Масштабируемость.
    \item Побочные эффекты.
    \item Порядок инициализации.
\end{itemize}
\end{frame}

\begin{frame}[fragile]{Статические глобальные переменные}
    Статическая глобальная переменная — это глобальная переменная,
    доступная только в пределах модуля.\\[1em]
Определение: 
    \begin{lstlisting}
static int global = 10;

void f () {
    ++global; 
}
    \end{lstlisting}
                
Проблемы статических глобальных переменных:
\begin{itemize}
    \item Масштабируемость.
    \item Побочные эффекты.
\end{itemize}
\end{frame}

\begin{frame}[fragile]{Статические локальные переменные}
    Статическая локальная переменная — это глобальная переменная,
    доступная только в пределах функции.\\[1em]
Время жизни такой переменной~--- от первого вызова функции {\tt next}
до конца программы.
    \begin{lstlisting}
int next(int start = 0) {
    static int k = start;
    return k++;
}
    \end{lstlisting}
                
Проблемы статических локальных переменных:
\begin{itemize}
    \item Масштабируемость.
    \item Побочные эффекты.
\end{itemize}
\end{frame}

\begin{frame}[fragile]{Статические функции}
    Статическая функция,
    доступная только в пределах модуля.\\[1em]
Файл {\tt 1.cpp}:
    \begin{lstlisting}
static void test() {
    cout << "A\n";
}
    \end{lstlisting}
Файл {\tt 2.cpp}:
    \begin{lstlisting}
static void test() {
    cout << "B\n";
}
    \end{lstlisting}
Статические глобальные переменные и статические функции
проходят {\em внутреннюю линковку}.
\end{frame}

\begin{frame}[fragile]{Статические поля класса}
    Статические поля класса — это глобальные переменные,
    определённые внутри класса.\\[1em]
Объявление:
\begin{lstlisting}
struct User {
    ...
private:
    static size_t instances_;
}; 
\end{lstlisting}
Определение:
\begin{lstlisting}
size_t User::instances_ = 0;
\end{lstlisting}
Для доступа к статическим полям не нужен объект.
\end{frame}

\begin{frame}[fragile]{Статические методы}
    Статические методы — это функции, определённые внутри класса и имеющие
    доступ к закрытым полям и методам.\\[1em]
Объявление:
\begin{lstlisting}
struct User {
    ...
    static size_t count() { return instances_; }
private:
    static size_t instances_;
}; 
\end{lstlisting}
Для вызова статических методов не нужен объект.
\begin{lstlisting}
    cout << User::count();
\end{lstlisting}
\end{frame}
    
\begin{frame}[fragile]{Ключевое слово \code{inline}}
    Советует компилятору встроить данную функцию.
\begin{lstlisting}
inline double square(double x) { return x * x; }
\end{lstlisting}
\begin{itemize}
    \item В месте вызова \code{inline}-функции должно 
        быть известно её определение.

    \item \code{inline} функции можно определять 
        в заголовочных файлах.

    \item Все методы, определённые внутри класса, являются
        \code{inline}.

    \item При линковке из всех версий \code{inline}-функции
        (т.е. её код из разных единиц трансляции)
        выбирается только одна.

    \item Все определения одной и той же \code{inline}-функции
        должны быть идентичными.

    \item \code{inline} — это совет компилятору, а не указ.
\end{itemize}
\end{frame}

\begin{frame}[fragile]{Правило одного определения}
Правило одного определения\\ (One Definition Rule, ODR)
\begin{itemize}
    \item В пределах любой единицы трансляции сущности не могут иметь более одного определения. 
    \item В пределах программы глобальные переменные и не-\code{inline} функции не могут иметь больше
        одного определения.
    \item Классы и \code{inline} функции могут определяться в более чем одной единице
        трансляции, но определения обязаны совпадать.
\end{itemize}
Вопрос: к каким проблемам могут привести разные
определения одного класса в разных частях программы?
\end{frame}
\end{document}


\end{document}

 
