\documentclass[aspectration=1610,t]{beamer}

\usepackage{mooc}

\title{{\bf Программирование на языке \langcpp\protect\\Лекция
10\protect\vspace{1em}\\}Спецификация исключений}

\begin{document}
\begin{frame} 
  \titlepage
\end{frame}

\begin{frame}[fragile]{Спецификация исключений}
Устаревшая возможность C++,
позволяющая указать список исключений, которые могут быть выброшены из функции.
\begin{lstlisting}
void foo() throw(std::logic_error) {
    if (...) throw std::logic_error();
    if (...) throw std::runtime_error();
}               
\end{lstlisting}
Если сработает второй {\tt if}, то программа аварийно завершится.
\begin{lstlisting}
void foo() {
    try {
        if (...) throw std::logic_error();
        if (...) throw std::runtime_error();
    } catch (std::logic_error & e) { 
        throw e;
    } catch (...) { 
        terminate();
    }
}
\end{lstlisting}
\end{frame}


\begin{frame}[fragile]{Ключевое слово \texttt{noexcept}}
\begin{itemize}
    \item Используется в двух значениях:
    \begin{itemize}
        \item Спецификатор функции, которая не бросает исключение.
        \item Оператор, проверяющий во время компиляции, что выражение
            специфицированно как небросающее исключение.
    \end{itemize}

\item Если функцию со спецификацией \code{noexcept} покинет\\ 
    исключение, то стек не обязательно будет свёрнут, \\перед тем как программа завершится.\\  
    В отличие от аналогичной ситуации с \code{throw}\texttt{()}.

\item Использование спецификации \code{noexcept} позволяет\\
    компилятору лучше оптимизировать код, т.к. 
    не нужно\\ заботиться о сворачивании стека.
\end{itemize}

\end{frame}

\begin{frame}[fragile]{Использование \texttt{noexcept}}
    \begin{lstlisting}
void no_throw() noexcept;
void may_throw();

// копирующий конструктор noexcept
struct NoThrow { int m[100] = {}; };

// копирующий конструктор noexcept(false)
struct MayThrow { std::vector<int> v; };
    \end{lstlisting}

\begin{lstlisting}
MayThrow mt; 
NoThrow  nt;

bool a = noexcept(may_throw());  // false
bool b = noexcept(no_throw());   // true

bool c = noexcept(MayThrow(mt)); // false
bool d = noexcept(NoThrow(nt));  // true
\end{lstlisting}
\end{frame}

\begin{frame}[fragile]{Условный \texttt{noexcept}}
    В спецификации \texttt{noexcept} можно использовать\\
    условные выражения времени компиляции.
\begin{lstlisting}
template <class T, size_t N>
void swap(T (&a)[N], T (&b)[N]) 
        noexcept(noexcept(swap(*a, *b)));

template <class T1, class T2>
struct pair {
    void swap(pair & p) 
        noexcept(noexcept(swap(first, p.first)) &&
                 noexcept(swap(second, p.second)))
    {
        swap(first, p.first);
        swap(second, p.second);
    }

    T1 first;
    T2 second;
};
\end{lstlisting}
\end{frame}

% TODO: добавить задачу про declval

\begin{frame}[fragile]{Зависимость от \texttt{noexcept}}
    Проверка \code{noexcept} используется в стандартной 
    библиотеке для обеспечения строгой гарантии безопасности исключений с помощью
    \texttt{std::move\_if\_noexcept} (например, \texttt{vector::push\_back}).

\begin{lstlisting}
struct Bad {
    Bad() {}
    Bad(Bad&&);      // может бросить
    Bad(const Bad&); // не важно
};
struct Good {
    Good() {}
    Good(Good&&) noexcept; // не бросает
    Good(const Good&) ;    // не важно
};
\end{lstlisting}
\begin{lstlisting}
Good g1;
Bad  b1;
Good g2 = std::move_if_noexcept(g1); // move
Bad  b2 = std::move_if_noexcept(b1); // copy
\end{lstlisting}
    
\end{frame}

\end{document}
