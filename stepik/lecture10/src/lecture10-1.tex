\documentclass[aspectration=1610,t]{beamer}

\usepackage{mooc}

\title{{\bf Программирование на языке \langcpp\protect\\Лекция
10\protect\vspace{1em}\\}Обработка ошибок}

\begin{document}
\begin{frame} 
  \titlepage
\end{frame}

\newcommand{\fakeitem}{{\color{MOOCBlue}\textbullet}\ \ }

\begin{frame}[fragile]{Логические ошибки и исключительные ситуации}
\begin{itemize}
\item \textbf{Логические ошибки}.\\
Ошибки в логике работы программы, которые происходят\\
из-за неправильно написанного кода, т.е. это ошибки программиста:
\begin{itemize} 
    \item выход за границу массива,
    \item попытка деления на ноль,
    \item обращение по нулевому указателю,
    \item \ldots
\end{itemize}
\pause
\item \textbf{Исключительные ситуации}.\\
Ситуации, которые требуют особой обработки.\\
Возникновение таких ситуаций~— это ,,нормальное``\\ поведение программы.
\begin{itemize} 
    \item ошибка записи на диск,
    \item недоступность сервера,
    \item неправильный формат файла,
    \item \ldots
\end{itemize}
\end{itemize}
\end{frame}

\begin{frame}[fragile]{Выявление логических ошибок на этапе разработки}
    \fakeitem Оператор \code{static\_assert}.\\
    \mbox{}
    \begin{lstlisting}
#include<type_traits>



template<class T> 
void countdown(T start) {
    static_assert(std::is_integral<T>::value
               && std::is_signed<T>::value, 
                  "Requires signed integral type");
                  
    while (start >= 0) {
        std::cout << start-- << std::endl;
    }
}
    \end{lstlisting}
\end{frame}

\begin{frame}[fragile]{Выявление логических ошибок на этапе разработки}
    \fakeitem Оператор \code{static\_assert}.\\
    \fakeitem Макрос \texttt{assert}.
    \begin{lstlisting}
#include<type_traits>
//\#define NDEBUG
#include <cassert>

template<class T> 
void countdown(T start) {
    static_assert(std::is_integral<T>::value
               && std::is_signed<T>::value, 
                  "Requires signed integral type");
    assert(start >= 0);
    while (start >= 0) {
        std::cout << start-- << std::endl;
    }
}
    \end{lstlisting}
\end{frame}

\begin{frame}[fragile]{Способы сообщения об ошибке}
    \begin{lstlisting}
size_t write(string file, string data);
    \end{lstlisting}

        \pause\fakeitem Возврат статуса операции:
    \begin{lstlisting}
bool write(string file, string data, size_t & bytes);
    \end{lstlisting}

        \pause\fakeitem Возврат кода ошибки:
    \begin{lstlisting}
int const OK = 0, IO_WRITE_FAIL = 1, IO_OPEN_FAIL = 2;
int write(string file, string data, size_t & bytes);
    \end{lstlisting}

        \pause\fakeitem Глобальная переменная для кода ошибки:
    \begin{lstlisting}
size_t write(string file, string data);
    \end{lstlisting}

    \begin{lstlisting}
size_t bytes = write(f, data);
if (errno) {
    cerr << strerror(errno);
    errno = 0;
}
\end{lstlisting}
\pause\fakeitem Исключения.
\end{frame}

\begin{frame}[fragile]{Исключения}
\begin{lstlisting}
size_t write(string file, string data) {
    if (!open(file)) throw FileOpenError(file);
    //...
}
double safediv(int x, int y) {
    if (y == 0) throw MathError("Division by zero");
    return double(x) / y;
}
void write_x_div_y(string file, int x, int y) {
    try { 
        write(file, to_string(safediv(x, y)));
    } catch (MathError & s) { 
        // обработка ошибки в safediv
    } catch (FileError & e) { 
        // обработка ошибки в write
    } catch (...) { 
        // все остальные ошибки
    }
}
\end{lstlisting}
\end{frame}

\begin{frame}[fragile]{Stack unwinding}
При возникновении исключения объекты на стеке
уничтожаются\\ в естественном (обратном) порядке.
\begin{lstlisting}
void foo() {
    D d;
    E e(d);
    if (!e) throw F();
    G g(e);
}
void bar() {
    A a;
    try {
        B b;
        foo();
        C c;
    } catch (F & f) {
        // обработка и пересылка
        throw f;
    }
}
\end{lstlisting}
\end{frame}

\begin{frame}[fragile]{Почему не стоит бросать встроенные типы}
\begin{lstlisting}
int foo() {
    if (...) throw -1;
    if (...) throw 3.1415;
}
void bar(int a) {
    if (a == 0) throw string("Not my fault!");
}
int main () {
    try { bar(foo());
    } catch (string & s) {
        // только текст
    } catch (int a) { 
        // мало информации
    } catch (double d) { 
        // мало информации
    } catch (...) { 
        // нет информации 
    }
}
\end{lstlisting}
\end{frame}

\begin{frame}[fragile]{Стандартные классы исключений}
    Базовый класс для всех исключений (в {\tt <exception>}):
    \begin{lstlisting}
struct exception {
  virtual ~exception();
  virtual const char* what() const;
}; 
    \end{lstlisting}

    Стандартные классы ошибок (в {\tt <stdexcept>}):\\
    \begin{itemize}
        \item {\tt logic\_error}: 
     {\tt domain\_error}, {\tt invalid\_argument}, {\tt length\_error}, {\tt
     out\_of\_range}\\
        \item {\tt runtime\_error}: 
        {\tt range\_error},
        {\tt overflow\_error}, 
        {\tt underflow\_error}
    \end{itemize}

\begin{lstlisting}
int main() {
    try { ... }
    catch (std::exception const& e) {
       std::cerr << e.what() << '\n';
    }
}
\end{lstlisting}
\end{frame}
 

\begin{frame}[fragile]{Исключения в стандартной библиотеке}
    \begin{itemize}
        \item Метод \texttt{at} контейнеров {\tt array}, {\tt vector}, {\tt deque}, {\tt basic\_string}, {\tt bitset}, {\tt map}, {\tt unordered\_map} бросает {\tt out\_of\_range}.

        \item Оператор \code{new} {\tt T} бросает {\tt bad\_alloc}.\\
            Оператор {\tt \code{new} (std::nothrow) T} в возвращает 0.
            
        \item Оператор \code{typeid} от разыменованного нулевого указателя бросает {\tt bad\_typeid}.

        \item Потоки ввода-вывода. 
    \end{itemize}
            \begin{lstlisting}
std::ifstream file;
file.exceptions( std::ifstream::failbit 
               | std::ifstream::badbit );
try {
    file.open ("test.txt");
    cout << file.get() << endl;
    file.close();
}
catch (std::ifstream::failure const& e) {
    cerr << e.what() << endl;
}
            \end{lstlisting}
\end{frame}


\begin{frame}[fragile]{Как обрабатывать ошибки?}
    Есть несколько ,,правил хорошего тона``.
    \begin{itemize}
        \item Разделяйте ,,ошибки программиста``
            и ,,исключительные ситуации``.
            
        \item Используйте \texttt{assert} и \texttt{static\_assert} 
        для выявления ошибок на этапе разработки.
%    There is no point in handling such an error by using an exception because the error indicates that something in the code has to be fixed, and doesn't represent a condition that the program has to recover from at run time.
        
        \item В пределах одной логической части кода 
        обрабатывайте\\ ошибки централизованно и однообразно.

        \item Обрабатывайте ошибки там, где их можно обработать.
        
        \item Если в данном месте ошибку не обработать, \\то
             пересылайте её выше при помощи исключения.
 
        \item Бросайте только стандартные классы исключений\\ или производные от них.

        \item Бросайте исключения по значению, \\а отлавливайте по ссылке.

        \item Отлавливайте все исключения в точке входа.
       
    \end{itemize}
    
    %Use asserts to check for errors that should never occur. Use exceptions to check for errors that might occur, for example, errors in input validation on parameters of public functions. For more information, see the section titled Exceptions vs. Assertions.

    %Use exceptions when the code that handles the error might be separated from the code that detects the error by one or more intervening function calls. Consider whether to use error codes instead in performance-critical loops when code that handles the error is tightly-coupled to the code that detects it.

    %For every function that might throw or propagate an exception, provide one of the three exception guarantees: the strong guarantee, the basic guarantee, or the nothrow (noexcept) guarantee. 

    %Throw exceptions by value, catch them by reference. Don’t catch what you can't handle. 
    
    %Don't use exception specifications, which are deprecated in C++11.
    
    %Use standard library exception types when they apply. Derive custom exception types from the exception Class hierarchy. 

    %Don't allow exceptions to escape from destructors or memory-deallocation functions.
\end{frame}
\end{document}
