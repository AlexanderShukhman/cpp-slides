\documentclass{beamer}

\usepackage{mooc}

\usetikzlibrary{arrows, decorations.markings, shapes}
\usetikzlibrary{positioning}

% The face style, can be changed

\title{{\bf Программирование на языке \langcpp\protect\\Лекция
4\protect\vspace{1em}\\}Наследование}

\begin{document}
\begin{frame} 
  \titlepage
\end{frame}

\begin{frame}[fragile]{Наследование}
Наследование — это механизм, позволяющий создавать производные классы,
расширяя уже существующие.
\begin{lstlisting}
struct Person {
    string name() const { return name_; }
    int     age() const { return age_; }
private:
    string name_;
    int age_;    
};

struct Student : Person {
    string university() const { return uni_; }
private:
    string uni_;
};
\end{lstlisting}
\end{frame}

\begin{frame}[fragile]{Класс-наследник}
У объектов класса-наследника можно вызывать публичные
методы родительского класса. 
\begin{lstlisting}
Student s;
cout << s.name() << endl
     << s.age()  << endl
     << s.university() << endl;
\end{lstlisting}
Внутри объекта класса-наследника хранится экземпляр родительского
класса.
\begin{center}
\begin{tikzpicture}[                                                          
      start chain=1 going right, node distance=-0.15mm,
    ]
    \tikzstyle{cell} = [rectangle,minimum width=4mm, minimum height=6mm,on chain=1,draw];
    \tikzstyle{arrow} = [>=stealth']
    \tikzstyle{mbrace} = [decorate,decoration={brace,amplitude=5pt}];

    \node[cell] (p1) {\tt name\_};
    \node[cell] (p2) {\tt age\_};
    \node[minimum width=1cm, on chain=1] () {};
    \draw[mbrace] 
       (p2.south east)--(p1.south west) node[midway,anchor=west,xshift=6pt] {};
    \node[above of=p1, shift={(6mm,-7mm)}] () {\tt Person};
    
    \node[cell] (s1) {\tt name\_};
    \node[cell] (s2) {\tt age\_};
    \node[cell] (s3) {\tt uni\_};
    \draw[mbrace] 
       (s1.north west)--(s2.north east) node[midway,anchor=west,xshift=6pt] {};
    \node[above of=s1, shift={(6mm,7mm)}] () {\tt Person};
    \draw[mbrace] 
       (s3.south east)--(s1.south west) node[midway,anchor=east,xshift=6pt] {};
    \node[below of=s2, shift={(0mm,-7mm)}] () {\tt Student};
\end{tikzpicture}
\end{center}
\end{frame}

\begin{frame}[fragile]{Создание/удаление объекта производного класса}
При создании объекта производного класса сначала вызывается конструктор
родительского класса.
\begin{lstlisting}
struct Person {
    Person(string name, int age)
        : name_(name), age_(age)
    {}
    ...
};
struct Student : Person {
    Student(string name, int age, string uni)
        : Person(name, age), uni_(uni)
    {}
    ...
};
\end{lstlisting}
После деструктора {\tt Student} вызывается деструктор {\tt Person}.
\end{frame}
\begin{frame}[fragile]{Приведения}
Для производных классов определены следующие приведения:
\begin{lstlisting}
Student s("Alex", 21, "Oxford");
Person & l = s;  // \verb!Student & -> Person &!
Person * r = &s; // \verb!Student * -> Person *!
\end{lstlisting}
Поэтому объекты класса-наследника могут присваиваться
объектам родительского класса:
\begin{lstlisting}
Student s("Alex", 21, "Oxford");
Person p = s; // \verb!Person("Alex", 21);!
\end{lstlisting}
При этом копируются только поля класса-родителя (срезка).

(Т.е. в данном случае вызывается конструктор копирования
{\tt Person(Person \code{const}\& p)}, который не знает про {\tt uni\_}.)
\end{frame}

\begin{frame}[fragile]{Модификатор доступа \code{protected}}
\begin{itemize}
    \item Класс-наследник не имеет доступа к {\tt private}-членам 
        родительского класса.
        
    \item Для определения закрытых членов класса доступных наследникам
        используется модификатор {\tt protected}.
\begin{lstlisting}
struct Person {
    ...
protected:
    string name_;
    int age_;    
};

struct Student : Person {
    ... // можно менять поля name\_ и age\_
};
\end{lstlisting}
\end{itemize}
\end{frame}
\end{document}

