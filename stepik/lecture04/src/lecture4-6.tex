\documentclass{beamer}

\usepackage{mooc}

\usetikzlibrary{arrows, decorations.markings, shapes}
\usetikzlibrary{positioning}

% The face style, can be changed

\title{{\bf Программирование на языке \langcpp\protect\\Лекция
4\protect\vspace{1em}\\}Особенности наследования в \langcpp}

\begin{document}
\begin{frame} 
  \titlepage
\end{frame}

\begin{frame}[fragile]{Модификаторы при наследовании}
    При наследовании можно использовать модификаторы доступа:
\begin{lstlisting}
    struct A {};
    struct B1 : public A {};
    struct B2 : private A {};
    struct B3 : protected A {};
\end{lstlisting}
Для классов, объявленных как \code{struct}, по-умолчанию используется
\code{public}, для объявленных как \code{class}~— \code{private}.
\vspace{3mm}

{\bf Важно:} {\em отношение наследования} (в терминах ООП) задаётся только \code{public}-наследованием.
\vspace{3mm}

Использование \code{private}- и \code{protected}-наследований целесообразно,
если необходимо не только агрегировать другой класс, но и переопределить его виртуальные методы.
\end{frame}

\begin{frame}[fragile]{Переопределение \code{private} виртуальных методов}
    \begin{lstlisting}[basicstyle=\fontsize{9pt}{1em}\ttfamily,commentstyle=\fontsize{9pt}{1em}\ttfamily\color{MOOCGreen}]
struct NetworkDevice {
    void send(void * data, size_t size) {
        log("start sending");
        send_impl(data, size);
        log("stop sending");
    }
    ...
private:
    virtual void send_impl(void * data, size_t size) 
    {...}
};

struct Router : NetworkDevice {
private:
    void send_impl(void * data, size_t size) {...}
};
    \end{lstlisting}
\end{frame}

\begin{frame}[fragile]{Реализация чистых виртуальных методов}
    Чистые виртуальные методы могут иметь определения:
    \begin{lstlisting}[basicstyle=\fontsize{9pt}{1em}\ttfamily,commentstyle=\fontsize{9pt}{1em}\ttfamily\color{MOOCGreen}]
struct NetworkDevice {
    virtual void send(void * data, size_t size) = 0;
    ...
};

void NetworkDevice::send(void * data, size_t size) {
    ...
}

struct Router : NetworkDevice {

    void send(void * data, size_t size) {
        // невиртуальный вызов
        NetworkDevice::send(data, size);
    }
};
    \end{lstlisting}
\end{frame}

\begin{frame}[fragile]{Интерфейсы}{}
    {\em Интерфейс}~— это абстрактный класс, у которого отсутствуют поля,
    а все методы являются чистыми виртуальными.

    \begin{lstlisting}
struct IConvertibleToString {
    virtual ~IConvertibleToString() {}
    virtual string toString() const = 0;
};
    \end{lstlisting}
    \begin{lstlisting}
struct IClonable {
    virtual ~IClonable() {}
    virtual IClonable * clone() const = 0;
};
    \end{lstlisting}
    \begin{lstlisting}
struct Person : IClonable {
    Person * clone() {return new Person(*this);}
};
    \end{lstlisting}
\end{frame}

\begin{frame}[fragile]{Множественное наследование}

    В \langcpp разрешено множественное наследование.
\begin{lstlisting}
    struct Person {};
    struct Student : Person {};
    struct Worker  : Person {};
    struct WorkingStudent : Student, Worker {};
\end{lstlisting}
    Стоит избегать {\em наследования реализаций} более чем от одного 
    класса, вместо этого использовать интерфейсы.

\begin{lstlisting}
    struct IWorker {};
    struct Worker : Person, IWorker {};
    struct Student : Person {};
    struct WorkingStudent : Student, IWorker {}
\end{lstlisting}

Множественное наследование — это отдельная большая тема.
\end{frame}


\end{document}

 
