\documentclass[aspectration=1610,t]{beamer}

\usepackage{mooc}

\title{{\bf Программирование на языке \langcpp\protect\\Лекция
7\protect\vspace{1em}\\}Преобразование в стиле \langcpp}

\begin{document}
\begin{frame} 
  \titlepage
\end{frame}

\begin{frame}[fragile]{Преобразование в стиле \langc}{}
В \langc этот оператор преобразует встроенные типы и указатели.
\begin{lstlisting}
int a = 2;
int b = 3;
    
// int $\to$ double
double size = ((double)a) / b * 100; 

// double $\to$ int
void * data = malloc(sizeof(double) * int(size));

// void * $\to$ double *
double * array = (double *)data; 

// double * $\to$ char *
char * bytes = (char *)array;
\end{lstlisting}
\end{frame}

\begin{frame}[fragile]{Преобразования в \langcpp: \code{static\_cast}}
    Служит для преобразований связанных типов:
    \begin{itemize}
        \pitem Стандартные преобразования.
        \begin{itemize}
\item Преобразования числовых типов.
\begin{lstlisting}
double s = static_cast<double>(2) / 3 * 100; 
s = static_cast<int>(d);
\end{lstlisting}
\item Указатель/ссылка на производный класс в указатель/ссылку на базовый класс.
\item \texttt{T*} в \code{void}\texttt{*}.
        \end{itemize}
        \pitem Явное (пользовательское) приведение типа:
\begin{lstlisting}
Person p = static_cast<Person>("Ivan");
\end{lstlisting}
        \pitem Обратные варианты стандартных преобразований:
            \begin{itemize}
                %\item целочисленные типы в перечисляемые,
                \item Указатель/ссылка на базовый класс в указатель/ссылку на производный класс (преобразование вниз, downcast),
                %\item {\tt T Base::*} в {\tt T Derived::*},
                \item \code{void}\texttt{*} в любой {\tt T*}.                          
            \end{itemize}
        \pitem Преобразование к \code{void}.
    \end{itemize}
\end{frame}

\begin{frame}[fragile]{Преобразования в \langcpp: \code{const\_cast}}
    Служит для снятия/добавления константности.\pause
\begin{lstlisting}
    void foo(double const& d) {
        const_cast<double &>(d) = 10;
    } 
\end{lstlisting}
Использование \code{const\_cast}~--- признак плохого дизайна.\\
\medskip\pause
Кроме редких исключений:
\begin{lstlisting}
T & operator[](size_t i) {
    return const_cast<T &>(
           const_cast<Vector const &>(*this)[i]);
}

T const & operator[](size_t i) const {
    assert(i < size_);
    return data_[i];
}
\end{lstlisting}
\end{frame}

\begin{frame}[fragile]{Преобразования в \langcpp: \code{reinterpret\_cast}}{}
    Служит для преобразований указателей и ссылок на несвязанные типы.\pause
    
\begin{lstlisting}
void send(char const * data, size_t length);
char * receive(size_t * length);
\end{lstlisting}\pause
\begin{lstlisting}
double * m = static_cast<double*>
                 (malloc(sizeof(double) * 100)); 
... // инициализация m
char * mc = reinterpret_cast<char *>(m);
send(mc, sizeof(double) * 100);
\end{lstlisting}\pause
\begin{lstlisting}
size_t length = 0;
double * m = reinterpret_cast<double*>
                             (receive(&length));
length = length / sizeof(double);
\end{lstlisting}\pause

Поможет преобразовать указатель в число.
\begin{lstlisting}
size_t ms = reinterpret_cast<size_t>(m);
\end{lstlisting}
\end{frame}

\begin{frame}[fragile]{Границы применимости преобразования в стиле \langc}{}
  \begin{itemize}
    \pitem Преобразования в стиле \langc может заменить любое из рассмотренных преобразований: 
    \begin{itemize}
        \item \code{static\_cast},
        \item \code{reinterpret\_cast},
        \item \code{const\_cast}.
    \end{itemize}

    \pitem Преобразования в стиле \langc можно использовать для
    \begin{itemize}
        \item преобразование встроенных типов,
        \item преобразование указателей на явные типы.
    \end{itemize}

    \pitem Преобразования в стиле \langc не стоит использовать:
    \begin{itemize}
        \item с пользовательскими типами и указателями на них,
        \item в шаблонах.        
    \end{itemize}
  \end{itemize}
\end{frame}

\begin{frame}[fragile]{Когда преобразование в стиле \langc приводит к ошибке}{}
\begin{lstlisting}
// abc.h
struct A { int a; };

struct B {};

struct C : A, B {};
\end{lstlisting}

\pause
\begin{minipage}{.45\textwidth}
\begin{lstlisting}
#include "abc.h"

C * foo(B * b) {
    return (C *)b;
}
\end{lstlisting}

\only<4>{\raggedright 
Если в этой точке известны определения классов,\\ то происходит преобразование \code{static\_cast}.}

\end{minipage}\pause\hfill
\begin{minipage}{.51\textwidth}
\begin{lstlisting}
struct A; struct B; struct C;

C * foo(B * b) {
    return (C *)b;
}
\end{lstlisting}
\only<4>{\raggedright Если известны только\\ объявления, то происходит преобразование \code{reinterpret\_cast}.}
\end{minipage}

\end{frame}
\end{document}
