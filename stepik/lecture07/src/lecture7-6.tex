\documentclass[aspectration=1610,t]{beamer}

\usepackage{mooc}

\title{{\bf Программирование на языке \langcpp\protect\\Лекция
7\protect\vspace{1em}\\}Указатели на методы и поля класса}

\begin{document}
\begin{frame} 
  \titlepage
\end{frame}


\begin{frame}[fragile]{Указатели на методы: параметризация алгоритмов}
Для вызова метода по указателю нужен объект.
    \begin{lstlisting}
struct Unit 
{
    virtual unsigned id()    const;
    virtual unsigned hp()    const;
};

typedef unsigned (Unit::*UnitMethod)() const;

void sort(Unit* p, Unit* q, UnitMethod mtd)
{
    for (Unit * m = q; m != p; --m)
        for (Unit * r = p; r + 1 < m; ++r)
            if ( (r->*mtd)() > ((r+1)->*mtd)() )
                swap(*r, *(r+1));
}
    \end{lstlisting}
    \begin{lstlisting}
    sort(p, q, &Unit::hp);
    \end{lstlisting}
\end{frame}


\begin{frame}[fragile]{Указатели на поля: параметризация алгоритмов}
Для обращения к полю по указателю нужен объект.
    \begin{lstlisting}
struct Unit 
{
        unsigned id;
        unsigned hp;
};

typedef unsigned Unit::*UnitField;

void sort(Unit* p, Unit* q, UnitField f)
{
    for (Unit * m = q; m != p; --m)
        for (Unit * r = p; r + 1 < m; ++r)
            if ( (r->*f) > ((r+1)->*f) )
                swap(*r, *(r+1));
}
    \end{lstlisting}
    \begin{lstlisting}
    sort(p, q, &Unit::id);
    \end{lstlisting}
\end{frame}

\begin{frame}[fragile]{Резюме по синтаксису}
Указатели на методы и поля класса. 
    \begin{lstlisting}
struct Unit 
{ 
    unsigned id() const;
    unsigned hp;
};
    \end{lstlisting}
    \begin{lstlisting}
unsigned (Unit::*mtd)() const = &Unit::id;
unsigned  Unit::*fld          = &Unit::hp;

Unit u;
Unit * p = &u;

(u.*mtd)() == (p->*mtd)();
(u.*fld)   == (p->*fld);
    \end{lstlisting}

\end{frame}

%TODO: сделать текстовым степом
%\begin{frame}[fragile]{Шаблон проектирования Listener}
%    Решение с помощью ООП:
%\begin{lstlisting}
%struct Button;
%
%struct ButtonListener {
%    virtual void onButtonClick(Button * b, bool down) = 0;
%    virtual ~ButtonListener(){}
%};
%struct Button {
%    void subscribe( ButtonListener * bl );
%};
%    \end{lstlisting}
%
%Решение с помощью указателей на функции:
%    \begin{lstlisting}
%struct Button;
%typedef void (*ButtonProc)(Button *, bool, void *);
%struct Button {
%    void subscribe( ButtonProc bp, void * arg );
%};
%    \end{lstlisting}
%\end{frame}

\begin{frame}[fragile]{Как такие указатели устроены?}
    \pause
    \begin{block}{Что хранится в указателе на функцию?}
        \pause Хранится адрес функции.
    \end{block}

    \pause
    \begin{block}{Что хранится в указателе на поле класса?}
        \pause
        Хранится смещение поля от начала объекта.
    \end{block}

    \pause
    \begin{block}{Что хранится в указателе на метод?}
        \pause
        Там хранятся:
        \begin{enumerate}
            \pause\item адрес метода,
            \pause\item номер в таблице виртуальных методов,
            \pause\item смещение.
        \end{enumerate}
    \end{block}
\end{frame}

\begin{frame}[fragile]{Зачем нужно смещение?}
\begin{lstlisting}
struct Elf {
    string secretName;
};

struct Archer {
    unsigned arrows() { return arrows_; }
    unsigned arrows_;
};

struct ElfArcher : Elf, Archer {};

void foo() {
    ElfArcher ea;
    unsigned (ElfArcher::*m)() = &Archer::arrows;   
    (ea.*m)();
}
\end{lstlisting}
\end{frame}

\begin{frame}[fragile]{Важные моменты}
\begin{itemize}
    \pause\item Использование неинициализированных указателей на функции и методы
        влечёт неопределённое поведение.

    \pause\item Для использования указателей на методы и поля классов нужны экземпляры
        этих классов.

    \pause\item Указатели на методы и поля класса ни к чему не приводятся.
    
    \pause\item Указатель на статический метод — это указатель на функцию, а указатель на статическое поле — это обычный указатель.

    \pause\item В шаблонном коде указатель на функцию ведёт себя так же, как
        и объект класса с оператором \texttt{()}. Это позволяет использовать
        указатели на функции в качестве функторов.

        \pause\item Используйте \code{typedef}! $=)$.
\end{itemize}
\end{frame}

%TODO: сделать текстовый степ expat

\end{document}



