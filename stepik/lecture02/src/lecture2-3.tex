\documentclass{beamer}

\usepackage{mooc}

\usetikzlibrary{arrows, decorations.markings, shapes}
\usetikzlibrary{positioning}

% The face style, can be changed

\title{{\bf Программирование на языке \langcpp\protect\\Лекция
2\protect\vspace{1em}\\}Указатели и массивы}

\begin{document}
\begin{frame} 
  \titlepage
\end{frame}

%Использование памяти в работающей программе.
%Стек и куча.

\begin{frame}[fragile]{Указатели}
    \begin{itemize}
        \item Указатель — это переменная, хранящая адрес некоторой 
            ячейки памяти.
           
        \item Указатели являются типизированными.
\begin{lstlisting}
int   i = 3; // переменная типа int
int * p = 0; // указатель на переменную типа int
\end{lstlisting}
        
        \item Нулевому указателю (которому присвоено значение \code{0}) не~соответствует никакая ячейка памяти.

        \item Оператор взятия адреса переменной \code{\&}.

        \item Оператор разыменования \code{*}.

\begin{lstlisting}
p  = &i; // указатель p указывает на переменную i
*p = 10; // изменяется ячейка по адресу p, т.е. i
\end{lstlisting}
    \end{itemize}

\end{frame}

\begin{frame}[fragile]{Передача параметров по указателю}
Рассмотрим функцию, меняющую параметры местами:
\begin{lstlisting}
void swap (int a, int b) {
    int t = a;
    a = b;
    b = t;
}

int main() {
    int k = 10, m = 20;
    swap (k, m);
    cout << k << ' ' << m << endl; // 10 20
    return 0;
}
\end{lstlisting}

{\tt swap} изменяет локальные копии переменных {\tt k} и {\tt m}.
\end{frame}

\begin{frame}[fragile]{Передача параметров по указателю}
    Вместо значений типа \code{int} будем передавать указатели.
\begin{lstlisting}
void swap (int * a, int * b) {
    int t = *a;
    *a = *b;
    *b = t;
}

int main() {
    int k = 10, m = 20;
    swap (&k, &m);
    cout << k << ' ' << m << endl; // 20 10
    return 0;
}
\end{lstlisting}

{\tt swap} изменяет переменные {\tt k} и {\tt m} по указателям на них.
\end{frame}


\begin{frame}[fragile]{Массивы}
    \begin{itemize}
        \item Массив~— это набор однотипных элементов, расположенных в памяти
            друг за другом, доступ к которым осуществляется по индексу.

        \item \langcpp позволяет определять массивы на стеке.
\begin{lstlisting}
// массив 1 2 3 4 5 0 0 0 0 0
int m[10] = {1, 2, 3, 4, 5}; 
\end{lstlisting}

        \item Индексация массива начинается с \code{0}, последний элемент
            массива длины {\tt n} имеет индекс {\tt n - 1}.
\begin{lstlisting}
for (int i = 0; i < 10; ++i)
    cout << m[i] << ' '; 
cout << endl;
\end{lstlisting}
    \end{itemize}
\end{frame}

\begin{frame}[fragile]{Связь массивов и указателей}
    \begin{itemize}
        \item Указатели позволяют передвигаться по массивам.
        \item Для этого используется арифметика указателей:
\begin{lstlisting}
int m[10] = {1, 2, 3, 4, 5}; 
int * p = &m[0]; // адрес начала массива
int * q = &m[9]; // адрес последнего элемента
\end{lstlisting}
            \begin{itemize}
                \item {\tt (p + k)} — сдвиг на {\tt k} ячеек типа \code{int} вправо.
                \item {\tt (p - k)} — сдвиг на {\tt k} ячеек типа \code{int} влево.
                \item {\tt (q - p)} — количество ячеек между
                    указателями.
                \item {\tt p[k]} эквивалентно {\tt *(p + k)}.
            \end{itemize}
    \end{itemize}
\begin{center}
\begin{tikzpicture}[
      start chain=1 going right,node distance=-0.15mm
    ]
    \tikzstyle{cell} = [rectangle,minimum width=5mm, minimum height=5mm,draw,on chain=1];
    \tikzstyle{arrow} = [>=stealth',shorten >=2pt, shorten <=2pt]

    \foreach \i in {1,2}
        \node[cell,dashed] {};
        
    \node[cell,dashed] (b) {};

    \foreach \i in {1,2,3,4,5}
        \node[cell,thick] (m\i) {\i};
        
    \foreach \i in {0,0,0}
        \node[cell,thick] {\i};

    \node[cell,thick] (e) {0};
    \node[cell,thick] {0};

    \node[below of=b,yshift=-10mm] (p) {{\tt p}};
    \path [arrow,->] (p) edge node {} (b.south east);

    \node[below of=e,yshift=-10mm] (q) {{\tt q}};
    \path [arrow,->] (q) edge node {} (e.south east);

    \node[below of=m2,yshift=-10mm] (p2) {{\tt (p + 2)}};
    \path [arrow,->] (p2) edge node {} (m2.south east);

    \node[below of=m5,yshift=-10mm] (q4) {{\tt (q - 4)}};
    \path [arrow,->] (q4) edge node {} (m5.south east);

    \foreach \i in {1,2,3}
        \node[cell,dashed] {};
\end{tikzpicture}
\end{center}
\end{frame}

\begin{frame}[fragile]{Примеры}
Заполнение массива:
\begin{lstlisting}
int m[10] = {}; // изначально заполнен нулями
// \hspace{2cm} \&m[0] \hspace{13mm}  \&m[9]
for (int * p = m ; p <= m + 9; ++p )
    *p = (p - m) + 1;    
// Массив заполнен числами от 1 до 10
\end{lstlisting}
Передача массива в функцию:
\begin{lstlisting} 
int max_element (int * m, int size) {
    int max = *m; 
    for (int i = 1; i < size; ++i)
        if (m[i] > max)
            max = m[i];
    return max;
}
\end{lstlisting}
\end{frame}


\end{document}
