\documentclass{beamer}

\usepackage{mooc}

\usetikzlibrary{arrows, decorations.markings, shapes}
\usetikzlibrary{positioning}

% The face style, can be changed

\title{{\bf Программирование на языке \langcpp\protect\\Лекция
2\protect\vspace{1em}\\}Использование указателей}

\begin{document}
\begin{frame} 
  \titlepage
\end{frame}

%Использование памяти в работающей программе.
%Стек и куча.

\begin{frame}[fragile]{Два способа передачи массива}
\small
Функция для поиска элемента в массиве:
\begin{lstlisting}
bool contains(int * m, int size, int value) {
    for (int i = 0; i != size; ++i)
        if (m[i] == value)
            return true;
    return false;
}
bool contains(int * p, int * q, int value) {
    for (; p != q; ++p)
        if (*p == value)
            return true;
    return false;
}
\end{lstlisting}
\vspace{-5mm}
\begin{center}
\begin{tikzpicture}[
      start chain=1 going right,node distance=-0.15mm
    ]
    \tikzstyle{cell} = [rectangle,minimum width=5mm, minimum height=5mm,draw,on chain=1];
    \tikzstyle{arrow} = [>=stealth',shorten >=2pt, shorten <=2pt]

    \foreach \i in {1,2}
        \node[cell,dashed] {};
        
    \node[cell,dashed] (b) {};

    \foreach \i in {1,2,3,4,5}
        \node[cell,thick] (m\i) {\i};
        
    \foreach \i in {0,0,0}
        \node[cell,thick] {\i};

    \node[cell,thick] {0};
    \node[cell,thick] (e) {0};

    \node[below of=b,yshift=-10mm] (p) {{\tt p}};
    \path [arrow,->] (p) edge node {} (b.south east);

    \node[below of=e,yshift=-10mm] (q) {{\tt q}};
    \path [arrow,->] (q) edge node {} (e.south east);

    \node[cell,dashed,color=red] {};

    \foreach \i in {2,3}
        \node[cell,dashed] {};
\end{tikzpicture}
\end{center}
\end{frame}

\begin{frame}[fragile]{Возрат указателя из функции}
Функция для поиска максимума в массиве:
\begin{lstlisting}
int max_element (int * p, int * q) {
    int max = *p;
    for (; p != q; ++p)
        if (*p > max)
            max = *p;

    return max;
}
\end{lstlisting}
\begin{lstlisting}
    int m[10] = {...};
    int max = max_element(m, m + 10); 
    cout << "Maximum = " << max << endl; 
\end{lstlisting}
\end{frame}

\begin{frame}[fragile]{Возрат указателя из функции}
Функция для поиска максимума в массиве:
\begin{lstlisting}
int * max_element (int * p, int * q) {
    int * pmax = p;
    for (; p != q; ++p)
        if (*p > *pmax)
            pmax = p;

    return pmax;
}
\end{lstlisting}
\begin{lstlisting}
    int m[10] = {...};
    int * pmax = max_element(m, m + 10); 
    cout << "Maximum = " << *pmax << endl; 
\end{lstlisting}
\end{frame}

\begin{frame}[fragile]{Возрат значения через указатель}
Функция для поиска максимума в массиве:
\begin{lstlisting}
bool max_element (int * p, int * q, int * res) {
    if (p == q)
        return false;
    *res = *p;
    for (; p != q; ++p)
        if (*p > *res)
            *res = *p;
    return true;
}
\end{lstlisting}
\begin{lstlisting}
    int m[10] = {...};
    int max = 0;
    if (max_element(m, m + 10, &max)) 
        cout << "Maximum = " << max << endl; 
\end{lstlisting}
\end{frame}

\begin{frame}[fragile]{Возрат значения через указатель на указатель}
Функция для поиска максимума в массиве:
\begin{lstlisting}
bool max_element (int * p, int * q, int ** res) {
    if (p == q)
        return false;
    *res = p;
    for (; p != q; ++p)
        if (*p > **res)
            *res = p;
    return true;
}
\end{lstlisting}
\begin{lstlisting}
    int m[10] = {...};
    int * pmax = 0;
    if (max_element(m, m + 10, &pmax)) 
        cout << "Maximum = " << *pmax << endl; 
\end{lstlisting}
\end{frame}

\end{document}
