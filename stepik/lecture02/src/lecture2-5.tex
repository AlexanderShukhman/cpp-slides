\documentclass{beamer}

\usepackage{mooc}

\usetikzlibrary{arrows, decorations.markings, shapes}
\usetikzlibrary{positioning}

% The face style, can be changed

\title{{\bf Программирование на языке \langcpp\protect\\Лекция
2\protect\vspace{1em}\\}Ссылки}

\begin{document}
\begin{frame} 
  \titlepage
\end{frame}

\begin{frame}[fragile]{Недостатки указателей}
    \begin{itemize}
        \item Использование указателей синтаксически загрязняет код
            и усложняет его понимание.
            (Приходится использовать операторы \code{*} и \code{\&}.)

        \item Указатели могут быть неинициализированными (некорректный код).

        \item Указатель может быть нулевым (корректный код), 
            а значит указатель нужно проверять на равенство нулю.

        \item Арифметика указателей может сделать из корректного указателя
            некорректный (легко промахнуться).
    \end{itemize}
\end{frame}

\begin{frame}[fragile]{Ссылки}
    \begin{itemize}
        \item Для того, чтобы исправить некоторые недостатки указателей,
            в \langcpp введены ссылки.

        \item Ссылки являются ``красивой обёрткой'' над указателями:
\begin{lstlisting}
void swap (int & a, int & b) {
    int t = b;
    b = a;
    a = t;
}

int main() {
    int k = 10, m = 20;
    swap (k, m);
    cout << k << ' ' << m << endl; // 20 10
    return 0;
}
\end{lstlisting}
    \end{itemize}
\end{frame}

\begin{frame}[fragile]{Различия ссылок и указателей}
    \begin{itemize}
        \item Ссылка не может быть неинициализированной.
\begin{lstlisting}
int * p; // OK
int & l; // ошибка
\end{lstlisting}

        \item У ссылки нет нулевого значения.            
\begin{lstlisting}
int * p = 0; // OK
int & l = 0; // ошибка
\end{lstlisting}
            
    \item Ссылку нельзя переинициализировать.
\begin{lstlisting}
int a = 10, b = 20;
int * p = &a; // p указывает на a
p = &b;       // p указывает на b
int & l = a;  // l ссылается на a
l = b;        // a присваивается значение b
\end{lstlisting}
\end{itemize}
\end{frame}

\begin{frame}[fragile]{Различия ссылок и указателей}
    \begin{itemize}
        \item Нельзя получить адрес ссылки или ссылку на ссылку.
\begin{lstlisting}
int a = 10;
int * p = &a;  // p указывает на a
int ** pp = &p;// pp указывает на переменную p
int & l = a;   // l ссылается на a
int * pl = &l; // pl указывает на переменную a
int && ll = l; // ошибка
\end{lstlisting}

        \item Нельзя создавать массивы ссылок.
\begin{lstlisting}
int * mp[10] = {};  // массив указателей на int
int & ml[10] = {};  // ошибка
\end{lstlisting}

        \item Для ссылок нет арифметики.

    \end{itemize}
\end{frame}

\begin{frame}[fragile]{lvalue и rvalue}
    \begin{itemize}
        \item Выражения в \langcpp можно разделить на два типа:
            \begin{enumerate}
                \item {\bf lvalue}~--- выражения, значения которых являются
                    {\em ссылкой} на переменную/элемент массива, 
                    а значит могут быть указаны слева от оператора \code{=}.
                    
                \item {\bf rvalue}~--- выражения, значения которых являются
                    временными и не соответствуют никакой переменной/элементу
                    массива.
            \end{enumerate}

        \item Указатели и ссылки могут указывать только на lvalue.
\begin{lstlisting}
int a = 10, b = 20;
int m[10] = {1,2,3,4,5,5,4,3,2,1};
int & l1 = a;       // OK
int & l2 = a + b;   // ошибка
int & l3 = *(m + a / 2);        // OK
int & l4 = *(m + a / 2) + 1;    // ошибка
int & l5 = (a + b > 10) ? a : b;  // OK
\end{lstlisting}
\end{itemize}
\end{frame}

\begin{frame}[fragile]{Время жизни переменной}
    Следует следить за временем жизни переменных.
\begin{lstlisting}
int * foo() {
    int a = 10;
    return &a;
}

int & bar() {
    int b = 20;
    return b;
}
\end{lstlisting}
\begin{lstlisting}
    int * p = foo();
    int & l = bar();
\end{lstlisting}

\end{frame}

\end{document}
