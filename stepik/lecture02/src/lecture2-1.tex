\documentclass{beamer}

\usepackage{mooc}

\usetikzlibrary{arrows, decorations.markings, shapes}
\usetikzlibrary{positioning}

% The face style, can be changed

\title{{\bf Программирование на языке \langcpp\protect\\Лекция
2\protect\vspace{1em}\\}Как выполняются программы на \langcpp}

\begin{document}
\begin{frame} 
  \titlepage
\end{frame}

\begin{frame}[fragile]{Архитектура фон Неймана}
Современных компьютеры построены по принципам 
архитектуры фон Неймана:
\begin{enumerate}
    \item {\bf Принцип однородности памяти.}\\
        Команды и данные хранятся в одной и той же памяти и внешне в памяти неразличимы.
    \item {\bf Принцип адресности.}\\
        Память состоит из пронумерованных ячеек.
    \item {\bf Принцип программного управления.}\\
        Все вычисления представляются в виде последовательности команд.
    \item {\bf Принцип двоичного кодирования.}\\
        Вся информация (данные и команды) кодируются двоичными числами.
\end{enumerate}
\end{frame}

\begin{frame}[fragile]{Сегментация памяти}
    \begin{itemize}
        \item Оперативная память, используемая в программе на C++,
            разделена на области двух типов:
            \begin{enumerate}
                \item сегменты данных,
                \item сегменты кода (текстовые сегменты).
            \end{enumerate}

        \item В сегментах кода содержится код программы.

        \item В сегментах данных располагаются данные программы 
            (значения переменных, массивы и пр.).

        \item При запуске программы выделяются два сегмента данных:
            \begin{enumerate}
                \item сегмент глобальных данных,
                \item стек.
            \end{enumerate}

        \item В процессе работы программы могут выделяться и освобождаться
            дополнительные сегменты памяти.

        \item Обращения к адресу вне выделенных сегментов~--- ошибка времени
            выполнения (access violation, segmentation fault).
    \end{itemize}
\end{frame}

\begin{frame}[fragile]{Как выполняется программа?}
    \begin{itemize}
        \item Каждой функции в скомпилированном коде соответствует отдельная
            секция.
            
        \item Адрес начала такой секции — это адрес функции.

        \item Телу функции соответствует последовательность команд процессора.
            
        \item Работа с данными происходит на уровне байт,
            информация о типах отсутствует.

        \item В процессе выполнения адрес следующей инструкции хранится 
            в специальном регистре процессора IP (Instruction Pointer).

        \item Команды выполняются последовательно, пока не встретится
            специальная команда (например, условный переход или вызов 
            функции), которая изменит IP.
    \end{itemize}
\end{frame}

\begin{frame}[fragile]{Ещё раз о линковке}
    \begin{itemize}
        \item На этапе компиляции объектных файлов
            в места вызова функций подставляются имена функций.

        \item На этапе линковки в места вызова вместо имён функций
            подставляются их адреса.

        \item Ошибки линковки:
            \begin{enumerate}
                \item {\tt undefined reference}\\
                    Функция имеет объявление, но не имеет тела.

                \item {\tt multiple definition}\\
                    Функция имеет два или более определений.
            \end{enumerate}

        \item Наиболее распространённый способ получить 
            {\tt multiple definition}~--- определить функцию
            в заголовочном файле, который включён в несколько
            {\tt .cpp} файлов.
    \end{itemize}
\end{frame}
\end{document}
