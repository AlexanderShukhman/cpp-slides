\documentclass[aspectration=1610,t]{beamer}

\usepackage{mooc}

\title{{\bf Программирование на языке \langcpp\protect\\Лекция
9\protect\vspace{1em}\\}Стандартная библиотека шаблонов}

\begin{document}
\begin{frame} 
  \titlepage
\end{frame}

\begin{frame}{STL: введение}
    \begin{itemize}
        \item STL = Standard Template Library.
        
        \item STL является частью стандартной библиотеки \langcpp, описана в стандарте, но не упоминается там явно.
            
        \item Авторы: Александр Степанов, Дэвид Муссер и Менг Ли\\ 
            (сначала для HP, а потом для SGI).

        \item Основана на разработках для языка Ада.
    \end{itemize}
\pause
\begin{block}{Основные составляющие}
    \begin{itemize}
        \item контейнеры (хранение объектов в памяти),
        \item итераторы (доступ к элементам контейнера),
        \item адаптеры (обёртки над контейнерами),
        \item алгоритмы (для работы с последовательностями),
        \item функциональные объекты, функторы (обобщение функций).
    \end{itemize}
\end{block}
\end{frame}

\begin{frame}{Преимущества стандартной библиотеки}{}
    \begin{itemize}
        \item стандартизированность,
        \item общедоступность,
        \item эффективность,
        \item общеизвестность,
        \item \dots
    \end{itemize}
\pause
\begin{block}{Чего нет в стандартной библиотеке?}
    \begin{itemize}
        \item сложных структур данных,
        \item сложных алгоритмов,
        \item работы с графикой/звуком,
        \item ...
    \end{itemize}
\end{block}
\end{frame}


\end{document}


